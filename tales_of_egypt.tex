The Project Gutenberg eBook of Tales of Secret Egypt

This ebook is for the use of anyone anywhere in the United States and
most other parts of the world at no cost and with almost no restrictions
whatsoever. You may copy it, give it away or re-use it under the terms
of the Project Gutenberg License included with this ebook or online
at www.gutenberg.org. If you are not located in the United States,
you will have to check the laws of the country where you are located
before using this eBook.

Title: Tales of Secret Egypt

Author: Sax Rohmer

Release date: June 29, 2012 [eBook #40108]
                Most recently updated: August 20, 2024

Language: English

Credits: Produced by Stephen Hutcheson, Christoph W. Kluge, Rod
        Crawford, Dave Morgan and the Online Distributed
        Proofreading Team at http://www.pgdp.net


*** START OF THE PROJECT GUTENBERG EBOOK TALES OF SECRET EGYPT ***
  Transcriber's Note:

  Obvious mis-spellings and printing errors have been
  corrected; a list is included at the end of this e-book.

  Missing periods at paragraph-end have silently been
  supplied. Inconsistencies in hyphenation and accentuation
  have been retained.

  The romanization of Arabic is the author's.




  [Illustration: "She stood there ... her slim body swaying
  in a perfect rapture of admiration for her own beauty."]




  TALES OF
  SECRET EGYPT

  BY SAX ROHMER



  McKINLAY, STONE & MACKENZIE
  NEW YORK




  _Printed in the United States of America_

  _Second Printing
  November, 1920_

  Published February, 1919




  CONTENTS


  PART I

  TALES OF ABÛ TABÂH

  CHAPTER                             PAGE

    I. THE YASHMAK OF PEARLS             1
   II. THE DEATH-RING OF SNEFERU        31
  III. THE LADY OF THE LATTICE          58
   IV. OMAR OF ISPAHÂN                  87
    V. BREATH OF ALLAH                 114
   VI. THE WHISPERING MUMMY            144


  PART II

  OTHER TALES

  CHAPTER                             PAGE

    I. LORD OF THE JACKALS             169
   II. LURE OF SOULS                   194
  III. THE SECRET OF ISMAIL            216
   IV. HARÛN PASHA                     239
    V. IN THE VALLEY OF THE SORCERESS  267
   VI. POMEGRANATE FLOWER              290




TALES OF SECRET EGYPT




PART I

TALES OF ABÛ TABÂH




I

THE YASHMAK OF PEARLS


The _duhr_, or noonday call to prayer, had just sounded from the
minarets of the Mosques of Kalaûn and En-Nasîr, and I was idly noting
the negligible effect of the _adan_ upon the occupants of the
neighboring shops--coppersmiths for the most part--when suddenly my
errant attention became arrested.

A mendicant of unwholesome aspect crouched in the shadow of the narrow
gateway at the entrance to the Sûk es-Saîgh, or gold and silver
bazaar, having his one serviceable eye fixed in a malevolent stare
upon something or someone immediately behind me.

It is part and parcel of my difficult profession to subdue all
impulses and to think before acting. I sipped my coffee and selected
a fresh cigarette from the silver box upon the rug beside me. In this
interval I had decided that the one-eyed mendicant cherished in his
bosom an implacable and murderous hatred for my genial friend, Ali
Mohammed, the dealer in antiques; that he was unaware of my having
divined his bloody secret; and that if I would profit by my accidental
discovery, I must continue to feign complete ignorance of it.

Turning casually to Ali Mohammed, I was startled to observe the
expression upon his usually immobile face: he was positively gray,
and I thought I detected a faint rattling sound, apparently produced
by his teeth; his eyes were set as if by hypnosis upon the uncleanly
figure huddled in the shadow of the low gate.

"You are unwell, my friend," I said.

Ali Mohammed shook his head feebly, removed his eyes by a palpable
effort from the watcher in the gateway, but almost instantly reverted
again to that fixed and terrified scrutiny.

"Not at all, Kernaby Pasha," he chattered; "not in the least."

He passed a hand rapidly over a brow wet with perspiration, and
moistened his lips, which were correspondingly dry. I determined upon
a diplomatic _tour de force_; I looked him squarely in the face.

"For some reason," I said distinctly, "you are in deadly fear of the
wall-eyed mendicant who is sitting by the gate of the Sûk es-Saîgh,
O Ali Mohammed, my friend."

I turned with assumed carelessness. The beggar of murderous appearance
had vanished, and Ali Mohammed was slowly recovering his composure.
I knew that I must act quickly, or he would deny with the urbane
mendacity of the Egyptian all knowledge of the one-eyed one;
therefore--

"Acquaint me with the reason of your apprehensions," I said, at the
same time offering him one of his own cigarettes; "it may be that I
can assist you."

A moment he hesitated, glancing doubtfully in the direction of the
gate and back to my face; then--

"It is one of the people of Tîr," he whispered, bending close to my
ear; "of the evil _ginn_ who are the creatures of Abû Tabâh."

I was puzzled and expressed my doubt in words.

"Alas," replied Ali Mohammed, "the Imám Abû Tabâh is neither a man nor
an official; he is a magician."

"Indeed! then you speak of one bearing the curious name of Abû Tabâh,
who is at once the holder of a holy office and also one who has
dealings with the _ginn_ and the _Efreets_. This is strange, Ali
Mohammed, my friend."

"It is strange and terrible," he whispered, "and I fear that my path
is beset with pitfalls and slopeth down to desolation." He pronounced
the _Takbîr_, "_Alláhu akbar!_" and uttered the words "_Hadeed! yá
mashûm!_" (Iron! thou unlucky!), a potent invocation, as the _ginn's_
dread of that metal is well known. "There are things of which one may
not speak," he declared; "and this is one of them."

Sorely puzzled as I was by this most mysterious happening, yet,
because of the pious words of my friend, I knew that the incident was
closed so far as confidences were concerned; and I presently took my
departure, my mind filled with all sorts of odd conjectures by which I
sought to explain the matter. I was used to the superstitions of that
quarter where almost every gate and every second street has its
guardian _ginnee_, but who and what was Abû Tabâh? An Imám,
apparently, though to what mosque attached Ali Mohammed had not
mentioned. And why did Ali Mohammed fear Abû Tabâh?

So my thoughts ran, more or less ungoverned, whilst I made my way
through streets narrow and tortuous in the direction of the Rondpoint
du Mûski. I saw no more of the wall-eyed mendicant; but in a court
hard by the Mosque of el-Ashraf I found myself in the midst of a
squabbling crowd of natives surrounding someone whom I gathered, from
the direction of their downward glances, to be prone upon the ground.
Since the byways of the Sûk el-Attârin are little frequented by
Europeans, at midday, I thrust my way into the heart of the throng,
thinking that some stray patron of Messrs. Cook and Son (Egypt, Ltd.)
might possibly have got into trouble or have been overcome by the
heat.

Who or what lay at the heart of that gathering I never learned. I was
still some distance from the centre of the disturbance when an
evil-smelling sack was whipped over my head and shoulders from behind,
a hand clapped upon my mouth and jaws; and, lifted in muscular arms,
I found myself being borne inarticulate down stone steps, as I gathered
from the sound, into some cool cellar-like place.


II

In my capacity as Egyptian representative of Messrs. Moses, Murphy &
Co., of Birmingham, I have sometimes found myself in awkward corners;
but in Cairo, whether the native or European quarter, I had hitherto
counted myself as safe as in London and safer than in Paris. The
unexpectedness of the present outrage would have been sufficient to
take my breath away without the agency of the filthy sack, which had
apparently contained garlic at some time and now contained my head.

I was deposited upon a stone-paved floor and my wrists were neatly
pinioned behind me by one of my captors, whilst another hung on to my
ankles. The sack was raised from my body but not from my face; and
whilst a hand was kept firmly pressed over the region of my mouth,
nimble fingers turned my pockets inside out. I assumed at first that I
had fallen into the clutches of some modern brethren of the famous
Forty, but when my purse, note-case, pocket-book, and other belongings
were returned to me, I realized that something more underlay this
attempt than the mere activity of a gang of footpads.

At this conclusion I had just arrived when the stinking sack was
pulled off entirely and I found myself sitting on the floor of a small
and very dark cellar. Beside me, holding the sack in his huge hands,
stood a pock-marked negro of most repulsive appearance, and before me,
his slim, ivory-colored hands crossed and resting upon the head of an
ebony cane, was a man, apparently an Egyptian, whose appearance had
something so strange about it that the angry words which I had been
prepared to utter died upon my tongue and I sat staring mutely into
the face of my captor; for I could not doubt that the outrage had been
dictated by this man's will.

He was, then, a young man, probably under thirty, with perfectly
chiseled features and a slight black moustache. He wore a black
_gibbeh_, and a white turban, and brown shoes upon his small feet. His
face was that of an ascetic, nor had I ever seen more wonderful and
liquid eyes; in them reposed a world of melancholy; yet his red lips
were parted in a smile tender as that of a mother. Inclining his head
in a gesture of gentle dignity, this man--whom I hated at
sight--addressed me in Arabic.

"I am desolated," he said, "and there is no comfort in my heart
because of that which has happened to you by my orders. If it is
possible for me to recompense you by any means within my power,
command and you shall find a slave."

He was poisonously suave. Beneath the placid exterior, beneath the
sugar-lipped utterances, in the deeps of the gazelle-like eyes, was
hid a cold and remorseless spirit for which the man's silken demeanor
was but a cloak. I hated him more and more. But my trade--for I do not
blush to own myself a tradesman--has taught me caution. My ankles were
free, it is true, but my hands were still tied behind me and over me
towered the hideous bulk of the negro. This might be modern Cairo, and
no doubt there were British troops quartered at the Citadel and at
the Kasr en-Nîl; probably there was a native policeman, a
representative of twentieth-century law and order, somewhere in the
maze of streets surrounding me: but, in the first place, I was at a
physical disadvantage, in the second place I had reasons for not
desiring unduly to intrude my affairs upon official notice, and in the
third place some hazy idea of what might be behind all this business
had begun to creep into my mind.

"Have I the pleasure," I said, and electing to speak, not in Arabic
but in English, "of addressing the _Imám Abû Tabâh_?"

I could have sworn that despite his amazing self-control the man
started slightly; but the lapse, if lapse it were, was but momentary.
He repeated the dignified obeisance of the head--and answered me in
English as pure as my own.

"I am called Abû Tabâh," he said; "and if I assure you that my
discourteous treatment was dictated by a mistaken idea of duty, and if
I offer you this explanation as the only apology possible, will you
permit me to untie your hands and call an _arabîyeh_ to drive you to
your hotel?"

"No apology is necessary," I assured him. "Had I returned direct to
Shepheard's I should have arrived too early for luncheon; and the odor
of garlic, which informed the sack that your zeal for duty caused to
be clapped upon my head, is one for which I have a certain penchant if
it does not amount to a passion."

Abû Tabâh smiled, inclined his head again, and slightly raising the
ebony cane indicated my pinioned wrists, at the same time glancing at
the negro. In a trice I was unbound and once more upon my feet. I
looked at the dilapidated door which gave access to the cellar, and I
made a rapid mental calculation of the approximate weight in pounds of
the large negro; then I looked hard at Abû Tabâh--who smilingly met my
glance.

"Any one of my servants," he said urbanely, "who wait in the adjoining
room, will order you an _arabîyeh_."


III

When the card of Ali Mohammed was brought to me that evening, my
thoughts instantly flew to the wall-eyed mendicant of the Sûk
en-Nahhasîn, and to Abû Tabâh, the sugar-lipped. I left the pleasant
company of the two charming American ladies with whom I had been
chatting on the terrace and joined Ali Mohammed in the lounge.

Without undue preamble he poured his tale of woe into my sympathetic
ears. He had been lured away from his shop later that afternoon, and,
in his absence, someone had ransacked the place from floor to roof.
That night on his way to his abode, somewhere out Shubra direction I
understood, he had been attacked and searched, finally to reach his
house and to find there a home in wild disorder.

"I fear for my life," he whispered and glanced about the lounge in
blackest apprehension; "yet where in all Cairo may I find an
intermediary whom I can trust? Suppose," he pursued, and dropped his
voice yet lower, "that a commission of ten per cent--say, one hundred
pounds, English--were to be earned, should you care, Kernaby Pasha,
to earn it?"

I assured him that I should regard such a proposal with the utmost
affection.

"It would be necessary," he continued, "for you to disguise yourself
as an aged woman and to visit the _harêm_ of a certain wealthy Bey.
I have a ring which must be shown to the _bowwab_ at the gate of the
_harêm_ gardens upon which you would knock three times slowly and then
twice rapidly. You would collect the thousand _ginêh_ agreed upon and
would deliver to a certain lady a sandalwood box, the possession of
which endangers my life and has brought about me the hosts of Abû
Tabâh the magician."

So the head of the cat was out of the bag at last. But there was more
to come and it was not a proposition to plunge at, as I immediately
perceived; and I parted from Ali Mohammed upon the prudent
understanding that I should acquaint him with my decision on the
morrow.

The terrace of Shepheard's was deserted, when, having escorted my
visitor to the door, he made his way down into the Shâria Kâmel Pasha.
Two white-robed figures who looked like hotel servants, and a little
nondescript group of natives, stood at the foot of the steps. At the
instant that doubt entered my mind and too late to warn the worthy
Ali Mohammed, the group parted to give him passage; then ... a
terrific scuffle was in progress and one of the wealthiest merchants
of the Mûski was being badly hustled.

I ran down the steps, the carriage-despatcher and some other
officials, whom the disturbance had aroused from their secret lairs,
appearing almost simultaneously. As I reached the street, out from the
feet of the wrestling throng, like a football from a scrum, rolled a
neat _tarbûsh_.

Automatically I stooped and picked it up. Its weight surprised me.
Then, glancing inside the _tarbûsh_, I perceived that a little oblong
box, together with a quaint signet ring, were ingeniously attached to
the crown by means of silk threads tied around the knot of the tassel.
I glanced rapidly about me. I, alone, had seen the cap roll out upon
the pavement.

A hard jerk, and I had the box and the ring free in my hand. The tall
carriage-despatcher, his ferocious efforts now seconded by a native
policeman who freely employed his cane upon the thinly-clad persons of
the group, had terminated the scuffle.

Right and left active figures darted, pursued for some little distance
by the policeman and the two men from the hotel. There were no
captures.

A very dusty and bemused Ali Mohammed, his shaven skull robbing him of
much of the dignity which belonged to his _tarbûsh_, confronted me,
ruefully dusting his garments.

"Your _tarbûsh_, my friend," I said, restoring his property to him
with a bow.

One piercing glance he cast into the interior, then--

"O Allah!" he wailed--"O Allah! I am robbed! Yet----"

A sort of martyred resignation, a beatific peace, crept over his
features.

"To war against Abû Tabâh is the act of a fool," he declared. "To have
obtained the Bey's money would have been good, but to have obtained
peace is better!"


IV

I awoke that night from a troubled sleep and from a dream wherein
magnetic fingers caressed my forehead hypnotically. For a moment I
could not believe that I was truly awake; the long ivory hand of my
dreams was still moving close before me with a sort of slow fanning
movement--and other, nimble, fingers crept beneath my pillow!

Of my distaste for impulse I have already spoken, and even now, with
my mind not wholly under control, I profited by those years of
self-imposed discipline. Without fully opening my eyes, cautiously,
inch by inch, I moved my hand to that side of the bed nearer to the
wall, where there reposed a leather holster containing my pistol.

My fingers closed over the butt of the weapon; and in a flash I
became wide awake ... and had the ring of the barrel within an inch of
the smiling face of Abû Tabâh!

I sat up.

"Be good enough, my friend," I said, "to turn on the center lamp. The
switch, as you have probably noted, is immediately to the left of the
door."

Abû Tabâh, straightening his figure and withdrawing his hand from
beneath my pillow, inclined his picturesque head in grave salute and
moved stately in the direction indicated. The room was flooded with
yellow light. Its disorder was appalling; apparently no item of my
gear had escaped attention.

"Pray take a seat," I said; "this one close beside me."

Abû Tabâh gravely accepted the invitation.

"This is the second occasion," I continued, "upon which you have
unwarrantably submitted me to a peculiar form of outrage----"

"Not unwarrantably," replied Abû Tabâh, his speech suave and gentle;
"but I fear I am too late!"

His words came as a beam of enlightenment. At last I had the game in
my hands did I but play my cards with moderate cunning.

"You must pursue your inquiries in the _harêm_ of the Bey," I said.

Abû Tabâh shrugged his shoulders.

"The house of Yûssuf Bey has been watched," he replied; "therefore my
agents have failed me and must be punished."

"They are guiltless. It was humanly impossible to perceive my entrance
to the house," I declared truthfully.

Abû Tabâh smiled into my face.

"So it was _you_ who carried the sacred _burko_ of the Seyyîdeh
Nefîseh," he said; "and to-night Ali Mohammed brought you the reward
for your perilous journey."

"Your reasoning is sound," I replied, "and the accuracy of your
information remarkable."

I had scored the first point in the game; for I had learned that the
wonderful silken _yashmak_, pearl embroidered, which I had found in
the sandalwood box, was no less a curiosity than the face-veil of the
Seyyîdeh Nefîseh and must therefore be of truly astounding antiquity
and unique of its kind.

"The woman Sháhmarâh," continued my midnight visitor, the eerie light
of fanaticism dawning in his eyes, "who was once a dancing girl, and
who will ruin Yûssuf Bey as she ruined Ghûri Pasha before him, must be
for ever accursed and meet with the fate of courtesans if she dare to
wear the _burko_ of Nefîseh."

I had scored my second point; I had learned that the lady to whom Ali
Mohammed would have had me deliver the _yashmak_ was named Sháhmarâh
and was evidently the favorite of the notorious Yûssuf Bey. The
complacent self-satisfaction of Abû Tabâh amused me vastly, for he
clearly entertained no doubts respecting his efficiency as a searcher.

He was watching me now with his strange hypnotic eyes, which had
softened again, and his fixed stare caused me a certain uneasiness.
For a captured thief, sitting covered by the pistol of his captor,
he was ridiculously composed.

"You have performed an immoral deed," he said sweetly, "and have
pandered to the base desires of a woman of poor repute. I offer you an
opportunity of performing a good deed--and of trebling your profit."

This was as I would have it, and I nodded encouragingly.

"Unfold to me the thing that is in your mind," I directed him.

"I am a Moslem," he said; "and although Yûssuf Bey is a dog of dogs,
he is nevertheless a True Believer--and I may not force my way into
his _harêm_."

"He might return the veil if he knew that Sháhmarâh had it,"
I suggested ingenuously.

Abû Tabâh shook his head.

"There are difficulties," he replied, "and if the theft is not to
be proclaimed to the world, there is no time to be lost. This is my
proposal: Return to the woman Sháhmarâh, and acquaint her with the
fact that the sacred veil has been traced to her abode and her death
decided upon by the Grand Mufti if it be not given up. Force the
merchant Ali Mohammed to return the money received by him, using the
same threat--which will prove a talisman of power. Return to the
infidel woman the full amount; I will make good your commission,
to which, if you be successful, I will add two hundred pounds."

I performed some rapid thinking.

"You must give me a little time to consider this matter," I said.

Abû Tabâh graciously inclined his head.

"On Tuesday next a company of holy men who have journeyed hither from
Ispahân, go to view this relic; you have therefore five days to act."

"And if I decline?"

Abû Tabâh shrugged his shoulders.

"The loss must be made known--it would be a great scandal; the
merchant Ali Mohammed, and the woman, Sháhmarâh, must be
arrested--very undesirable; _you_ must be arrested--most undesirable;
and your banking account will be poorer by three hundred pounds."

"Frightfully undesirable," I declared. "But suppose I strike the first
blow and give you in charge of the police here and now?"

"You may try the experiment," he said.

I waved my hand in the direction of the door (I had reasons for
remaining in bed). "_Ma'salâma!_ (Good-bye)," I said. "Don't stay to
restore the room to order. I shall expect you early in the morning.
You will find the door of the hotel open any time after eight and I
can highly recommend it as a mode of entrance."

Having saluted me with both hands, Abû Tabâh made his stately
departure, leaving me much exercised in mind as to how he proposed to
account to the _bowwab_ for his sudden appearance in the building.
This, however, was no affair of mine, and, first reclosing the window,
I unfastened from around my left ankle the sandalwood box and the ring
which I had bound there by a piece of tape--a device to which I owed
their preservation from the subtle fingers of Abû Tabâh. Furthermore,
to their presence there I owed my having awakened when I did. I am
persuaded that the mysterious Egyptian's passes would have continued
to keep me in a profound sleep had it not been for the pain occasioned
by the pressure of the tape.

Opening the sandalwood box, and then the silver one which it enclosed,
I re-examined the really wonderful specimen of embroidery whereof they
formed the reliquary. The _burko_ was of Tussur silk, its texture so
fine that the whole veil, which was some four feet long by two wide,
might have been passed through the finger ring and would readily be
concealed in the palm of the hand.

It was of unusual form, having no forehead band, more nearly
resembling a _yashmak_ than a true _burko_, and was heavily
embroidered with pearls of varying sizes and purity, although none of
them were large. Its intrinsic value was considerable, but in view of
its history such a valuation must have fallen far below the true one.
When its loss became known, I estimated that Messrs. Moses, Murphy &
Co. could readily dispose of three duplicates through various channels
to wealthy collectors whose enthusiasms were greater than their
morality. The sale to a museum, or to the lawful owners, of the
original (known technically as "the model") would crown a sound
commercial transaction.

Cock-crow that morning discovered me at the private residence, in the
Boulevard Clot-Bey, of one Suleyman Levi, with whom I had had minor
dealings in the past.


V

At nine o'clock on the following Monday night, an old Egyptian woman,
enveloped from head to foot in a black _tôb_ and wearing a black crêpe
face-veil boasting a hideous brass nose-piece, halted before a doorway
set in the wall guarding the great gardens of the palace of Yûssuf
Bey. I was the impersonator of this decrepit female. Abû Tabâh, who
thus far had accompanied me, stepped into the dense shadow of the
opposite wall and was thereby swallowed up.

I rapped three times slowly upon the doorway, then twice rapidly.
Almost at once a little wicket therein flew open, and a bloated negro
face showed framed in the square aperture.

"The messenger from Ali Mohammed of the Sûk en-Nahhasîn," I said, in
a croaky voice. "Conduct me to the Lady Sháhmarâh."

"Show her seal," answered the eunuch, extending through the opening
a large, fat hand.

I gave him the ring so fortunately discovered in the _tarbûsh_ of my
friend the merchant and the hand was withdrawn. Within a colloquy took
place in which a female voice took part. Then the door was partly
opened for my admittance--and I found myself in the gardens of the
Bey.

In the moonlight it was a place of wonder, an enchanted demesne; but
more like an Edmond Dulac water-color than a real garden. The palace
with its magnificent _mushrabîyeh_ windows, so poetically symbolical
of veiled women, guarded by several fine, straight-limbed palm trees,
spoke of the Old Cairo which saw the birth of _The Arabian Nights_
and which so many of us imagine to have vanished with the _khalîfate_.

A girl completely muffled up in many-hued shawls and scarves, so that
her red-slippered feet and two bright eyes heavily darkened with
_kohl_ were the only two portions of her person visible, stood before
me, her figure seeming childish beside that of the gross negro--whom
I hated at sight because he reminded me of the one whom I had
encountered in Abû Tabâh's cellar.

"Follow me, quickly, mother," said the girl. "You"--pointing
imperiously at the black man--"remain here."

I followed her in silence, noting that she pursued a path which ran
parallel with the wall and lay wholly in its shadow. The gardens were
fragrant with the perfume of roses, and in the center was a huge
marble fountain surrounded by kiosks projecting into the water, tall
acacias overshadowing them. We skirted two sides of the palace, its
_mushrabîyeh_ windows mysteriously lighted by the moon but showing no
illumination from within. There we came to the entrance to a kind of
trellis-covered walk, mosaic paved and patched delightfully with
mystic light. It terminated before a small but heavy and nail-studded
door, of which my guide held the key.

Entering, whilst she held the door ajar, I found myself in utter
darkness, to be almost immediately dispelled by the yellow gleam of
a lamp which the girl took from some niche, wherein, already lighted,
it had been concealed. Up a flight of bare wooden stairs she conducted
me, and opened a second prison-like door at their head. Leaving the
lamp upon the top step, she pushed me gently forward into a small,
octagonal room, paneled in dark wood inlaid with mother-o'-pearl and
reminding me of the interior of a magnified _kursee_ or coffee table.

Rugs and carpets strewed the floor and the air was heavy with the
smell of musk, a perfume which I detest, it having characterized the
personality of a certain Arab lady who sold me so marvelous a Damascus
scimitar that I was utterly deceived by it until too late.

Raising a heavy curtain draped in a door shaped like an old-fashioned
keyhole, and embellished with an intricate mass of fretwork carving,
my guide went out, leaving me alone with my reflections. This interval
was very brief, however, and was terminated by the reappearance of the
girl, who this time made her entrance through a second doorway masked
by the paneling. A faint musical splashing sound greeted me through
the opening; and when my guide beckoned me to enter and I obeyed,
I found myself in a chamber of barbaric beauty and in the presence
of the celebrated Sháhmarâh.

The apartment, save for one end being wholly occupied by a magnificent
_mushrabîyeh_ screen, was walled with what looked like Verde Antico
marble or green serpentine. An ebony couch having feet shaped as those
of a leopard and enriched with gleaming bronze, having the skins of
leopards cast across it, and, upon the skins, silken soft cushions
wrought in patterns of green and gold, stood upon the mosaic floor at
the head of three shallow steps which descended to a pool where a
fountain played, softly musical; wherein lurked gleaming shapes of
silver and gold. Bright mats were strewn around, and at one corner of
the pool a huge silver _mibkharah_ sent up its pencilings of aromatic
smoke.

Upon this couch Sháhmarâh reclined, and I perceived immediately that
her reputation for beauty was richly deserved. There was something
leopardine in her pliant shape, which seemed to harmonize with the
fierce black and gold of the skins upon which she was stretched; she
had the limbs of a Naiad and the eyes of an Egyptian Circe. Upon her
head she wore a _rabtah_, or turban, of pure white, secured and
decorated in front by a brooch of ancient Egyptian enamel-work
probably fourteenth dynasty, and for which I would gladly have given
her one hundred pounds. If I have forgotten what else she wore it may
be because my senses were in somewhat of a turmoil as I stood before
her in that opulent apartment--which I suddenly recognized, and not
without discomfiture, to be the _meslakh_ of the _hammám_. I can only
relate, then, that the image left upon my mind was one of jewels and
dusky peach-like loveliness. Jewels there were in abundance, clasped
about the warm curves of her arms and overloading her fingers; she
wore gold bands thickly encrusted with gems about her ankles (the slim
ankles of a dancing girl); and a fiery ruby of the true pigeon's-blood
color gleamed upon the first toe of her left foot, the nails of which
were highly manicured and stained with henna.

Fixing her wonderful eyes upon me--

"You have brought the veil?" she said.

"The merchant Ali Mohammed ordered me to convey to him the price
agreed upon, O jewel of Egypt," I mumbled, "ere I yielded up this a
poor man's only treasure."

Sháhmarâh sat upright upon the couch. Her delicate brows were drawn
together in a frown, and her eyes, rendered doubly luminous by the
pigment with which they were surrounded, glared fiercely at me, whilst
she stamped one bare foot upon a cushion lying on the mosaic floor.

"The veil!" she cried imperiously. "I will send the merchant Ali
Mohammed an order on the treasury of the Bey."

"O moon of the Orient," I replied, "O ravisher of souls, I am but a
poor ugly old woman basking in the radiance of beauty and loveliness.
Would you ruin one so old and feeble and helpless? I must have the
price agreed upon; let it be counted into this bag"--and concealing
my tell-tale hands as much as possible, I bent humbly and placed a
leather wallet upon a little table beside her which bore fruits,
sweetmeats, and a long-necked gold flagon. "When it is done, the
_yashmak_ of pearls, which only thy dazzling perfection might dare to
wear, shall be yielded up to thee, O daughter of musk and ambergris."

There fell a short silence, wherein the fountain musically plashed and
Sháhmarâh shot little inquiring glances laden with venom into the
mists of my black veil, and others which held a query over my shoulder
at her confidant.

"I might have you cast into a dungeon beneath this palace," she hissed
at me, bending lithely forward and extending a jeweled forefinger. "No
one would miss thee, O mother of afflictions."

"In that event," I crooned quaveringly, "O tree of pearls, the veil
could never be thine; for the merchant Ali Mohammed, who awaits me at
the gate, refuses to deliver it up until the price agreed upon has
been placed in his hands."

"He is a Jew, and a son of Jews, who eats without washing! a devourer
of pork, and an unclean insect," she cried.

She extended the jeweled hand towards the girl who stood behind me
and who, having loosened her wraps, proved to be a comely but
shrewd-looking Assyrian. "Let the money be counted into the bag,"
she ordered, "that we may be rid of the presence of this garrulous
and hideous old hag."

"O fountain of justice," I exclaimed; "O peerless _houri_, to behold
whom is to swoon with delight and rapture."

From a locked closet the Assyrian girl took a wooden coffer, and
before my gratified eyes began to count out upon the little table
notes and gold until a pile lay there to have choked a miser with
emotion. (The ready-money transactions of the East have always
delighted me.) But, with the chinking of the last piece of gold upon
the pile--

"There is no more," said the girl. "It is one hundred pounds short."

"It is more than enough!" cried Sháhmarâh. "I am ruined. Give me the
veil and go."

"O vision of paradise," I exclaimed in anguish, "the merchant Ali
Mohammed would never consent. In lieu of the remainder"--I pointed to
the antique enamel in her turban--"give me the brooch from thy
_rabtah_."

"O sink of corruption!" was her response, her whole body positively
quivering with rage, "it is not for thy filthy claws. Here!"--she
pulled a ring containing a fair-sized emerald from one of her fingers
and tossed it contemptuously upon the pile of money--"thou art more
than repaid. The veil! the veil!"

I turned to the girl who had counted out the gold.

"O minor moon, whom even the glory of paradise cannot dim," I said,
"put the money in the wallet, for my hands are old and infirm, and
give it to me."

The Assyrian scooped the gold and notes into the leather bag with
the utmost unconcern, and as though she had been shelling peas into
a basket. The profound disregard for wealth exhibited in the _harêm_
of Yûssuf Bey was extraordinary; and I mentally endorsed the opinion
expressed by Abû Tabâh that the ruin of the Bey was imminent.

Securing the heavy wallet to the girdle which I wore beneath my
veilings, I placed upon the table where the money had lain a small
silken packet.

"Here is the veil," I said; "for my story of the merchant, Ali
Mohammed, who had refused to yield it up, was but a stratagem to
test the generosity of thy soul, as thy refusal to give me the
price agreed upon was but a subterfuge to test my honesty."

Heedless of the words, Sháhmarâh snatched up the packet, tore off the
wrappings, and in a trice was standing upright before me wearing the
_yashmak_ of pearls.

I think I had never seen a figure more barbarically lovely than that
of this soulless Egyptian so adorned.

"My mirror, Sáfiyeh! my mirror!" she cried.

And the girl placing a big silver mirror in her hand, she stood there
looking into its surface, her wonderful eyes swimming with ecstasy and
her slim body swaying in a perfect rapture of admiration for her own
beauty.

Suddenly she dropped the mirror upon the cushions and threw wide her
arms.

"Am I not the fairest woman in Egypt?" she exclaimed. "I tread upon
the hearts of men and my power is above the power of kings!"

Then a subtle change crept over her features; and ere I could utter
the first of the honeyed compliments ready upon my tongue--

"Send Amineh to warn Mahmûd that the old woman is about to depart,"
she directed her attendant; and, turning to me: "Wait in the outer
room. Thy presence is loathsome to me, O mother of calamities."

"I hear and obey," I replied, "O pomegranate blossom"--and, following
the direction of her rigidly extended finger, I shuffled back to the
little octagonal apartment and the masked door was slammed almost upon
my heels.

This room, which possessed no windows, was solely illuminated by a
silken-shaded lantern, but I had not long to wait in that weird
half-light ere my conductress, again closely muffled in her shawls,
opened the door at the head of the steps and signed to me to descend.

"Lead the way, my beautiful daughter," I said; for I had no intention
of submitting myself to the risk of a dagger in the back.

She consented without demur, which served to allay my suspicions
somewhat, and in silence we went down the uncarpeted stairs and out
into the trellis-covered walk. The shadow beneath the high wall had
deepened and widened since we had last skirted the gardens, and I
felt my way along with my hand cautiously outstretched.

At a point within sight of the flower-grown arbor beneath which I
knew the gate to be concealed, my guide halted.

"I must return, mother," she said quickly. "There is the gate, and
Mahmûd will open it for you."

"Farewell, O daughter of the willow branch," I replied. "May Allah,
the Great, the Compassionate, be with thee, and may thou marry a
prince of Persia."

Light of foot she sped away, and, my forebodings coming to a sudden
climax, I crept forward with excessive caution, holding my clenched
hand immediately in front of my face--a device which experience of
the hospitable manners of the East had taught me.

It was well that I did so. Within three spaces of the gate a noose
fell accurately over my head and was drawn tight with a strangling
jerk!

But that it also encircled my upraised arm, its clasp must have
terminated my worldly affairs.

My assailant had sprung upon me from behind; and, in the fleeting
instant between the fall of the noose and its tightening, I turned
about ... and thrust the nose of my Colt repeater (which I grasped in
that protective upraised hand) fully into the grinning mouth of the
negro gate-keeper!

There was a rattle and gleam of falling ivory, for several of the
_bowwab's_ teeth had been dislodged by the steel barrel. Keeping the
weapon firmly thrust into the man's distended jaws, I circled around
him, whilst his hands relaxed their hold upon the strangling-cord,
and pushed him backward in the direction of the door.

"Open thou black son of offal!" I said, "or I will blow thee a cavity
as wide as thy blubber mouth through the back of that fat and greasy
neck! This was, no doubt, a stratagem of thy mistress to test my
fitness to be entrusted with large sums of money?"

When, a few moments later, I stood in the lane outside the gardens
of Yûssuf Bey, and felt with my hand the fat wallet at my waist,
I experienced a thrill of professional satisfaction, for had I not
successfully negotiated a duplicate veil, embroidered with imitation
pearls which the excellent Suleyman Levi by dint of four days of
almost ceaseless toil had made for me?...

From the shadows of the opposite wall Abû Tabâh stepped forth,
stately.

"Quick!" I said. "I fear pursuit at any moment! Is the _arabîyeh_
waiting?"

"You have it?" he demanded, some faint sign of human animation
creeping over his impassive face.

"I have!" I replied. "I will give it to you in the _arabîyeh_."

Side by side we passed down the deserted thoroughfare to where, beside
a solitary palm, a pair-horse carriage was waiting. Appreciating
something of my companion's natural impatience, I pressed into his
hand the famous sandalwood box which once had reposed in the _tarbûsh_
of Ali Mohammed. The carriage rolled around a corner and out into the
lighted Shâria Mobâdayân. Abû Tabâh opened the sandalwood box, and
then, reverently, the inner box of silver. Within shimmered the pearls
of the sacred _burko_. He did not touch the relic with his hands, but
reclosed the boxes and concealed the reliquary beneath his black robe.
I heard the crackle of notes; and a little packet surrounded by a band
of elastic was pressed into my hand.

"Three hundred pounds, English," said Abû Tabâh. "One hundred pounds
in recompense for the commission you returned, and two hundred pounds
for the recovery of the relic."

I thrust the wad into the bag beneath my robe containing the other
spoils of the evening. A second and even more grateful glow of
professional joy warmed my heart. For in the reliquary which I had
handed to Abû Tabâh reposed the second product of Suleyman Levi's
scientific toils; his four days' labor having resulted in the
production of two quite passable duplicates; although neither were by
any means up to the standard of Messrs. Moses, Murphy & Co.

Coming to the house wherein I had endued my disguise, Abû Tabâh left
me to metamorphose myself into a decently dressed Englishman suitable
for admission to an hotel of international repute.

"_Lîltâk sa'îda_, Abû Tabâh," I said.

In the open doorway he turned.

"_Lîltâk sa'îda_, Kernaby Pasha," he replied, and smiled upon me very
sweetly.


VI

It was after midnight when I returned to Shepheard's, but I went
straight to my room, and switching on the table-lamp, wrote a long
letter to my principals. Something seemed to have gone wrong with the
lock of my attaché-case, and my good humor was badly out of joint by
the time that I succeeded in opening it. From underneath a mass of
business correspondence I took out a large, sealed envelope, which I
enclosed with a letter in one yet larger, to be registered to Messrs.
Moses, Murphy & Co., Birmingham, in the morning. I turned in utterly
tired but happy, to dream complacently of the smile of Abû Tabâh and
of the party of holy men who had journeyed from Ispahân.

Exactly a fortnight later the following registered letter was handed
to me as I was about to sit down to lunch--

  The Hon. Neville Kernaby.
    Shepheard's Hotel,
      Cairo, Egypt.

  DEAR MR. NEVILLE KERNABY--

  We are returning herewith the silken veil which you describe as "the
  authentic _burko_ of the Seyyîdeh Nefîseh, stolen from her shrine in
  the Tombs of the Khalîfs." Your statement that you can arrange for its
  purchase at the cost of one thousand pounds does not interest us, nor
  do we expect so high-salaried an expert as yourself to send us
  palpable and very inferior forgeries. We are manufacturers of
  duplicates, not buyers of same.

             Yours truly,

                 LLOYD LLEWELLYN.

     (For Messrs. Moses, Murphy & Co.).

I was positively aghast. Tearing open the enclosed package, I glared
like a madman at the _yashmak_ which it contained. The silk, in
comparison with that of which the real veil was compared, was coarse
as cocoanut matting; the embroidery was crude; the pearls shrieked
"imitation" aloud! At a glance I knew the thing for one of the pair
made by Suleyman Levi!

The truth crashed in upon my mind. Following my visit to the _harêm_
of Yûssuf Bey, I had bestowed no more than a glance upon the envelope
wherein, early on the morning of the same day, I had lovingly sealed
the authentic veil; and a full hour had elapsed between the time of
parting with the sugar-lipped one and my return to my rooms at the
hotel.

I understood, now, why the lock of my attaché-case had been out of
order on that occasion ... and I comprehended the sweet smile of Abû
Tabâh!




II

THE DEATH-RING OF SNEFERU


I

The orchestra had just ceased playing; and, taking advantage of the
lull in the music, my companion leaned confidentially forward,
shooting suspicious glances all around him, although there was nothing
about the well-dressed after-dinner throng filling Shepheard's that
night to have aroused misgiving in the mind of a cinema anarchist.

"I have a very big thing in view," he said, speaking in a husky
whisper. "I shall be one up on you, Kernaby, if I pull it off."

He glanced sideways, in the manner of a pantomime brigand, at a party
of New York tourists, our immediate neighbors, and from them to an
elderly peer with whom I was slightly acquainted and who, in addition
to his being stone deaf, had never noticed anything in his life, much
less attempted so fatiguing an operation as intrigue.

"Indeed," I commented; and rang the bell with the purpose in view of
ordering another cooling beverage.

True, I might be the Egyptian representative of a Birmingham
commercial enterprise, but I did not gladly suffer the society of
this individual, whose only claim to my acquaintance lay in the fact
that he was in the employ of a rival house. My lack of interest
palpably disappointed him; but I thought little of the man's qualities
as a connoisseur and less of his company. His name was Theo Bishop and
I fancy that his family was associated with the tanning industry.
I have since thought more kindly of poor Bishop, but at the time of
which I write nothing could have pleased me better than his sudden
dissolution.

Perhaps unconsciously I had allowed my boredom to become rudely
apparent; for Bishop slightly turned his head aside, and--

"Right-o, Kernaby," he said; "I know you think I am an ass, so we
will say no more about it. Another cocktail?"

And now I became conscience-stricken; for mingled with the
disappointment in Bishop's tone and manner was another note. Vaguely
it occurred to me that the man was yearning for sympathy of some kind,
that he was bursting to unbosom himself, and that the vanity of a
successful rival was by no means wholly responsible. I have since
placed that ambiguous note and recognized it for a note of tragedy.
But at the time I was deaf to its pleading.

We chatted then for some while longer on indifferent topics, Bishop
being, as I have indicated, a man difficult to offend; when, having
correspondence to deal with, I retired to my own room. I suppose I
had been writing for about an hour, when a servant came to announce
a caller. Taking an ordinary visiting-card from the brass salver,
I read--

     ABÛ TABÂH.

No title preceded the name, no address followed, but I became aware of
something very like a nervous thrill as I stared at the name of my
visitor. Personality is one of the profoundest mysteries of our being.
Of the person whose card I held in my hand I knew little, practically
nothing; his actions, if at times irregular, had never been wantonly
violent; his manner was gentle as that of a mother to a baby and his
singular reputation among the natives I thought I could afford to
ignore; for the Egyptian, like the Celt, with all his natural
endowments, is yet a child at heart. Therefore I cannot explain why,
sitting there in my room in Shepheard's Hotel, I knew and recognized,
at the name of Abû Tabâh, the touch of fear.

"I will see him downstairs," I said.

Then, as the servant was about to depart, recognizing that I had made
a concession to that strange sentiment which the Imám Abû Tabâh had
somehow inspired in me--

"No," I added; "show him up here to my room."

A few moments later the man returned again, carrying the brass salver,
upon which lay a sealed envelope. I took it up in surprise, noting
that it was one belonging to the hotel, and, ere opening it--

"Where is my visitor?" I said in Arabic.

"He regrets that he cannot stay," replied the man; "but he sends you
this letter."

Greatly mystified, I dismissed the servant and tore open the envelope.
Inside, upon a sheet of hotel notepaper I found this remarkable
message--

  KERNABY PASHA--

  There are reasons why I cannot stay to see you personally, but I
  would have you believe that this warning is dictated by nothing
  but friendship. Grave peril threatens. It is associated with the
  hieroglyphic--

  [Illustration]

  If you would avert it, and if you value your life, avoid all contact
  with anything bearing this figure.

     ABÛ TABÂH.

The mystery deepened. There had been something incongruous about the
modern European visiting-card used by this representative of Islam,
this living illustration of the _Arabian Nights_; now, his
incomprehensible "warning" plunged me back again into the mediæval
Orient to which he properly belonged. Yet I knew Abû Tabâh, for all
his romantic aspect, to be eminently practical, and I could not credit
him with descending to the methods of melodrama.

As I studied the precise wording of the note, I seemed to see the slim
figure of its author before me, black-robed, white-turbaned, and
urbane, his delicate ivory hands crossed and resting upon the head of
the ebony cane without which I had never seen him. Almost, I succumbed
to a sort of subjective hallucination; Abû Tabâh became a veritable
presence, and the poetic beauty of his face struck me anew, as, fixing
upon me his eyes, which were like the eyes of a gazelle, he spoke the
strange words cited above, in the pure and polished English which he
held at command, and described in the air, with a long nervous
forefinger, the queer device which symbolized the Ancient Egyptian
god, Set, the Destroyer.

Of course, it was the aura of a powerful personality, clinging even to
the written message; but there was something about the impression made
upon me which argued for the writer's sincerity.

That Abû Tabâh was some kind of agent, recognized--at any rate
unofficially--by the authorities, I knew or shrewdly surmised; but the
exact nature of his activities, and how he reconciled them with his
religious duties, remained profoundly mysterious. The episode had
rendered further work impossible, and I descended to the terrace, with
no more definite object in view than that of finding a quiet corner
where I might meditate in the congenial society of my briar, and at
the same time seek inspiration from the ever-changing throng in the
Shâria Kâmel Pasha.

I had scarcely set my foot upon the terrace, however, ere a hand was
laid upon my arm. Turning quickly I recognized, in the dusk, Hassan
es-Sugra, for many years a trusted employee of the British
Archæological Society.

His demeanor was at once excited and furtive, and I recognized with
something akin to amazement that he, also, had a story to unfold. I
mentally catalogued this eventful evening "the night of strange
confidences."

Seated at a little table on the deserted balcony (for the evening was
very chilly) and directly facing the shop of Philip, the dealer in
Arab woodwork, Hassan es-Sugra told his wonder tale; and as he told it
I knew that Fate had cast me, willy-nilly, for a part in some comedy
upon which the curtain had already risen here in Cairo, and whereof
the second act should be played in perhaps the most ancient setting
which the hand of man has builded. As the narrative unrolled itself
before me, I perceived wheels within wheels; I was wholly absorbed,
yet half incredulous.

"... When the professor abandoned work on the pyramid, Kernaby Pasha,"
he said, bending eagerly forward and laying his muscular brown hand
upon my sleeve, "it was not because there was no more to learn
there."

"I am aware of this, O Hassan," I interrupted, "it was in order that
they might carry on the work at the Pyramid of Illahûn, which resulted
in a find of jewelery almost unique in the annals of Egyptology."

"Do I not know all this!" exclaimed Hassan impatiently; "and was not
mine the hand that uncovered the golden uræus? But the work projected
at the Pyramid of Méydûm was never completed, and I can tell you why."

I stared at him through the gloom; for I had already some idea
respecting the truth of this matter.

"It was that the men, over two hundred of them, refused to enter the
passage again," he whispered dramatically, "it was because misfortune
and disaster visited more than one who had penetrated to a certain
place therein." He bent further forward. "The Pyramid of Méydûm is the
home of a powerful _Efreet_, Kernaby Pasha! But I who was the last to
leave it, know what is concealed there. In a certain place, low down
in the corner of the King's Chamber, is a ring of gold, bearing a
cartouche. It is the royal ring of the Pharaoh who built the pyramid."

He ceased, watching me intently. I did not doubt Hassan's word, for I
had always counted him a man of integrity; but there was much that was
obscure and much that was mysterious in his story.

"Why did you not bring it away?" I asked.

"I feared to touch it, Kernaby Pasha; it is an evil talisman. Until
to-day I have feared to speak of it."

"And to-day!"

Hassan extended his hands, palms upward.

"I am threatened with the loss of my house," he said simply, "if I do
not find a certain sum of money within a period of twelve days."

I sat resting my chin on my hand and staring into the face of Hassan
es-Sugra. Could it be that from superstitious motives such a treasure
had indeed been abandoned? Could it be that Fate had delivered into
my hands a relic so priceless as the signet-ring of Sneferu, one of
the earliest Memphite Pharaohs? Since I had recently incurred the
displeasure of my principals, Messrs. Moses, Murphy & Co., of
Birmingham, the mere anticipation of such a "find" was sufficient to
raise my professional enthusiasm to white heat, and in those few
moments of silence I had decided upon instant action.

"Meet me at Rikka Station, to-morrow morning at nine o'clock," I said,
"and arrange for donkeys to carry us to the pyramid."


II

On my arrival at Rikka, and therefore at the very outset of my
inquiry, I met with what one slightly prone to superstition might
have regarded as an unfortunate omen. A native funeral was passing
out of the town amid the wailing of women and the chanting by the
_Yemeneeyeh_, of the Profession of the Faith, with its queer
monotonous cadences, a performance which despite its familiarity in
the Near East never failed to affect me unpleasantly. By the token of
the _tarbûsh_ upon the bier, I knew that this was a man who was being
hurried to his lonely resting-place on the fringe of the desert.

As the procession wound its way out across the sands, I saw to the
removal of my baggage and joined Hassan es-Sugra, who awaited me by
the wooden barrier. I perceived immediately that something was wrong
with the man; he was palpably laboring under the influence of some
strong excitement, and his dark eyes regarded me almost fearfully.
He was muttering to himself like one suffering from an over-indulgence
in _Hashish_, and I detected the words "_Allahu akbar!_" (God is most
great) several times repeated.

"What ails you, Hassan, my friend?" I said; and noting how his gaze
persistently returned to the melancholy procession wending its way
towards the little Moslem cemetery:--"Was the dead man some relation
of yours?"

"No, no, Kernaby Pasha," he muttered gutturally, and moistened his
lips with his tongue; "I was but slightly acquainted with him."

"Yet you are much disturbed."

"Not at all, Kernaby Pasha," he assured me; "not in the slightest."

By which familiar formula I knew that Hassan es-Sugra would conceal
from me the cause of his distress, and therefore, since I had no
appetite for further mysteries, I determined to learn it from another
source.

"See to the loading of the donkey," I directed him--for three sleek
little animals were standing beside him, patiently awaiting the toil
of the day.

Hassan setting about the task with a cheerful alacrity obviously
artificial, I approached the native station master, with whom I was
acquainted, and put to him a number of questions respecting his
important functions--in which I was not even mildly interested. But to
the Oriental mind a direct inquiry is an affront, almost an insult;
and to have inquired bluntly the name of the deceased and the manner
of his death would have been the best way to have learned nothing
whatever about the matter. Therefore having discussed in detail the
slothful incompetence of Arab ticket collectors and the lazy condition
and innate viciousness of Egyptian porters as a class, I mentioned
incidentally that I had observed a funeral leaving Rikka.

The station master (who was bursting to talk about this very matter,
but who would have declined on principle to do so had I definitely
questioned him) now unfolded to me the strange particulars respecting
the death of one, Ahmed Abdulla, who had been a retired dragoman
though some time employed as an excavator.

"He rode out one night upon his white donkey," said my informant, "and
no man knows whither he went. But it is believed, Kernaby Pasha, that
it was to the Haram el-Kaddâb" (the False Pyramid)--extending his hand
to where, beyond the belt of fertility, the tomb of Sneferu up-reared
its three platforms from the fringe of the desert. "To enter the
pyramid even in day time is to court misfortune; to enter at night is
to fall into the hands of the powerful _Efreet_ who dwells there. His
donkey returned without him, and therefore search was made for Ahmed
Abdulla. He was found the next day"--again the long arm shot out
towards the desert--"dead upon the sands, near the foot of the
pyramid."

I looked into the face of the speaker; beyond doubt he was in deadly
earnest.

"Why should Ahmed Abdulla have wanted to visit such a place at night?"
I asked.

My acquaintance lowered his voice, muttered "_Sahâm Allah fee 'adoo
ed--dîn!_" (May God transfix the enemies of the religion) and touched
his forehead, his mouth, and his breast with the iron ring which he
wore.

"There is a great treasure concealed there, Kernaby Pasha," he
replied; "a treasure hidden from the world in the days of Suleyman
the Great, sealed with his seal, and guarded by the servants of Gánn
Ibn-Gánn."

"So you think the guardian _ginn_ killed Ahmed Abdulla?"

The station master muttered invocations, and--

"There are things which may not be spoken of," he said; "but those who
saw him dead say that he was terrible to look upon. A great _Welee_, a
man of wisdom famed throughout Egypt, has been summoned to avert the
evil; for if the anger of the _ginn_ is aroused they may visit the
most painful and unfortunate penalties upon all Rikka...."

Half an hour later I set out, having confidentially informed the
station master that I sought to obtain a fine turquoise necklet which
I knew to be in the possession of the Sheikh of Méydûm. Little did I
suspect how it was written that I should indeed visit the house of the
venerable Sheikh. Out through the fields of young green corn, the palm
groves and the sycamore orchards I rode, Hassan plodding silently
behind me and leading the donkey who bore the baggage. Curious eyes
watched our passage, from field, doorway, and _shadûf_; but nothing of
note marked our journey save the tremendous heat of the sun at noon,
beneath which I knew myself a fool to travel.

I camped on the western side of the pyramid, but well clear of the
marshes, which are the home of countless wild-fowl. I had no idea how
long it would take me to extract the coveted ring from its
hiding-place (which Hassan had closely described to me); and,
remembering the speculative glances of the villagers, I had no
intention of exposing myself against the face of the pyramid until
dusk should have come to cloak my operations.

Hassan es-Sugra, whose new taciturnity was remarkable and whose
behavior was distinguished by an odd disquiet, set out with his gun to
procure our dinner, and I mounted the sandy slope on the southwest of
the pyramid, where from my cover behind a mound of rubbish, I studied
through my field-glasses the belt of vegetation marking the course of
the Nile. I could detect no sign of surveillance, but in view of the
fact that the smuggling of relics out of Egypt is a punishable offence
my caution was dictated by wisdom.

We dined excellently, Hassan the Silent and I, upon quail, tinned
tomatoes, fresh dates, bread, and Vichy-water (to which in my own case
was added a stiff three fingers of whisky).

When the newly risen moon cast an ebon shadow of the Pyramid of
Sneferu upon the carpet of the sands, I made my way around the angle
of the ancient building towards the mound on the northern side whereby
one approaches the entrance. Three paces from the shadow's edge, I
paused, transfixed, because of that which confronted me.

Outlined against the moon-bright sky upon a ridge of the desert behind
and to the north of the great structure, stood the motionless figure
of a man!

For a moment I thought that my mind had conjured up this phantasmal
watcher, that he was a thing of moon-magic and not of flesh and blood.
But as I stood regarding him, he moved, seemed to raise his head, then
turned and disappeared beyond the crest.

How long I remained staring at the spot where he had been I know not;
but I was aroused from my useless contemplation by the jingling of
camel bells. The sound came from behind me, stealing sweetly through
the stillness from a great distance. I turned in a flash, whipped out
my glasses and searched the remote fringe of the Fáyûm. Stately
across the jeweled curtain of the night moved a caravan, blackly
marked against that wondrous background. Three walking figures I
counted, three laden donkeys, and two camels. Upon the first of the
camels a man was mounted, upon the second was a _shibreeyeh_, a sort
of covered litter, which I knew must conceal a woman. The caravan
passed out of sight into the palm grove which conceals the village of
Méydûm.

I returned my glasses to their case, and stood for some moments deep
in reflection; then I descended the slope, to the tiny encampment
where I had left Hassan es-Sugra. He was nowhere to be seen; and
having waited some ten minutes I grew impatient, and raising my voice:

"Hassan!" I cried; "Hassan es-Sugra!"

No answer greeted me, although in the desert stillness the call must
have been audible for miles. A second and a third time I called his
name ... and the only reply was the shrill note of a pyramid bat that
swooped low above my head; the vast solitude of the sands swallowed up
my voice and the walls of the Tomb of Sneferu mocked me with their
echo, crying eerily:

"Hassan! Hassan es-Sugra.... Hassan!..."


III

This mysterious episode affected me unpleasantly, but did not divert
me from my purpose: I succeeded in casting out certain demons of
superstition who had sought to lay hold upon me; and a prolonged
scrutiny of the surrounding desert somewhat allayed my fears of human
surveillance. For my visit to the chamber in the heart of the ancient
building I had arrayed myself in rubber-soled shoes, an old pair of
drill trousers, and a pyjama jacket. A Colt repeater was in my hip
pocket, and, in addition to several instruments which I thought might
be useful in extracting the ring from its setting, I carried a
powerful electric torch.

Seated on the threshold of the entrance, fifty feet above the desert
level, I cast a final glance backward towards the Nile valley, then,
the lighted torch carried in my jacket pocket, I commenced the descent
of the narrow, sloping passage. Periodically, when some cranny between
the blocks offered a foothold, I checked my progress, and inspected
the steep path below for snake tracks.

Some two hundred and forty feet of labored, descent discovered me in
a sort of shallow cavern little more than a yard high and partly hewn
out of the living rock which formed the foundation of the pyramid.
In this place I found the heat to be almost insufferable, and the smell
of remote mortality which assailed my nostrils from the sand-strewn
floor threatened to choke me. For five minutes or more I lay there,
bathed in perspiration, my nerves at high tension, listening for the
slightest sound within or without. I cannot pretend that I was
entirely master of myself. The stuff that fear is made of seemed to
rise from the ancient dust; and I had little relish for the second
part of my journey, which lay through a long horizontal passage rarely
exceeding fourteen inches in height. The mere memory of that final
crawl of forty feet or so is sufficient to cause me to perspire
profusely; therefore let it suffice that I reached the end of the
second passage, and breathing with difficulty the deathful, poisonous
atmosphere of the place, found myself at the foot of the rugged shaft
which gives access to the King's Chamber. Resting my torch upon a
convenient ledge, I climbed up, and knew myself to be in one of the
oldest chambers fashioned by human handiwork.

The journey had been most exhausting, but, allowing myself only a few
moments' rest, I crossed to the eastern corner of the place and
directed a ray of light upon the crevice which, from Hassan's
description, I believed to conceal the ring. His account having been
detailed, I experienced little difficulty in finding the cavity; but
in the very moment of success the light of the torch grew dim ... and
I recognized with a mingling of chagrin and fear that it was burnt out
and that I had no means of recharging it.

Ere the light expired, I had time to realize two things: that the
cavity was empty ... and that someone or something was approaching the
foot of the shaft along the horizontal passage below!

Strictly though I have schooled my emotions, my heart was beating in
a most uncomfortable fashion, as, crouching near the edge of the shaft,
I watched the red glow fade from the delicate filament of the lamp.
Retreat was impossible; there is but one entrance to the pyramid; and
the darkness which now descended upon me was indescribable; it
possessed horrific qualities; it seemed palpably to enfold me like the
wings of some monstrous bat. The air of the King's Chamber I found to
be almost unbearable, and it was no steady hand with which I gripped
my pistol.

The sounds of approach continued. The suspense was becoming
intolerable--when, into the Memphian gloom below me, there suddenly
intruded a faint but ever-growing light. Between excitement and
insufficient air, I regarded suffocation as imminent. Then, out into
view beneath me, was thrust a slim ivory hand which held an electric
pocket lamp. Fascinatedly I watched it, saw it joined by its fellow,
then observed a white-turbaned head and a pair of black-robed
shoulders follow. In my surprise I almost dropped the weapon which
I held. The new arrival now standing upright and raising his head,
I found myself looking into the face of _Abû Tabâh_!

"To Allah, the Great, the Compassionate, be all praise that I have
found you alive," he said simply.

He exhibited little evidence of the journey which I had found so
fatiguing, but an expression strongly like that of real anxiety rested
upon his ascetic face.

"If life is dear to you," he continued, "answer me this, Kernaby
Pasha; have you found the ring?"

"I have not," I replied; "my lamp failed me; but I think the ring is
gone."

And now, as I spoke the words, the strangeness of his question came
home to me, bringing with it an acute suspicion.

"What do you know of this ring, O my friend?" I asked.

Abû Tabâh shrugged his shoulders.

"I know much that is evil," he replied; "and because you doubt the
purity of my motives, all that I have learned you shall learn also;
for Allah the Great, the Merciful, this night has protected you from
danger and spared you a frightful death. Follow me, Kernaby Pasha,
in order that these things may be made manifest to you."


IV

A pair of fleet camels were kneeling at the foot of the slope below
the entrance to the pyramid, and having recovered somewhat from the
effect of the fatiguing climb out from the King's Chamber--

"It might be desirable," I said, "that I adopt a more suitable raiment
for camel riding?"

Abû Tabâh slowly shook his head in that dignified manner which never
deserted him. He had again taken up his ebony walking-stick and was
now resting his crossed hands upon it and regarding me with his
strange, melancholy eyes.

"To delay would be unwise," he replied. "You have mercifully been
spared a painful and unfortunate end (all praise to Him who averted
the peril); but the ring, which bears an ancient curse, is gone: for
me there is no rest until I have found and destroyed it."

He spoke with a solemn conviction which bore the seal of verity.

"Your destructive theory may be perfectly sound," I said; "but as one
professionally interested in relics of the past, I feel called upon to
protest. Perhaps before we proceed any further you will enlighten me
respecting this most obscure matter. Can you inform me, for example,
what became of Hassan es-Sugra?"

"He observed my approach from a distance, and fled, being a man of
little virtue. Respecting the other matters you shall be fully
enlightened, to-night. The white camel is for you."

There was a gentle finality in his manner to which I succumbed. My
feelings towards this mysterious being had undergone a slight change;
and whilst I cannot truthfully say that I loved him as a brother,
a certain respect for Abû Tabâh was taking possession of my mind. I
began to understand his reputation with the natives; beyond doubt his
uncanny wisdom was impressive; his lofty dignity awed. And no man is
at his best arrayed in canvas shoes, very dirty drill trousers, and
a pyjama jacket.

As I had anticipated, the village of Méydûm proved to be our
destination, and the gait of the magnificent creatures upon which we
were mounted was exhausting. I shall always remember that moonlight
ride across the desert to the palm groves of Méydûm. I entered the
house of the Sheikh with misgivings; for my attire fell short of the
ideal to which every representative of protective Britain looks up,
but often fails to realize.

In a _mandarah_, part of it inlaid with fine mosaic and boasting a
pretty fountain, I was presented to the imposing old man who was
evidently the host of Abû Tabâh. Ere taking my seat upon the _dîwan_,
I shed my canvas shoes, in accordance with custom, accepted a pipe and
a cup of excellent coffee, and awaited with much curiosity the next
development. A brief colloquy between Abû Tabâh and the Sheikh, at the
further end of the apartment resulted in the disappearance of the
Sheikh and the approach of my mysterious friend.

"Because, although you are not a Moslem, you are a man of culture and
understanding," said Abû Tabâh, "I have ordered that my sister shall
be brought into your presence."

"That is exceedingly good of you," I said, but indeed I knew it to be
an honor which spoke volumes at once for Abû Tabâh's enlightenment and
good opinion of myself.

"She is a virgin of great beauty," he continued; "and the excellence
of her mind exceeds the perfection of her person."

"I congratulate you," I answered politely, "upon the possession of a
sister in every way so desirable."

Abû Tabâh inclined his head in a characteristic gesture of gentle
courtesy.

"Allah has indeed blessed my house," he admitted; "and because your
mind is filled with conjectures respecting the source of certain
information which you know me to possess, I desire that the matter
shall be made clear to you."

How I should have answered this singular man I know not; but as he
spoke the words, into the _mandarah_ came the Sheikh, followed by a
girl robed and veiled entirely in white. With gait slow and graceful
she approached the _dîwan_. She wore a white _yelek_ so closely
wrapped about her that it concealed the rest of her attire, and a
white _tarbar_, or head-veil, decorated with gold embroidery, almost
entirely concealed her hair, save for one jet-black plait in which
little gold ornaments were entwined and which hung down on the left of
her forehead. A white _yashmak_ reached nearly to her feet, which were
clad in little red leather slippers.

As she approached me I was impressed, not so much with the details of
her white attire, nor with the fine lines of a graceful figure which
the gossamer robe quite failed to conceal, but with her wonderful
gazelle-like eyes, which wore uncannily like those of her brother,
save that their bordering of _kohl_ lent them an appearance of being
larger and more luminous.

No form of introduction was observed; with modestly lowered eyes the
girl saluted me and took her seat upon a heap of cushions before a
small coffee table set at one end of the _dîwan_. The Sheikh, seated
himself beside me, and Abû Tabâh, with a reed pen, wrote something
rapidly on a narrow strip of paper. The Sheikh clapped his hands, a
man entered bearing a brazier containing live charcoal, and, having
placed it upon the floor, immediately withdrew. The _dîwan_ was
lighted by a lantern swung from the ceiling, and its light, pouring
fully down upon the white figure of the girl, and leaving the other
persons and objects in comparative shadow, produced a picture which
I am unlikely to forget.

Amid a tense silence, Abû Tabâh took from a box upon the table some
resinous substance. This he sprinkled upon the fire in the brazier;
and the girl extending a small hand and round soft arm across the
table, he again dipped his pen in the ink and drew upon the upturned
palm a rough square which he divided into nine parts, writing in each
an Arabic figure. Finally, in the centre he poured a small drop of
ink, upon which, in response to words rapidly spoken, the girl fixed
an intent gaze.

Into the brazier Abû Tabâh dropped one by one fragments of the paper
upon which he had written what I presumed to be a form of invocation.
Immediately, standing between the smoking brazier and the girl, he
commenced a subdued muttering. I recognized that I was about to be
treated to an exhibition of _darb el-mendel_, Abû Tabâh being
evidently a _sahhar_, or adept in the art called _er-roohânee_. Save
for this indistinct muttering, no other sound disturbed the silence of
the apartment, until suddenly the girl began to speak Arabic and in a
sweet but monotonous voice.

"Again I see the ring," she said, "a hand is holding it before me.
The ring bears a green scarab, upon which is written the name of a
king of Egypt.... The ring is gone. I can see it no more."

"Seek it," directed Abû Tabâh in a low voice, and threw more incense
upon the fire. "Are you seeking it?"

"Yes," replied the girl, who now began to tremble violently, "I am in
a low passage which slopes downwards so steeply that I am afraid."

"Fear nothing," said Abû Tabâh; "follow the passage."

With marvelous fidelity the girl described the passage and the shaft
leading to the King's Chamber in the Pyramid of Méydûm. She described
the cavity in the wall where once (if Hassan es-Sugra was worthy of
credence) the ring had been concealed.

"There is a freshly made hole in the stonework," she said. "The
picture has gone; I am standing in some dark place and the same hand
again holds the ring before me."

"Is it the hand of an Oriental," asked Abû Tabâh, "or of a European?"

"It is the hand of a European. It has disappeared; I see a funeral
procession winding out from Rikka into the desert."

"Follow the ring," directed Abû Tabâh, a queer, compelling note in his
voice.

Again he sprinkled perfume upon the fire and--

"I see a Pharaoh upon his throne," continued the monotonous voice,
"upon the first finger of his left hand he wears the ring with the
green scarab. A prisoner stands before him in chains; a woman pleads
with the king, but he is deaf to her. He draws the ring from his
finger and hands it to one standing behind the throne--one who has a
very evil face. Ah!..."

The girl's voice died away in a low wail of fear or horror. But--

"What do you see?" demanded Abû Tabâh.

"The death-ring of Pharaoh!" whispered the soft voice tremulously; "it
is the death-ring!"

"Return from the past to the present," ordered Abû Tabâh. "Where is
the ring now?"

He continued his weird muttering, whilst the girl, who still shuddered
violently, peered again into the pool of ink. Suddenly--

"I see a long line of dead men," she whispered, speaking in a kind of
chant; "they are of all the races of the East, and some are swathed in
mummy wrappings; the wrappings are sealed with the death-ring of
Pharaoh. They are passing me slowly, on their way across the desert
from the Pyramid of Méydûm to a narrow ravine where a tent is erected.
They go to summon one who is about to join their company...."

I suppose the suffocating perfume of the burning incense was chiefly
responsible, but at this point I realized that I was becoming dizzy
and that immediate departure into a cooler atmosphere was imperative.
Quietly, in order to avoid disturbing the séance, I left the
_mandarah_. So absorbed were the three in their weird performance that
my departure was apparently unnoticed. Out in the coolness of the palm
grove I soon recovered. I doubt if I possess the temperament which
enables one to contemplate with equanimity a number of dead men
promenading in their shrouds.


V

"The truth is now wholly made manifest," said Abû Tabâh; "the
revelation is complete."

Once more I was mounted upon the white camel and the mysterious _imám_
rode beside me upon its fellow, which was of less remarkable color.

"I hear your words," I replied.

"The poor Ahmed Abdulla," he continued, "who was of little wisdom,
knew, as Hassan es-Sugra knew, of the hidden ring; for he was one of
those who fled from the pyramid refusing to enter it again. Greed
spoke to him, however, and he revealed the secret to a certain
Englishman, called Bishop, contracting to aid him in recovering the
ring."

At last enlightenment was mine ... and it brought in its train a
dreadful premonition.

"Something I knew of the peril," said Abû Tabâh, "but not, at first,
all. The Englishman I warned, but he neglected my warning. Already
Ahmed Abdulla was dead, having been despatched by his employer to the
pyramid; and the people of Rikka had sent for me. Now, by means known
to you, I learned that evil powers threatened your life also, in what
form I knew not at that time save that the sign of Set had been
revealed to me in conjunction with your death."

I shuddered.

"That the secret of the pyramid was a Pharaoh's ring I did not learn
until later; but now it is made manifest that the thing of power is
the death-ring of Sneferu...."

The huge bulk of the Pyramid of Méydûm loomed above us as he spoke the
words, for we were nearly come to our destination; and its proximity
occasioned within me a physical chill. I do not think an open check
for a thousand pounds would have tempted me to enter the place again.
The death-ring of Sneferu possessed uncomfortable and supernatural
properties. So far as I was aware, no example of such a ring (the
_lettre de cachet_ of the period) was included in any known collection.
One dating much after Sneferu, and bearing the cartouche of Apepi II
(one of the Hyksos, or Shepherd Kings) came to light late in the
nineteenth century; it was reported to be the ring which, traditionally,
Joseph wore as emblematical of the power vested in him by Pharaoh.
Sir Gaston Maspero and other authorities considered it to be a forgery
and it vanished from the ken of connoisseurs. I never learned by what
firm it was manufactured.

A mile to the west of the pyramid we found Theo Bishop's encampment.
I thought it to be deserted--until I entered the little tent....

An oil-lamp stood upon a wooden box; and its rays made yellow the face
of the man stretched upon the camp-bed. My premonition was realized;
Bishop must have entered the pyramid less than an hour ahead of me;
he it was who had stood upon the mound, silhouetted against the sky,
when I had first approached the slope. He had met with the fate of
Ahmed Abdulla.

He had been dead for at least two hours, and by the token of certain
hideous glandular swellings, I knew that he had met his end by the
bite of an Egyptian viper.

"Abû Tabâh!" I cried, my voice hoarsely unnatural--"the _recess_ in
the King's Chamber is a viper's nest!"

"You speak wisdom, Kernaby Pasha; the viper is the servant of the
_ginn_."

Upon the third finger of his swollen right hand Bishop wore the ring
of ghastly history; and the mysterious significance of the Sign of Set
became apparent. For added to the usual cartouche of the Pharaoh was
the symbol of the god of destruction, thus:

[Illustration]

We buried him deeply, piling stones upon the grave, that the jackals
of the desert might never disturb the last holder of the death-ring of
Sneferu.




III

THE LADY OF THE LATTICE


I

The interior of the room was very dark, but with the aid of the
electric torch which I carried I was enabled to form a fairly good
impression of its general character, and having now surveyed the
entire house I had concluded that it might possibly serve my purpose.
The real ownership of many native houses in Cairo is difficult to
establish, and the unveracious Egyptian from whom I had procured the
keys may or may not have been entitled to let the premises. However,
he had the keys; and that in the Near East is a sufficient evidence of
ownership. My viewing the place at night was dictated by motives of
prudence; for I did not propose unduly to impress my personality upon
the inhabitants of the Darb el-Ahmar.

Curiosity respecting the outlook at the rear now led me to enter the
deep recess at one end of the room, which boasted an imperfect but not
unpicturesque _mushrabîyeh_ window. Moonlight slanted down into the
narrow lane which the window overhung and cast a quaint fretwork
shadow upon the dusty floor at my feet. Idly I opened one of the
little square lattices and peered down into the shadowy gully beneath.
The lane was silent and empty, and I next directed my attention to a
similar window which protruded from the adjoining house.

A panel corresponding to mine stood open also in the neighboring
window; and by means of a soft light in the room I detected the head
and shoulders of a woman, who, her arm resting upon the ledge,
surveyed the vacant night.

By reason of her position, whilst her hand and arm lay fully in the
moonlight, her face and figure were indistinct. I, on the contrary,
was clearly visible to her, and although I knew that she must have
seen me she made no effort to withdraw. On the contrary, she leaned
artlessly forward as if to gaze upon the stars, permitting me a sight
of her unveiled face and of a portion of her shapely neck.

Her eyes, as is usual with Egyptian women, were large and fine, and
as is usual with all women, she was aware of the fact, casting glances
upward and to the right and left calculated to exhibit their beauty.

The coquetry of her movements was unmistakable; and when, lifting a
pretty arm, she brushed aside a lock of hair which overhung her brow
and uttered a tremulous sigh, I perceived that I had found favor in
her sight.

And indeed the graceful gesture had inclined my heart towards her;
for it had served to reveal not only the symmetry of her shape but the
presence upon her arm, immediately above the elbow, of a magnificent
bangle in gold and lapis-lazuli which, if I might trust my judgment,
was fashioned no later than the XIXth dynasty! Clearly the house next
door, and its occupant, were the property of some man of wealth and
taste.

There is a maxim in the East--"Avoid the veil"; and to this hitherto
I had paid the strictest attention. Soft glances from _harêm_ windows
usually leave me cold. But the presence of an armlet finer than
anything in the Treasure of Zagazig placed a new complexion upon this
affair, and the connoisseur within me took the matter out of my hands.

Across the intervening patch of darkness our glances met; the girl's
dark lashes were lowered demurely, then raised again, and the boldness
of my unfaltering gaze was rewarded by a smile. Thus encouraged:--

"O daughter of the moon," I whispered fancifully in Arabic,
"condescend to speak to one whom the sight of thy beauty hath
enslaved."

"I fear to be discovered, Inglîsi," came the soft reply; "or willingly
would I converse with thee, for I am lonely and wretched."

She sighed again and directed upon me a glance that was less wretched
than roguish. Evidently the adventure was much to her liking.

"Let me solace your loneliness," I replied; "for assuredly we can
conceive some plan of meeting."

She lowered her eyes at that, and seemed to hesitate; then--

"In the roof of your house," she whispered, often glancing over her
shoulder into the room beyond, "is a trap--which is bolted...."

Footsteps sounded in the lane beneath--whereat the vision at the
window vanished and the lattice was closed; but not before the girl
had intimated by a gesture that I was to remain.

Discreetly withdrawing into my dusty apartment, I endeavored to make
out the form of the intruder who now was passing underneath the
window; but the density of the shadows in the lane rendered it
impossible for me to do so. He seemed to pause for a time and I
imagined that I could see him staring upward; then he passed on and
silence again claimed that deserted quarter of Cairo.

For fully half an hour I waited, and was preparing to depart when a
part of the shadows overlying the projecting window seemed to grow
blacker, and I realized with joy that at last the lattice was
reopening, but that the room within was now in darkness. Whilst I
watched, remaining scrupulously invisible, a small parcel deftly
thrown dropped upon the floor at my feet--and my neighbor's window was
reclosed.

Closing my own, I picked up the parcel. It proved to be a small ivory
box, which at some time had evidently contained _kohl_, wrapped in a
piece of silk and containing a note. Returning to the lower floor I
directed the light of my electric torch upon this charmingly romantic
billet. It was conceived in English and characterized by the rather
alarming _naiveté_ of the Oriental woman. I give it in its entirety.

"To-morrow night, nine o'clock."


II

My cautious inquiries respecting the house in the Darb el-Ahmar led
only to the discovery that it belonged to a mysterious personage whose
real identity was unknown even to his servants; but this did not
particularly intrigue me; for in the East the maintenance of two
entirely self-contained establishments is not more uncommon than in
countries less generously provided in the matter of marriage laws.
After all the taking of a second wife does not so much depend on a
man's religious convictions as upon his first wife.

Reflecting upon the probable history of the armlet of lapis-lazuli,
I returned to Shepheard's in time to keep my appointment with Joseph
Malaglou--a professed Christian who claimed to be of Greek parentage.
I may explain here that it was necessary to provide for the safe
conduct through the customs and elsewhere of those cases of "Sheffield
cutlery" which actually contained the scarabs, necklaces, and other
"antiques," the sale of which formed a part of the business of my
firm. Joseph Malaglou had hitherto successfully conducted this matter
for me, receiving the goods and storing them at his own warehouse; but
for various reasons I had decided in future to lease an establishment
of my own for this purpose.

He was waiting in the lounge as I entered, and had he been less useful
to me I think I should have had him thrown out; for if ever a swarthy
villain stepped forth from the pages of an illustrated "penny
dreadful," that swarthy villain was Joseph Malaglou. He approached me
with outstretched hand; he was perniciously polite; his ingratiating
smile fired my soul with a lust of blood. Fortunately, our business
was brief.

"The latest consignment is in the hands of my agent at Alexandria,"
he said, "and if you are still determined that the ten cases shall
be despatched to you direct, I will instruct him; but you cannot
very well have them sent _here_."

He shrugged and smiled, glancing all about the lounge.

"I have no intention of converting Shepheard's Hotel into a cutlery
warehouse," I replied. "I will advise you in the morning of the
address to which the cases should be despatched."

Joseph Malaglou was palpably disturbed--a mysterious circumstance,
since, whilst I had made no mention of reducing his fees, under the
new arrangement he would be saved trouble and storage.

"As delay in these matters is unwise," he urged, "why not have the
goods despatched immediately, and consigned to you at my address?"

There was reason on the man's side, for I had not yet actually leased
the house in the Darb el-Ahmar; therefore--

"I will sleep on the problem," I said, "and communicate my decision in
the morning."

I stood on the steps watching him depart, a man palpably disturbed in
mind; indeed his behavior was altogether singular, and could only
portend one thing--knavery. I think it highly probable that the
Ottoman Empire had a certain claim upon Joseph Malaglou. He was one of
those nondescript brutes whose mere existence is a menace to our rule
in the Near East. He openly applauded British methods, and was the
worst possible advertisement for the cause he claimed to have
espoused. Altogether he left me in an uneasy mood; so that shortly
after the third, or daybreak, call to prayer had sounded from Cairo's
minarets on the morrow, I had arranged to lease the house in the Darb
el-Ahmar for a period of three months, in the name of one Ahmed Ben
Tawwab, a mythical friend, and had instructed Joseph Malaglou
accordingly.

Other affairs claimed my attention throughout the day; but dusk
discovered me at my newly acquired house in the quaint street
adjoining the Bâb ez-Zuwêla. I procured the keys from the venerable
old thief who had leased me the premises and learned from him that a
representative of Joseph Malaglou had been admitted to the house
earlier in the evening, in accordance with my instructions, and had
delivered a load of boxes there.

Thus, on opening the door, I was not surprised to find the ten cases
from Alexandria lying within, neatly labelled:

  To Ahmed Ben Tawwab,
    Darb el-Ahmar,
      Sukkarîya,
        Cairo.

Ascending to the top floor, I mounted the rickety ladder and unbolted
and opened the trap. A cautious glance to the right revealed the fact
that little difficulty existed in passing from roof to roof; for in
Egyptian houses these are flat and are used for various domestic
purposes. I consulted my watch: the hour of the tryst was come.

And even as I learned the fact, from my neighbor's roof sounded the
faint creaking of hinges ... and out into the moonlight stepped an odd
figure--that of the lady of the lattice, dressed in a "European" blue
serge costume which had obviously been purchased, ready made, in the
bazaars! She wore high-heeled French shoes upon her pretty feet and
her picturesque hair was concealed beneath a large Panama hat, from
the brim of which floated one of those voluminous green veils dear to
the heart of touring woman and so arranged as to hide her face. Only
the gleam of her eyes and teeth was visible through the gauze.

I assisted her to step across, wondering since she was thus attired,
to what crazy expedition I was committed.

"Please do not kiss me," she whispered, speaking in moderately good
English, "Fatimah is listening!"

Such ingenuousness was rather alarming.

"But," I replied, "you have left the trap open."

"It is all right. Fatimah has locked the door of my room and will
admit no one, because I have a headache and am sleeping!"

Resting her hand confidingly in mine, she descended the ladder into
the adjoining house, and, removing the veil from her face, looked up
at me.

"You will be kind to me, will you not?" she asked.

I suppose a lengthy essay upon the mentality of Oriental womanhood
would serve no purpose here, therefore I refrain from inserting it.
Seated upon the chests in the room below, Mizmûna--for this was her
name--confided her troubles with perturbing frankness. She had
conceived a characteristically Eastern and sudden infatuation for my
society; nor am I prepared to maintain that she would have remained
obdurate to anyone else who had been in a position to unbolt the door
which offered the only chance of escape from her prison. The house of
mystery, she informed me, belonged to a person styling himself Yûssuf
of Rosetta (a name that sounded factitious) and she hated him. For two
months, I gathered, she had been in Cairo, during which time she had
never passed beyond the walls of the neighboring courtyard. And the
object of her nocturnal adventure was innocent enough; she wanted to
see the European shops and the tourists passing in and out of the big
hotels in the Shâria Kâmel Pasha!


III

It was as we passed along the Shâria el-Maghribi, where I had pointed
out the St. James's Restaurant, better known as "Jimmy's," I remember,
that Mizmûna uttered a little, suppressed cry, and clutched my arm
sharply.

"Oh!" she whispered fearfully, "it is Hanna! and he has seen me!"

With frightened, fascinated eyes she was staring across the street,
apparently at a group of curiously muffled natives--and her whole body
was trembling.

"Quick!" she said, pulling me urgently, "take me back! if they find me
they will kill me!"

"But if they have already seen you----"

"Oh! take me back," she entreated piteously. "Hanna must not find out
where I live."

Here was mystery; but evidently my first dreadful theory that Hanna
was Mizmûna's husband had been incorrect. Apparently he was not even
acquainted with Yûssuf of Rosetta. But whoever or whatever he might
be, I silently cursed the lapis armlet which had led me to involve
myself in his affairs, as I hurried my companion across the Place de
l'Opera and homeward....

We were come indeed unmolested but breathless, as near our destination
as that nameless street beside the Mosque of Muayyâd, when Mizmûna
suddenly stopped, uttered a stifled shriek, and--

"Oh, save me!" she panted, winding her arms about my neck. "Look!
Look! in the shadow of the mosque door!"

Panic threatened me for one fleeting moment; for this part of Cairo is
utterly deserted at night and the mystery of the thing was taking toll
of my nerves; then firmly unclasping the trembling arms, I pushed
Mizmûna behind me and snatched out my Colt automatic ... as a group of
muffled figures became magically detached from the shadows that had
hidden them; and began silently to advance.

I raised the pistol.

"_Usbur!_" I cried "_âuz eh?_" (Stop! what do you want?)

They halted at once; but no answering voice broke the uncanny silence
in which they regarded me. Mizmûna plucked at my arm.

"Quick! Quick!" she whispered tremulously, "the keys! the keys!"

I was swift to grasp her meaning.

"My right pocket!" I whispered in answer.

The girl's shaking hand groped for the keys, found them; and, uttering
no parting word, Mizmûna darted off along the Sukkarîya, which here
bisects the Darb el-Ahmar. An angry muttering arose from the little
knot of oddly muffled figures, but not one of them had the courage to
attempt a pursuit of the fugitive. Keeping my back to the wall of the
mosque and feeling along it with one hand outstretched, I began to
back away from the attacking party; intending to take to my heels
along the first lane I came to.

This plan was sound enough; its weakness lay in the fact that I could
make no proper survey of that which lay immediately behind me. The
result was that I backed into someone who must have been stealthily
approaching from the rear.

I knew nothing of his presence until he suddenly threw himself upon
from behind, and I was down on my face in the dust! My pistol was
jerked out of my hand, and, still preserving that unbroken
disconcerting silence, the muffled group bore down upon me.

I gave myself up for lost. My unseen assailant, who seemingly
possessed wrists of steel, jerked my right hand up into the region of
my shoulder-blades and pinioned my left arm so as to render me
helpless as an infant. Then two of the muffled Nubians--for Nubians
the moonlight now showed them to be--raised me to my feet, and the
grip from behind was removed.

That I had unwittingly intruded upon the amours of some wealthy and
unscrupulous pasha I no longer doubted; and knowing somewhat of the
ways of outraged lovers of the East, the mental vision which arose
before me was unpleasing to contemplate. Yet even the extravagant
picture which my imagination had painted fell short of the ferocious
reality. For even as I was lifted upright, in the grasp of my huge
guards, a door in the side of the neighboring mosque burst open, and
there sprang into view an excessively tall, excessively lean and
hawk-faced old man carrying a naked scimitar in his hand.

He possessed eyes like the eyes of an eagle, and a thin, hooked nose
having dilated, quivering nostrils. In three huge strides he reached
me, towered over me like some evil _ginnee_ of Arabian lore, and
raised his gleaming scimitar with the unmistakable intention of
severing my head from my trunk at a single blow!

I think I have never experienced an identical sensation in my life;
my tongue clave to the roof of my mouth; my heart suspended its
functions; and I felt my eyes start forward in their sockets. I had
not thought my constitution capable of such profound and helpless
fear, nor had I hitherto paid proper respect to the memory of Charles
I. I would gladly have closed my eyes in order that I might not
witness the downward sweep of the fatal blade, but the lids seemed to
be paralysed. Never whilst memory serves me can I forget one detail of
the appearance of that frightful old devil; and never can I forget my
gratitude to that unseen captor, the man who had seized me from
behind, and who now, alone, averted the blade from my neck.

Over my head he lunged--with an ebony stick--and skilfully; so that
the pointed ferrule came well and truly into contact with the knuckles
of my would-be executioner. The weapon fell, jingling, at my feet ...
and a slim, black-robed figure was suddenly interposed between myself
and the furious old Arab.

It was Abû Tabâh!

Dignified, unruffled, his classically beautiful face composed and
resembling, in the moonlight, beneath the snowy turban, that of some
young prophet, he stood, one protective hand resting upon my shoulder,
and confronted my assailant. His eyes were aglow with the eerie light
of fanaticism.

"It is written that the wrath of fools is the joy of Iblees,"[A] he
declared.

  [A] Satan.

Their glances met in conflict, the eagle eyes of my aged but
formidable enemy glaring insanely into the fine, dark eyes of Abû
Tabâh. The Arab was by no means quelled; yet presently his glance fell
before the hypnotic stare of the mysterious _imám_.

"The Prophet (may God be kind to him) spared not the despoiler!" he
said heavily. "With these, my two hands"--he extended the twitching,
sinewy members before Abû Tabâh--"will I choke the life from the
throat of the dog who wronged me."

Abû Tabâh raised his hand sternly.

"This matter has been entrusted to _me_," he said, staring down the
enraged old man. "If you would have me abandon it, say so; if you
would have me pursue it, be silent."

For five seconds the other sustained the strange gaze of those big,
mysterious eyes, then folded his arms upon his breast, audibly
gnashing his large and strong-looking teeth and averting his head from
my direction in order that spleen might not consume him. Abû Tabâh
turned and confronted me.

"Explain the cause of your presence here," he demanded, continuing to
speak in Arabic, "and unfold to me the whole truth respecting your
case."

"My friend," I replied, steadily regarding him, "I am eternally your
debtor; but I decline to utter one word for explanation until these
fellows unhand me and until I am offered some suitable excuse for the
outrageous attack upon my person."

Abû Tabâh performed his curiously Gallic shrug of the shoulders--and
pointed, with his ebony cane, to my pinioned arms. In a trice the
Nubians fell back, and I was free. The infuriated old man directed
upon me a glance that was bloodily ferocious, but--

"O persons of little piety," I said, "is it thus that a true Moslem
rewards the generous impulse and the meritorious deed? To-night a
damsel in distress, flying from a brutal captor, solicited my aid.
I was treacherously assaulted ere I could escort her to a place of
safety, and all but murdered by the man who would appear to be that
damsel's natural protector. Alas, I fear to contemplate what may have
befallen her as a result of such vile and foolish conduct."

Abû Tabâh slightly inclined his body resting his slim, ivory hands
upon his cane; his face remained perfectly tranquil as he listened to
this correct, though misleading statement; but--

"Ah!" cried the old man of the scimitar, adopting an unpleasant,
crouching attitude, "perjured liar that thou art! Did I not see with
mine own eyes how she embraced thee? O, son of a mange, that I should
have lived to have witnessed so obscene a spectacle. Not content with
despoiling me of this jewel of my _harêm_, thou dost parade her
abandonment and my shame in the public highways of Cairo!..."

In vain Abû Tabâh strove to check this tirade. Step by step the Sheikh
approached closer; syllable by syllable his voice rose higher.

"What!" he shrieked, "is it for this that I have offered five thousand
English pounds to whomsoever shall restore her to me! Faugh! I spit
upon her memory!--and though I pursue thee to the Mountains of the
Moon, across the Bridge Es-Sîrat, and through the valley of Gahennam,
lo! my hour will come to slay thee, noisome offal!"

He ceased from lack of breath, and stood quivering before me. But at
last I had grasped the clue to this imbroglio into which fate had
thrust me.

"O misguided man," I replied, "grief hath upset thine intelligence.
Again I tell thee that I sought to deliver the damsel from her
persecutor, and, perceiving an ambush, she clung to me as her only
protector. Thou are demented. Let another earn the paltry reward;
I will have none of it."

I turned to Abû Tabâh, addressing him in English.

"Relieve me of the society of this infatuated old ruffian," I said,
"and accompany me to some place where I can quietly explain what I
know of the matter."

"Assuredly I will accompany you to such a spot," he answered suavely;
"for whilst, knowing your character, I do not believe you to be the
abductor of the damsel Mizmûna, a warrant to search your house was
issued an hour ago, on a charge of _hashish_ smuggling!"


IV

There are certain shocks that numb the brain. This was one of them.
My recollection of the period immediately following those words of Abû
Tabâh is hazy and indistinct. My narrow escape from decapitation at
the hands of the ferocious Arab assassin and the tangled love-affairs
of that aged Othello became insignificant memories. (I seem to
recollect that we left him in tears.)

My next clear-cut memory is that of walking beside the mysterious
_imâm_ along the Darb el-Ahmar and of stopping before the closed door
of my newly acquired premises!

The street was quite deserted again. Those muffled Nubians who seemed
to constitute a bodyguard for my inscrutable companion had disappeared
in company with the bereaved Sheikh.

"This is your house?" said Abû Tabâh sweetly.

My habit of thinking before I speak or act asserted itself
automatically.

"I recently leased it on another's behalf," I replied.

"In that event," continued the _imâm_, "unless the information lodged
with me to-night prove to be inaccurate, that other must speedily
proclaim himself."

He tested the cumbersome lock, and, as I knew would be the case, since
Mizmûna had recently entered, found it to be unfastened, opened the
door and stepped in.

"Have you a pocket lamp?" he asked.

I pressed the button of my electric torch and directed its rays fully
upon the stack of boxes. It was the great sage, Apollonius of Tyana,
who said "loquacity has many pitfalls, but silence none"; therefore I
silently watched Abû Tabâh consulting the label on the topmost chest.
Presently--

"Ahmed Ben Tawwab," he read aloud; "is that the name of the friend on
whose behalf you secured a lease of this house?"

"It is," I answered.

"If you will rest the light upon this box and assist me to open one of
the others, I shall be obliged to you," said Abû Tabâh.

Knowing, as I did, that this strange man was in some way connected
with the native police and with the guardianship of Egyptian morals,
I recognized refusal to be impolitic if not impossible. But, as we set
to work to raise the lid of the chest, my mind was more feverishly
busy than my fingers.

Ere long our task was successful, and the contents of the chest lay
exposed. These were: two hundred Osiris statuettes, twelve one-pound
tins of mummy heads ... _and fifty packets of hashish_.

Silence was no effort to me now; I was dumbfounded. The musical voice
of my companion broke in upon my painful reverie.

"The information upon which I now am acting," he said, "reached me
to-night in the form of a letter, bearing no address and no
signature. The suppression of this vile _hashish_ traffic is so near
to my heart that I immediately secured the necessary powers to search
the premises named, and was on my way hither when I observed you
(although I did not at once recognize you) in the act of escaping from
a group of my servants who had been detailed, some weeks ago, to trace
a missing damsel known to be in Cairo. Concerning your share in that
affair I await a full statement from your own lips; concerning your
share in this I can only say that unless Ahmed Ben Tawwab comes
forward by to-morrow and admits his guilt, I must apply to the British
agent for a formal inquiry. Is there anything that you would wish to
say, or any action you desire that I should take?"

I turned to him in the dim light. Habitually I am undemonstrative,
especially with natives. But there was a nobility and an implacable
sense of justice about this singular _religieux_ which conquered me
completely.

"Abû Tabâh," I said, "I thank you for your friendship. I have
committed a grave folly; but I am neither an abductor nor a _hashish_
dealer. This is the work of an unknown enemy, and already I have a
theory respecting his identity."

"Can I aid you--or do you prefer that I leave you to pursue this clue
in your own way?" he asked tactfully.

"I prefer to work alone."

"The affair is truly mysterious," he admitted, "and I purpose to
spend the night in meditation respecting it. After the hour of morning
prayer, therefore, I will visit you. _Lîltâk sa'îda_, Kernaby Pasha."

"_Lîltâk sa'îda_, Abû Tabâh," I said, as he stepped out of the door.

Slowly and stately the _imám_ passed down the street; and the _ginnee_
of solitude reclaimed that deserted spot. A night watchman, _nebbut_
on shoulder, passed along the distant Sukkarîya. A dog howled.

I re-entered the doorway conscious of a sudden mental excitement; for
an explanation of the anonymous letter had just presented itself to
my mind. The owner of the neighboring house must have detected my
rendezvous with his lady-love, have investigated the contents of the
cases, and denounced me from motives of revenge! That the villainous
Joseph Malaglou had been in the habit of smuggling _hashish_ into
Egypt in my cases of "cutlery" was evident enough and accounted for
his reluctance to fall in with the new arrangement; but my bemused
brain utterly failed to grapple with the problem of why, knowing their
damning contents, he had permitted these ten cases to be delivered
at _my_ address. Moreover, how my worthy neighbor--who had evidently
abducted Mizmûna from the old man of the scimitar--had learned my
real name was another mystery which I found no leisure to examine.
For I had but just set foot again within the ill-omened place when
there came a patter of swift, light footsteps--and out from behind
the fatal stack of boxes ran Mizmûna, and threw herself into my arms!

"Oh, my friend, my protector!" she cried distractedly, "what shall I
do? Yûssuf has discovered our plot! Fatimah, that mother of
calamities, has betrayed me, and I dare not return! I am an outcast;
for although I was stolen from the Sheikh Ismail without my consent,
how can I hope for his forgiveness?"

Such a flood of sorrows and confidences overwhelmed me, and I placed a
silent but deathless curse upon the lapis armlet which had brought me
to this pass. Mizmûna sobbed upon my shoulder.

"Yûssuf has planned your ruin as well as mine," she said brokenly.
"For it was he who denounced you to the Magician." (As "the Magician"
Abû Tabâh was known and feared throughout Lower Egypt.) "Oh that I
might return to the house of Ismail where I lived in luxury in a
marble pavilion, guarded by Hanna and a hundred negroes, where I
possessed the robes of a princess and was laden with costly jewels!"

So very human and natural an ambition met with my hearty approval,
and, upon consideration of the word-picture of his domestic state, the
old man of the scimitar rose immensely in my esteem. How my malevolent
neighbor had succeeded in abducting Mizmûna from such a fortress I
failed to imagine. But I began to see my way more clearly and hope was
reborn in my bosom.

"Fear nothing, child," I said to the weeping girl. "You shall return
to your marble pavilion and to the care of that worthy, if somewhat
hasty man, from whose arms you were torn. And now inform me--where is
Yûssuf?"

Mizmûna raised her face and looked up at me, her long lashes wet with
tears, but the slow, childish smile of the Eastern woman already
curving her red lips.

"He is in his own room destroying papers," she said.

"Who told you this?"

"Ali, the _bowwab_, who is faithful to me--and who hates Fatimah."

"Is the trap rebolted?"

"I know not."

"Remain here until I return," I said, seating her upon one of the
boxes. "Where are my keys?"

"I hid them upon the ledge of the window, beside the door yonder."

Taking them from this simple "hiding-place," I locked the door to
give Mizmûna courage, and, taking the lamp with me, began to mount
the stairs, first assuring myself of the presence in my pocket of
my Colt automatic, which Abû Tabâh had restored to me.

The ray of my lamp shining out ahead, I came to the crazy ladder
giving access to the trap. I climbed up, raising the trap, and gazed
upon the jeweled dome of midnight Egypt. Dire necessity spurred me,
and I walked across to the adjoining trap, carefully inserted two
fingers in the iron ring and pulled.

It was not fastened below! Inch by inch I raised it, and, finding the
room beneath it to be in darkness, opened the trap fully and descended
the ladder.

I flashed the light quickly about the place; then stood staring at
what it revealed. My heart began to beat rapidly, for in that dirty
attic I had found salvation ... and a further clue to the mystery of
all my misfortunes.

It was a _hashish_ warehouse!

Taking off my shoes, I thrust one into either pocket of my jacket,
and, perceiving that the house was constructed on a plan identical
with that adjoining it, I crept downstairs to the apartment of the
_mushrabîyeh_ window. A heavy curtain was draped in the doorway, but
I could see that the room within was illuminated.

I drew the curtains slowly aside and peeped in. I saw an apartment
that had evidently been furnished very luxuriantly, but which now was
partially dismantled. In the recess formed by the window a low table
was placed, bearing a shaded lamp. The table was littered with papers,
account books and ledgers; and, seated thereat, his back towards the
door, was a man who figured feverishly. I stepped into the room.

"Good evening, Yûssuf of Rosetta," I said; "you do well to set your
affairs in order."


V

Swiftly as though a serpent had touched him, the man in the recess
leaped to his feet and twisted about to confront me.

I found myself looking into a hideous, swarthy face--blanched now to
the lips, so that the cunning black eyes glared out as from a
mask--into the hideous swarthy face of _Joseph Malaglou_!

The store of _hashish_ in the upper room had somewhat prepared me for
this discovery; yet, momentarily, the consummate villainy of the Greek
had me bereft of speech. As I stood there glaring at him, he began
furtively to grope with one hand along the edge of the _dîwan_ behind
him. Then, suddenly, he became aware of the pistol which I
carried--and abandoned the quest of whatever weapon he had sought,
swallowing audibly.

"So, my good Malaglou," I said, "you sought to make me responsible for
your sins, my friend? I perceive now how the Fates have played with
me. My very first conversation with your charming protégée----"

He bit savagely at his black moustache, advanced upon me; then, his
gaze set upon the Colt, he stood still again.

"... was reported to you by the traitorous Fatimah," I continued
evenly; "and, when, on the morrow, I advised you of my new address,
the identity of the hitherto unknown Romeo who had raised his eyes to
your Juliet became apparent. You doubtless had designed to unpack my
boxes for me as you have been in the habit of doing; but green-eyed
jealousy suggested how, by the sacrifice of only one consignment of
_hashish_, you might wreak my ruin. I disapprove of your morals,
Malaglou. My own code may be peculiar, but it does not embrace
_hashish_ dealing; therefore, Malaglou, you are about to take a sheet
of note-paper--bearing your office heading--and write from my
dictation...."

"And suppose I refuse? You dare not shoot me!"

"You little know my true character, Malaglou. But I should not shoot
you, as you say; I should introduce you to a gentleman who is very
anxious to make your acquaintance--the venerable Sheikh Ismail."

The effect of this remark greatly exceeded my most sanguine
expectations. I think I have never seen a man so pitiably frightened.

"The Sheikh ... Ismail!" gasped Joseph Malaglou. "He is in Cairo?"

"He has generously offered me five thousand pounds for your name and
address."

"Ah, my God!" whispered Malaglou. "Kernaby, you will not betray me
to that fiend? You are an Englishman and you will not soil your hands
with such a deed!"

To my dismay--for it was a disgusting sight--Malaglou fell trembling
upon his knees before me. The threat of shooting had had no such
effect as the mere name of the Sheikh Ismail. My respect for that
really remarkable old ruffian rose by leaps and bounds.

"Get up," I said harshly, "and, if you can, write."

He obeyed me; the man was almost hysterical. And, very shakily, this
is what he wrote:

  "I, Joseph Malaglou, also known as _Ahmed Ben Tawwab_, confess
  that I am a dealer in _hashish_ and spurious antiques, which I have
  been in the habit of storing at my warehouse in Cairo, and also in
  my private residence in the Darb el Ahmar. Finding it desirable to
  enlarge the facilities of the latter, I induced the Hon. Neville
  Kernaby, who is ignorant of my real business, to lease for me a
  house which adjoins my own, as I did not desire it to be known that
  I was the lessee. Subsequently, learning that the suspicions of the
  authorities had been aroused, I anonymously denounced Kernaby, thus
  hoping to avert suspicion from myself and cause his arrest as the
  consignee of the cases which had been delivered at the new
  premises."

"Very good," I said, when this precious document had been completed.
"You understand that you will now accompany me to the central police
station in the Place Bâb el-Khalk and sign this confession in the
presence of suitable witnesses? You will doubtless be detained;
therefore in the interests of your safety, we must arrange that
Mizmûna be hidden securely until the case is settled. Oh! set your
evil mind at rest! I shall not betray you to the Sheikh; unless--"
I looked him squarely in the eyes--"any whisper of my name appears
in this matter!"

"But where is she?" he said hoarsely.

"She is hiding in the adjoining house."

"I have a small place at Shubra where I can conceal her."

"Very well. I will bring her here and permit you to make suitable
arrangements, but let them be complete; for if Ismail should find the
girl and thus discover your identity, nothing could save you--and you
will be unable to leave Cairo (I shall see to that) until the case is
settled."


VI

It was on the following evening, as I sat smoking upon the terrace of
the hotel and reflecting upon the execrably bad luck which pursued me,
that I observed Abû Tabâh mounting the carpeted steps with slow and
stately carriage. He saluted me gravely and accepted the seat which I
offered him.

My plan had run smoothly; Malaglou had given himself up to the
authorities, but had been released upon payment of a substantial bail.
Mizmûna was concealed at Shubra, and I was flogging my brain in a vain
endeavor to conjure up a plan whereby, without betraying the
villainous Greek and thus causing him to betray _me_, I might secure
the Sheikh's reward--or, at least, the lapis armlet.

"Alas," said Abû Tabâh, "that the wicked should prosper."

"To whose prosperity," I inquired, "do you more especially refer?"

He regarded me with his fine melancholy eyes.

"You have an English adage," he continued, "which says, 'set a thief
to catch a thief.'"

"Quite so. But might I inquire what bearing this crystallized wisdom
has upon our present conversation?"

"The man, Joseph Malaglou," he replied, "learning of the hue-and-cry
after a certain missing damsel----"

I remember I was about to light a cigar as he uttered those words, but
a dawning perception of the iniquitous truth crept poisonously into my
mind, and I threw both cigar and matches over the rail into the Shâra
Kâmel and clutched fiercely at the little table between us.

"And of the reward offered for her recovery," pursued the _imám_,
"denounced to us, one Yûssuf of Rosetta, a man owning a small house at
Shubra. Yûssuf had fled, and the only occupant of the place was the
missing damsel Mizmûna. Alas that fortune should so favor the sinful.
The abductor, the despoiler, escapes retribution; and the traitor, the
informer, the dealer in _hashish_ is rewarded."

The Turk has signally failed to rule Egypt; but there are certain
Ottoman institutions which are not without claims, as I realized at
that moment in regard to Joseph Malaglou: I was thinking,
particularly, of the bow-string.

"Already," said Abû Tabâh, with his sweet but melancholy smile, "the
heart of the Sheikh Ismail inclined toward the damsel, for whom his
soul yearned; and has not it been written that he who heals the breach
betwixt man and wife shall himself be blessed? Behold the reward of
the peace-maker--which I design as a gift to my sister."

I was unable to speak, but I became aware of a bitter taste upon my
palate as, from beneath his robe, the smiling _imám_ took out the
armlet of gold and lapis-lazuli!




IV

OMAR OF ISPAHAN


I

"I hear that the Harêm Suite is occupied," said Sir Bertram Collis,
bustling up to me as I sat smoking in the gardens of a certain Cairo
hotel, which I shall not name because of the matters that befell
there. "Daphne is full of curiosity respecting the romantic occupant."

"Don't let Lady Collis be too sure," put in Chundermeyer, "that there
is anything romantic about the occupant."

"Your definition of romance, Chundermeyer," I interrupted, "would
probably be 'a diamond the size of a Spanish onion.'"

Chundermeyer smiled, but it was a smile in which his dark eyes,
twinkling through the pebbles of horn-rimmed spectacles, played no
part. I must confess that the society of this unctuous partner in the
well-known Madras firm of Isaacs and Chundermeyer palled somewhat at
times. He, on the other hand, was eternally dropping into a chair
beside me, and proffering huge and costly cigars from a huge and
costly case. This sort of parvenu persecution is one of the penalties
of being recognized by Debrett.

"As a matter of fact," I continued, "the occupant of the Harêm Suite
is no less romantic a personage than the daughter of the Mudîr
(Governor) of the Fayûm."

"Really!" said Chundermeyer, with that sudden interest which mention
of a title always aroused in him. "Surely it is most unusual for so
highly placed a Moslem lady to reside at an hotel?"

"Most unusual," I replied. "Of course such a thing would be
inconceivable in India; but the management of this establishment, who
cater almost exclusively to tourists, find, I am told, that a 'harêm
suite' is quite a good advertisement. The reason of the presence of
this lady in the hotel is a diplomatic one. She is visiting Cairo in
order to witness the procession of Ashûra, peculiarly sacred to
Egyptian women, and it appears that, having no blood relations here,
she could not accept the hospitality of any one of the big families
without alienating the others."

"By Jove!" said Sir Bertram, "I must tell Daphne this yarn. She'll be
delighted! Come along, Kernaby; if we're to have tea at Mena House, it
is high time we were off."

I left Chundermeyer to his opulent cigar without regret. That he was
an astute man of affairs and an expert lapidary I did not doubt, for
he had offered to buy my Hatshepsu scarab ring at a price exactly ten
per cent below its trade value; but to my mind there is something
almost as unnatural about a Hindu-Hebrew as about a Græco-Welshman or
a griffin.

Of course, Daphne Collis was not ready; and, Sir Bertram going up to
their apartments to induce her to hurry, I strolled out again into the
gardens for a quiet cigarette and a cocktail. As I approached a
suitable seat in a sort of charming little arbor festooned with purple
blossom, a man who had been waiting there rose to greet me.

With a certain quickening of the pulse, I recognized Abû Tabâh,
arrayed, as was his custom, in black, only relieved by a small snowy
turban, which served to enhance the ascetic beauty of his face and the
mystery of the wonderful, liquid eyes.

He inclined his head in that gesture of gentle dignity which I knew;
and:

"I have been awaiting an opportunity of speech with you, Kernaby
Pasha," he said, in his flawless, musical English, "upon a matter in
which I hope you will consent to aid me."

Since this mysterious man, variously known as "the _imám_" and "the
Magician," but whom I knew to be some kind of secret agent of the
Egyptian Government, had recently saved me from assassination, to
decline to aid him was out of the question. We seated ourselves in the
arbor.

"I should welcome an opportunity of serving you, my friend," I assured
him, "since your services to me can never be repaid."

His lips moved slightly in the curiously tender smile which a poor
physiognomist might have mistaken for evidence of effeminacy, bending
towards me with a cautious glance about.

"You are staying at this hotel throughout the Christmas festivities?"
he asked.

"Yes; I have temporarily deserted Shepheard's in order to accept the
hospitality of Sir Bertram Collis, a very old friend. I shall probably
return on the Tuesday following Christmas Day."

"There is to be a carnival and masquerade ball here to-morrow. You
shall be present?"

"I hope so," I replied in surprise. "To what does all this tend?"

Abû Tabâh bent yet closer.

"Many of your friends and acquaintances possess valuable jewels?"

"They do."

"Then warn them--individually, in order to occasion no general
alarm--to guard these with the utmost care."

My surprise increased. "You alarm me," I said. "Are there rogues in
our midst?"

"No," answered the _imám_, fixing his melancholy gaze upon my face;
"so far as my knowledge bears me, there is but one, yet that one is
worse than a host of others."

"Do you mean that he is here--in the hotel?"

Abû Tabâh shrugged his slim shoulders.

"If I knew his exact whereabouts," he replied, "there would be no
occasion to fear him. All that I know is that he is in Cairo; and
since many richly attired women of Europe and America will be here
to-morrow night, of a surety Omar Ali Khân will be here also!"

I shook my head in perplexity.

"Omar Ali Khân?"----I began.

"Ah," continued Abû Tabâh, "to you that name conveys nothing, but to
me it signifies Omar of Ispahân, 'the Father of Thieves.' Do you
remember," fixing his strange eyes hypnotically upon me, "the theft
of the sacred _burko_ of Nefîseh?"

"Quite well," I replied hastily; since the incident represented an
unpleasant memory.

"It was Omar of Ispahân who stole it from the shrine. It was Omar of
Ispahân who stole the blue diamond of the Rajah of Bagore from the
treasure-room at Jullapore, and Omar of Ispahân"--lowering his voice
almost to a whisper--"who stole the Holy Carpet ere it reached Mecca!"

"What!" I cried. "When did that happen? I never heard of such an
episode!"

Abû Tabâh raised his long, slim hand warningly.

"Be cautious!" he whispered; "the flowers of the garden, the palms in
the grove, the very sands of the desert have ears! The lightest word
spoken in the _harêm_ of the Khedive, or breathed from a minaret of
the Citadel, is heard by Omar of Ispahân! The holy covering for the
Kaaba was restored, on payment of a ruinous ransom by the Sherîf of
Mecca, and none save the few ever knew of its loss."

For a time I was silent; words failed me; for the veil of the Kaaba,
miscalled "the Carpet," is about the size of a bowling-green; then--

"In what manner does this affair concern you, Abû Tabâh?" I asked.

"In this way: the daughter of the Mudîr el-Fáyûm is here, in order
that she may be present on the Night of Ashûra in the Mûski. For a
Moslem lady to stay in such a place as this"--there was a faint note
of contempt in the speaker's voice--"is without precedent, but the
circumstances are peculiar. The _khân_ near the Mosque of Hosein is
full, and it is not seemly that the Mudîr's daughter should live at
any lesser establishment. Therefore, as she brings her two servants,
it has been possible for her to remain here. But"--his voice sank
again--"her ornaments are famed throughout Islâm."

I nodded comprehendingly.

"To me," Abû Tabâh whispered, "has been entrusted the task of guarding
them; to you, I entrust that of guarding the possessions of the other
guests!"

I started.

"But, my friend," I said, "this is a dreadful responsibility which you
impose upon me."

"Other precautions are being taken," he replied calmly; "but you,
observing great circumspection, can speak to the guests, and, being
forewarned of his presence, can even watch for the coming of Omar of
Ispahân."


II

The effect of my news upon Lady Collis was truly dramatic.

"Oh," she cried, "my rope of pearls. Mr. Chundermeyer only told me
last week that it was worth at least two hundred pound more than I
gave for it."

Mr. Chundermeyer had made himself popular with many of the ladies in
the hotel by similar diplomatic means, but I think that if he had been
compelled to purchase at his own flattering valuations Messrs. Isaacs
and Chundermeyer would have been ruined.

"You need not wear it, my dear," said her husband tactlessly.

"Don't be so ridiculous!" she retorted. "You know I have brought my
Queen of Sheba costume for to-morrow night."

That, of course, settled the matter, so that beyond making one pretty
woman extremely nervous, my campaign against the dreaded Omar of
Ispahân had opened--blankly. Later in the day I circulated my warning
right and left, and everywhere sowed consternation without reaping any
appreciable result.

"One naturally expects thieves on these occasions," said a little
Chicago millionairess, "and if I only wore my diamonds when no rogues
were about, I might as well have none. There are crooks in America I'd
back against your Persian thief any day."

On the whole, I think, the best audience for my dramatic recitation
was provided by Mr. Chundermeyer, whom I found in the American bar,
just before the dinner hour. His yellow skin perceptibly blanched at
my first mention of Omar Ali Khân, and one hand clutched at a bulging
breast pocket of the dinner-jacket he wore.

"Good heavens, Mr. Kernaby," he said, "you alarm me--you alarm me,
sir!"

"The reputation of Omar is not unknown to you?"

"By no means unknown to me," he responded in the thick, unctuous voice
which betrayed the Semitic strain in his pedigree. "It was this man
who stole the pair of blue diamonds from the Rajah of Bagore."

"So I am told."

"But have you been told that it was my firm who bought those diamonds
for the Rajah?"

"No; that is news to me."

"It was my firm, Mr. Kernaby, who negotiated the sale of the blue
diamonds to the Rajah; therefore the particulars of their loss, under
most extraordinary circumstances, are well known to me. You have made
me very nervous. Who is your informant?"

"A member of the native police with whom I am acquainted."

Mr. Chundermeyer shook his head lugubriously.

"I am conveying a parcel of rough stones to Amsterdam," he confessed,
glancing warily about him over the rims of his spectacles, "and I feel
very much disposed to ask for more reliable protection than is offered
by your Egyptian friend."

"Why not lodge the stones in a bank, or in the manager's safe?"

He shook his head again, and proffered an enormous cigar.

"I distrust all safes but my own," he replied. "I prefer to carry such
valuables upon my person, foolish though the plan may seem to you. But
do you observe that squarely built, military looking person standing
at the bar, in conversation with M. Balabas, the manager?"

"Yes; an officer, I should judge."

"Precisely; a _police_ officer. That is Chief Inspector Carlisle of
New Scotland Yard."

"But he is a guest here."

"Certainly. The management sustained a severe loss last Christmas
during the progress of a ball at which all Cairo was present, and as
the inspector chanced to be on his way home from India, where official
business had taken him, M. Balabas induced him to break his journey
and remain until after the carnival."

"Wait a moment," I said; "I will bring him over."

Crossing to the bar, I greeted Balabas, with whom I was acquainted,
and--

"Mr. Chundermeyer and I have been discussing the notorious Omar of
Ispahân, who is said to be in Cairo," I remarked.

Inspector Carlisle, being introduced, smiled broadly.

"Mr. Balabas is very nervous about this Omar man," he replied, with
a slight Scottish accent; "but, considering that everybody has been
warned, I don't see myself that he can do much damage."

"Perhaps you would be good enough to reassure Mr. Chundermeyer,"
I suggested, "who is carrying valuables."

Chief Inspector Carlisle walked over to the table at which
Chundermeyer was seated.

"I have met your partner, sir," he said, "and I gathered that you
were on your way to Amsterdam with a parcel of rough stones; in fact,
I supposed that you had arrived there by now."

"I am fond of Cairo during the Christmas season," explained the other,
"and I broke my journey. But now I sincerely wish I were elsewhere."

"Oh, I shouldn't worry!" said the detective cheerily. "There are
enough of us on the look-out."

But Mr. Chundermeyer remained palpably uneasy.


III

The gardens of the hotel on the following night presented a fairy-like
spectacle. Lights concealed among the flower-beds, the bloom-covered
arbors, and the feathery leafage of the acacias, suffused a sort of
weird glow, suggesting the presence of a million fire-flies. Up
beneath the crowns of the lofty palms little colored electric lamps
were set, producing an illusion of supernatural fruit, whilst the
fountain had been magically converted into a cascade of fire.

In the ball-room, where the orchestra played, and a hundred mosque
lamps bathed the apartment in soft illumination, a cosmopolitan throng
danced around a giant Christmas tree, their costumes a clash of color
to have filled a theatrical producer with horror, outraging history
and linking the ages in startling fashion. Thus, St. Antony of the
Thebäid danced with Salome, the luresome daughter of Herodias; Nero's
arm was about the waist of Good Queen Bess; Charles II cantered
through a two-step with a red-haired Vestal Virgin; and the Queen of
Sheba (Daphne Collis) had no less appropriate a partner than Sherlock
Holmes.

Doubtless it was all very amusing, but, personally, I stand by my
commonplace dress-suit, having, perhaps, rather a ridiculous sense
of dignity. Inspector Carlisle also was soberly arrayed, and we had
several chats during the evening; he struck me as being a man of
considerable culture and great shrewdness.

For Abû Tabâh I looked in vain. Following our conversation on the
previous afternoon, he had vanished like a figment of a dream. I
several times saw Chundermeyer, who had elected to disguise himself
as Al-Mokanna, the Veiled Prophet of Khorassan. He seemed to be an
enthusiastic dancer, and there was no lack of partners.

But of these mandarins, pierrots, Dutch girls, monks, and court ladies
I speedily tired, and sought refuge in the gardens, whose enchanted
aspect was completed by that wondrous inverted bowl, jewel-studded,
which is the nightly glory of Egypt. In the floral, dim-lighted arbors
many romantic couples shrank from the peeping moon; but quiet and a
hushful sense of peace ruled there beneath the stars more in harmony
with my mood.

One corner of the gardens, in particular, seemed to be quite deserted,
and it was the most picturesque spot of all. For here a graceful palm
upstood before an outjutting _mushrabîyeh_ window, dimly lighted, over
which trailed a wealth of bougainvillea blossom, whilst beneath it lay
a floral carpet, sharply bisected by the shadow of the palm trunk. It
was like some gorgeous illustration to a poem by Hafiz, only lacking
the figure at the window.

And as I stood, enchanted, before the picture, the central panels of
the window were thrown open, and, as if conjured up by my imagination,
a woman appeared, looking out into the gardens--an Oriental woman,
robed in shimmering, moon-kissed white, and wearing a white _yashmak_.
Her arms and fingers were laden with glittering jewels.

I almost held my breath, drawing back into the sheltering shadow, for
I had not hitherto suspected myself of being a sorcerer. For perhaps
a minute, or less, she stood looking out, then the window closed, and
the white phantom disappeared. I recovered myself, recognizing that I
stood before the isolated wing of the hotel known as the Harêm Suite,
and that Fate had granted me a glimpse of the daughter of the Mudîr of
the Fáyûm.

Recollecting, in the nick of time, an engagement to dance with Lady
Collis, I hurried back to the ball-room. On its very threshold I
encountered Chundermeyer. I could see his spectacles glittering
through the veil of his ridiculous costume, and even before he spoke
I detected about him an aura of tragedy.

"Mr. Kernaby," he gasped, "for Heaven's sake help me to find Inspector
Carlisle! I have been robbed!"

"What?"

"My diamonds!"

"You don't mean----"

"Find the inspector, and come to my rooms. I am nearly mad!"

Daphne Collis, who had seen me enter, joined us at this moment, and,
overhearing the latter part of Chundermeyer's speech:

"Oh, whatever is the matter?" she whispered.

As for Chundermeyer the effect upon him of her sudden appearance was
positively magical. He stared through his veil as though her charming
figure had been that of some hideous phantom. Then slowly, as if he
dreaded to find her intangible, he extended one hand and touched her
rope of pearls.

"Ah, heavens!" he gasped. "I am really going mad, or is there a
magician amongst us?"

Daphne Collis's blue eyes opened very widely, and the color slowly
faded from her cheeks.

"Mr. Chundermeyer," she began. But--

"Let us go into this little recess, where there is a good light,"
mumbled Chundermeyer shakily, "and I will make sure."

The three of us entered the palm-screened alcove, Chundermeyer
leading. He stood immediately under a lamp suspended by brass chains
from the roof.

"Permit me to examine your pearls for one moment," he said.

Her hands trembling, Daphne Collis took off the costly ornament and
placed it in the hands of the greatly perturbed expert. Chundermeyer
ran the pearls through his fingers, then lifted the largest of the set
towards the light and scrutinized it closely. Suddenly he dropped his
arms, and extended the necklace upon one open palm.

"Look for yourself," he said slowly. "It does not require the eyes of
an expert."

Daphne Collis snatched the pearls and stared at them dazedly. Her
pretty face was now quite colorless.

"This is not my rope of pearls," she said, in a monotonous voice; "it
is a very poor imitation!"

Ere I could frame any kind of speech--

"Look at this," groaned Chundermeyer, "as you talk of a poor
imitation!"

He was holding out a leather-covered box, plush-lined, and bearing
within the words, "Isaacs and Chundermeyer, Madras." Nestling
grotesquely amid the blue velvet were six small pieces of coal!

Chundermeyer sank upon the cushions of the settee, tossing the casket
upon a little coffee table.

"I am afraid I feel unwell," he said feebly. "Mr. Kernaby, I wonder if
you would be so kind as to find Inspector Carlisle, and ask a waiter
to bring me some cognac."

"Oh, what shall I do, what shall I do?" whispered poor Daphne Collis.

"Just remain here," I said soothingly, "with Mr. Chundermeyer." And I
induced her to sit in a big cane rest-chair. "I will return in a
moment with Bertram and the inspector."

Desiring to avoid a panic, I walked quietly into the ball-room and
took stock of the dancers, for a waltz was in progress. The inspector
I could not see, but Sir Bertram I observed at the further end of the
floor, dancing with Mrs. Van Heysten, the Chicago lady whom I had
warned to keep a close watch upon her diamonds.

I managed to attract Collis's attention, and the pair, quitting the
floor, joined me where I stood. A few words sufficed in which to
inform them of the catastrophe, and, pointing out the alcove wherein
I had left Chundermeyer and Lady Collis, I set off in search of
Inspector Carlisle.

Ten minutes later, having visited every likely spot, I came to the
conclusion that he was not in the hotel, and with M. Balabas I
returned to the alcove adjoining the ball-room. Dancing was in full
swing, and I thought as we passed along the edge of the floor how
easily I could have checked the festivities by announcing that Omar
of Ispahân was present.

The first sight to greet me upon entering the little palm-shaded
alcove was that of Mrs. Van Heysten in tears. She had discovered
herself to be wearing a very indifferent duplicate of her famous
diamond tiara.

I think it was my action of soothingly patting her upon the shoulder
that drew Chundermeyer's attention to my Hatshepsu scarab.

"Mr. Kernaby!" he cried--"Mr. Kernaby!" And pointed to my finger.

I had had the scarab set in a revolving bezel, and habitually wore it
with its beetle uppermost and the cartouche concealed. As I glanced
down at the ring, Chundermeyer stretched out his hand and detached it
from my finger. Approaching the light, he turned the bezel.

The flat part of the scarab was quite blank, bearing no inscription
whatever. Like Lady Collis's rope of pearls, Mrs. Van Heysten's tiara,
and Chundermeyer's diamonds, it was a worthless and very indifferent
duplicate!


IV

Never can I forget the scene in that crowded little room--poor M.
Balabas all anxiety respecting the reputation of his establishment,
and vainly endeavoring to reason with the victims of the amazing Omar
Khân. Finally--

"I will search for Inspector Carlisle myself," said Mr. Chundermeyer;
"and if I cannot find him, I shall be compelled to communicate with
the local police authorities."

M. Balabas still volubly protesting, the unfortunate Veiled Prophet
made his way from the alcove. I cannot say if the inspiration came as
the result of a sort of auto-hypnosis induced by staring at the worthless
ring in my hand--the stone was not even real lapis-lazuli--but a theory
regarding the manner in which these ingenious substitutions had been
effected suddenly entered my mind.

Three minutes later I was knocking at the door of Chundermeyer's room.
I received no invitation to enter, and the door was locked. I sought
M. Balabas; and, without confiding to him the theory upon which I was
acting, I urged the desirability of gaining access to the apartment.
As a result, a master key was procured, and we entered.

At the first glance the room seemed to be empty, though it showed
evidence of having recently been occupied, for it was in the utmost
disorder. Perhaps we should have quitted it unenlightened, if I had
not detected the sound of a faint groan proceeding from the closed
wardrobe. Stepping across the room, I opened the double doors, and
out into my arms fell a limp figure, bound hand and foot, and having
a bath-towel secured tightly around the head to act as a gag. It was
Mr. Chundermeyer!

I think, as I helped to unfasten him, I was the most surprised man in
the land of Egypt. He was arrayed only in a bath-robe and slippers,
and his bare wrists and ankles were cruelly galled by the cords which
had bound him. For some minutes he was unable to utter a word, and
when at last he achieved speech, his first utterance constituted a
verbal thunderbolt.

"I have been robbed!" he cried huskily. "I was sand-bagged as I came
from my bath, and look--everyone of my cases is gone!"

It was M. Balabas who answered him.

"As you returned from your _bath_, Mr. Chundermeyer?" he said. "At
what time was that?"

"About a quarter-past seven," was the amazing reply.

"But, good Heaven!" cried M. Balabas, "I was speaking to you less than
ten minutes ago!"

"You are mad!" groaned Chundermeyer, rubbing his bruised wrists. "Have
I not been locked in the wardrobe all night!"

"Ah, merciful saints," cried M. Balabas, dramatically raising his
clenched fists to heaven, "I see it all! You understand, Mr. Kernaby.
It is _not_ Mr. Chundermeyer with whom we have been conversing, in
whose hands you have been placing your valuables, it is that devil
incarnate who three years ago impersonated the Emîr al-Hadj, in order
to steal the Holy Carpet; who can impersonate anyone; who, it is said,
can transform himself at will into an old woman, a camel, or a fig
tree; it is the conjuror, the wizard--Omar of Ispahân!"

My own ideas were almost equally chaotic; for although, as I now
recalled, I had never throughout the evening obtained a thoroughly
good view of the features of the Veiled Prophet, I could have sworn
to the voice, to the carriage, to the manner of Mr. Chundermeyer.

The puzzling absence of Chief Inspector Carlisle now engaged
everybody's attention; and, acting upon the precedent afforded by the
finding of Mr. Chundermeyer, we paid a visit to the detective's room.

Inspector Carlisle, fully dressed, and still wearing a soft felt hat,
as though he had but just come in, lay on the floor, unconscious, with
the greater part of a cigar, which examination showed to be drugged,
close beside him.

       *       *       *       *       *

As I entered my room that night and switched on the light, in through
the open window from the balcony stepped Abû Tabâh.

His frequent and mysterious appearances in my private apartments did
not surprise me in the least, and I had even ceased to wonder how he
accomplished them; but--

"You are too late, my friend," I said. "Omar of Ispahân has outwitted
you."

"Omar of Ispahân has outwitted men wiser than I," he replied gravely;
"but covetousness is a treacherous master, and I am not without hope
that we may yet circumvent the father of thieves."

"You are surely jesting," I replied. "In all probability he is now far
from Cairo."

"I, on the contrary, have reason to believe," replied Abû Tabâh
calmly, "that he is neither far from Cairo, far from the hotel,
nor far from this very apartment."

His manner was strange and I discovered excitement to be growing
within me.

"Accompany me on the balcony," he said; "but first extinguish the
light."

A moment later I stood looking down upon the moon-bathed gardens,
and Abû Tabâh, beside me, stretched out his hand.

"You see the projecting portion of the building yonder?"

"Yes," I replied; "the Harêm Suite."

"Immediately before the window there is a palm tree."

"I have observed it."

"And upon the opposite side of the path there is an acacia."

"Yes; I see it."

"The moon is high, and whilst all the side of the hotel is in shadow
the acacia is in the moonlight. Its branches would afford concealment,
however; and one watching there could see what would be hidden from
one on this balcony. I request you, Kernaby Pasha, to approach that
_lebbekh_ tree from the further side of the fountain, in order to
remain invisible from the hotel. Climb to one of the lower branches,
and closely watch four windows."

I stared at him in the darkness.

"Which are the four windows that I am to watch?"

"They are--one, that immediately below your own; two, that to the
right of it; three, the window above the Harêm Suite; and, four,
the extreme east window of this wing, on the first floor."

Now, my state of mystification grew even denser. For the windows
specified were, in the order of mention, that of Inspector Carlisle,
who had not yet recovered consciousness; of Mr. Chundermeyer; of
Major Redpath, a retired Anglo-Indian who had been confined to his
room for some time with an attack of malaria; and of M. Balabas,
the manager.

"For what," I inquired, "am I to watch?"

"For a man to descend."

"And then?"

"You will hold your open watch case where it is clearly visible from
this spot. Instant upon the man's appearance you will cover it up,
and then uncover it, either once, twice, thrice, or four times."

"After which?"

"Remain scrupulously concealed. Have the collar of your dinner jacket
turned up in order to betray as little whiteness as possible. Do not
interfere with the man who descends; but if he enters the Harêm Suite,
see that he does not come out again! There is no time for further
explanation, Kernaby Pasha; it is Omar of Ispahân with whom we have
to deal!"


V

Perched up amid the foliage of the acacia, I commenced that singular
guard imposed upon me by Abû Tabâh. Did he suspect one of these four
persons of being the notorious Omar? Or had his mysterious
instructions some other significance? The problem defied me; and,
recognizing that I was hopelessly at sea, I abandoned useless
conjecture and merely watched.

Nor was my vigil a long one. I doubt if I had been at my post for ten
minutes ere a vague figure appeared upon the shadow-veiled balcony of
one of the suspected windows--that of Major Redpath, above the Harêm
Suite!

Scarcely daring to credit my eyes, I saw the figure throw down on to
the projecting top of the _mushrabîyeh_ window below a slender rope
ladder. I covered the gleaming gold of my watch-case with my hand, and
gave the signal--_three_.

The spirit of phantasy embraced me; and, unmoved to further surprise,
I watched the unknown swarm down the ladder with the agility of an
ape. He seemed to wear a robe, surely that of the _Veiled Prophet_!
He silently manipulated one of the side-panels of the window, opened
it, and vanished within the Harêm Suite.

Raising my eyes, I beheld a second figure--that of Abû
Tabâh--descending a similar ladder to the balcony of Inspector
Carlisle's room. He gained the balcony and entered the room. Four
seconds elapsed; he reappeared, unfurled a greater length of ladder,
and came down to the flower-beds. Lithely as a cat he came to the
projecting _mushrabîyeh_, swung himself aloft, and as I watched
breathlessly, expecting him to enter in pursuit of the intruder,
climbed to the top and began to mount the ladder descending from
Major Redpath's room!

He had just reached the major's balcony, and was stepping through
the open window, when a most alarming din arose in the Harêm Suite;
evidently a fierce struggle was proceeding in the apartments of the
Mudîr's daughter!

I scrambled down from the acacia and ran to the spot immediately below
the window, arriving at the very moment that the central lattice was
thrown open, and a white-veiled figure appeared there and prepared to
spring down! Perceiving my approach:

"Oh, help me, in the name of Allah!" cried the woman, in a voice
shrill with fear. "Quick--catch me!"

Ere I could frame any reply, she clutched at the palm tree and dropped
down right into my extended arms, as a crashing of overturned
furniture came from the room above.

"Help them!" she entreated. "You are armed, and my women are being
murdered."

"Help, Kernaby Pasha!" now reached my ears, in the unmistakable voice
of Abû Tabâh, from somewhere within. "See that he does not escape from
the window!"

"Coming!" I cried.

And, by means of the palm trunk, I began to mount towards the open
lattice.

Gaining my objective, I stumbled into a room which presented a scene
of the wildest disorder. It was a large apartment, well but sparsely
furnished in the Eastern manner, and lighted by three hanging lamps.
Directly under one of these, beside an overturned cabinet of richly
carven wood inlaid with mother-o'-pearl, lay a Nubian, insensible,
and arrayed only in shirt and trousers. There was no one else in the
room, and, not pausing to explore those which opened out of it, I ran
and unbolted the heavy door upon which Abû Tabâh was clamoring for
admittance.

The _imám_ leaped into the room, rebolted the door, and glanced to
the right and left; then he ran into the adjoining apartments, and
finally, observing the insensible Nubian upon the floor, he stared
into my face, and I read anger in the eyes that were wont to be so
gentle.

"Did I not enjoin you to prevent his escape from the window?" he
cried.

"No one escaped from the window, my friend," I retorted, "except
the lady who was occupying the suite."

Abû Tabâh fixed his weird eyes upon me in a hypnotic stare of such
uncanny power that I was angrily conscious of much difficulty in
sustaining it; but gradually the quelling look grew less harsh, and
finally his whole expression softened, and that sweet smile, which
could so transform his face, disturbed the severity of the set lips.

"No man is infallible," he said. "And wiser than you or I have shown
themselves the veriest fools in contest with Omar Ali Khân. But know,
O Kernaby Pasha, that the lady who occupied this suite secretly left
it at sunset to-night, bearing her jewels with her, and he"--pointing
to the insensible Nubian on the floor--"took her place and wore her
raiment----"

"Then the Mudîr's daughter----"

"Is my sister Ayesha!"

I looked at him reproachfully, but he met my gaze with calm pride.

"Subterfuge was permitted by the Prophet, (on whom be peace)," he
continued; "but not lying! My sister _is_ the daughter of the Mudîr
el-Fáyûm."

It was a rebuke, perhaps a merited one; and I accepted it in silence.
Although, from the moment that I had first set eyes on him, I had
never doubted Abû Tabâh to be a man of good family, this modest avowal
was something of a revelation.

"Her presence here, which was permitted by my father," he said, "was a
trap; for it is well known throughout the Moslem world that she is the
possessor of costly ornaments. The trap succeeded. Omar of Ispahân, at
great risk of discovery, remained to steal her jewels, although he had
already amassed a choice collection."

Someone had begun to bang upon the bolted door, and there was an
excited crowd beneath the window.

"You supposed, no doubt," the _imám_ resumed calmly, "that I suspected
Major Redpath and M. Balabas, as well as Mr. Chundermeyer and the
English detective? It was not so. But I regarded the room of M.
Balabas as excellently situated for Omar's purpose, and I knew that M.
Balabas rarely retired earlier than one o'clock. Even more suitable
was that of Major Redpath, whose illness I believe to have been due to
some secret art of Omar's."

"But he is down with chronic malaria!"

"It may even be so; yet I believe the attack to have been induced by
Omar of Ispahân."

"But why?"

"Because, as I learned to-night, Major Redpath is the only person in
Cairo who has ever met Mr. Chundermeyer! I will confess that until
less than an hour ago I did not know if Inspector Carlisle was
_really_ an inspector! Oh, it is a seeming absurdity; but Omar of
Ispahân is a wizard! Therefore I entered the inspector's room, and
found him to be still unconscious. Major Redpath was in deep slumber,
and Omar had entered and quitted his room without disturbing him. I
did likewise, and visited Mr. Chundermeyer's--the door was ajar--on
my way downstairs."

"But, my friend," I said amazedly, "with my own eyes I beheld Mr.
Chundermeyer gagged and bound in his wardrobe! I saw his bruised
wrists!"

"He gagged, bound, and bruised himself!" replied Abû Tabâh calmly.
"With my own eyes I once beheld a blind mendicant hanging by the neck
from a fig tree, a bloody froth upon his lips. I cut him down and left
him for dead. Yet was he neither dead nor a blind mendicant; he was
Omar Ali Khân! Oblige me by opening the door, Kernaby Pasha."

I obeyed, and an excited throng burst in, headed by M. Balabas and
Inspector Carlisle, the latter looking very pale and haggard!

"Where is the man posing as Chundermeyer?" began the detective
hoarsely. "By sheer sleight-of-hand, and under ye're very
noses"--excitement rendered him weirdly Caledonian--"he has robbed
ye! I cabled Madras to-day, and the real Chundermeyer arrived at
Amsterdam last Friday! As I returned with the reply cable in my pocket
to-night I became so dizzy I was only just able to get to my room.
He'd doctored every smoke in my case! Where is he?"

"I assisted him to escape, disguised as a woman, some ten minutes
ago," I replied feebly. "I should be sincerely indebted to you if
you would kick me."

"Escaped!" roared Inspector Carlisle. "Then what are ye doing here?
Pursue him, somebody! Are ye all mad?"

"We should be," said Abû Tabâh, "to attempt pursuit. As well pursue
the shadow of a cloud, the first spear of sunrise, or the phantom
heifer of Pepi-Ankh, as pursue Omar of Ispahân! He is gone--but
empty-handed. Behold what I recovered from 'Mr. Chundermeyer's' room."

From beneath his black _gibbeh_ he took out a leather bag, opened it,
and displayed to our startled eyes the tiara of Mrs. Van Heysten, the
rope of pearls, and--my Hatshepsu scarab!

Ere anyone could utter a word, Abû Tabâh inclined his head in
dignified salutation, turned, and walked stately from the room.




V

BREATH OF ALLAH


I

For close upon a week I had been haunting the purlieus of the Mûski,
attired as a respectable dragoman, my face and hands reduced to a
deeper shade of brown by means of a water-color paint (I had to use
something that could be washed off and grease-paint is useless for
purposes of actual disguise) and a neat black moustache fixed to my
lip with spirit-gum. In his story _Beyond the Pale_, Rudyard Kipling
has trounced the man who inquires too deeply into native life; but if
everybody thought with Kipling we should never have had a Lane or a
Burton and I should have continued in unbroken scepticism regarding
the reality of magic. Whereas, because of the matters which I am about
to set forth, for ten minutes of my life I found myself a trembling
slave of the unknown.

Let me explain at once that my undignified masquerade was not prompted
by mere curiosity or the quest of the pomegranate, it was undertaken
as the natural sequel to a letter received from Messrs. Moses, Murphy
and Co., the firm which I represented in Egypt, containing curious
matters affording much food for reflection. "We would ask you," ran
the communication, "to renew your inquiries into the particular
composition of the perfume 'Breath of Allah,' of which you obtained
us a sample at a cost which we regarded as excessive. It appears to
consist in the blending of certain obscure essential oils and
gum-resins; and the nature of some of these has defied analysis to
date. Over a hundred experiments have been made to discover
substitutes for the missing essences, but without success; and as
we are now in a position to arrange for the manufacture of Oriental
perfume on an extensive scale we should be prepared to make it _well
worth your while_ (the last four words characteristically underlined
in red ink) if you could obtain for us a correct copy of the original
prescription."

The letter went on to say that it was proposed to establish a separate
company for the exploitation of the new perfume, with a registered
address in Cairo and a "manufactory" in some suitably inaccessible
spot in the Near East.

I pondered deeply over these matters. The scheme was a good one and
could not fail to reap considerable profits; for, given extensive
advertising, there is always a large and monied public for a new
smell. The particular blend of liquid fragrance to which the letter
referred was assured of a good sale at a high price, not alone in
Egypt, but throughout the capitals of the world, provided it could
be put upon the market; but the proposition of manufacture was beset
with extraordinary difficulties.

The tiny vial which I had despatched to Birmingham nearly twelve
months before had cost me close upon £100 to procure, for the reason
that "Breath of Allah" was the secret property of an old and
aristocratic Egyptian family whose great wealth and exclusiveness
rendered them unapproachable. By dint of diligent inquiry I had
discovered the _attár_ to whom was entrusted certain final processes
in the preparation of the perfume--only to learn that he was ignorant
of its exact composition. But although he had assured me (and I did
not doubt his word) that not one grain had hitherto passed out of
the possession of the family, I had succeeded in procuring a small
quantity of the precious fluid.

Messrs. Moses, Murphy and Co. had made all the necessary arrangements
for placing it upon the market, only to learn, as this eventful letter
advised me, that the most skilled chemists whose services were
obtainable had failed to analyse it.

One morning, then, in my assumed character, I was proceeding along
the Shâria el-Hamzâwi seeking for some scheme whereby I might win
the confidence of Mohammed er-Rahmân the _attár_, or perfumer. I had
quitted the house in the Darb el-Ahmar which was my base of operations
but a few minutes earlier, and as I approached the corner of the
street a voice called from a window directly above my head: "Saïd!
Saïd!"

Without supposing that the call referred to myself, I glanced up,
and met the gaze of an old Egyptian of respectable appearance who
was regarding me from above. Shading his eyes with a gnarled hand--

"Surely," he cried, "it is none other than Saïd the nephew of Yûssuf
Khalig! _Es-selâm 'aleykûm, Saïd!_"

"_Aleykûm, es-selâm_," I replied, and stood there looking up at him.

"Would you perform a little service for me, Saïd?" he continued.
"It will occupy you but an hour and you may earn five piastres."

"Willingly," I replied, not knowing to what the mistake of this
evidently half-blind old man might lead me.

I entered the door and mounted the stairs to the room in which he
was, to find that he lay upon a scantily covered _dîwan_ by the open
window.

"Praise be to Allah (whose name be exalted)!" he exclaimed, "that I am
thus fortunately enabled to fulfil my obligations. I sometimes suffer
from an old serpent bite, my son, and this morning it has obliged me
to abstain from all movement. I am called Abdûl the Porter, of whom you
will have heard your uncle speak; and although I have long retired from
active labor myself, I contract for the supply of porters and carriers
of all descriptions and for all purposes; conveying fair ladies to the
_hammám_, youth to the bridal, and death to the grave. Now, it was
written that you should arrive at this timely hour."

I considered it highly probable that it was also written how I should
shortly depart if this garrulous old man continued to inflict upon me
details of his absurd career. However--

"I have a contract with the merchant, Mohammed er-Rahmân of the Sûk
el-Attârin," he continued, "which it has always been my custom
personally to carry out."

The words almost caused me to catch my breath; and my opinion of Abdul
the Porter changed extraordinary. Truly my lucky star had guided my
footsteps that morning!

"Do not misunderstand me," he added. "I refer not to the transport of
his wares to Suez, to Zagazig, to Mecca, to Aleppo, to Baghdad,
Damascus, Kandahar, and Pekin; although the whole of these vast
enterprises is entrusted to none other than the only son of my father:
I speak, now, of the bearing of a small though heavy box from the
great magazine and manufactory of Mohammed er-Rahmân at Shubra, to his
shop in the Sûk el-Attârin, a matter which I have arranged for him on
the eve of the Molid en-Nebi (birthday of the Prophet) for the past
five-and-thirty years. Every one of my porters to whom I might entrust
this special charge is otherwise employed; hence my observation that
it was written how none other than yourself should pass beneath this
window at a certain fortunate hour."

Fortunate indeed had that hour been for me, and my pulse beat far from
normally as I put the question: "Why, O Father Abdul, do you attach so
much importance to this seemingly trivial matter?"

The face of Abdul the Porter, which resembled that of an intelligent
mule, assumed an expression of low cunning.

"The question is well conceived," he said, raising a long forefinger
and wagging it at me. "And who in all Cairo knows so much of the
secrets of the great as Abdul the Know-all, Abdul the Taciturn! Ask
me of the fabled wealth of Karafa Bey and I will name you every one
of his possessions and entertain you with a calculation of his income,
which I have worked out in _nûss-faddah_![B] Ask me of the amber mole
upon the shoulder of the Princess Azîza and I will describe it to you
in such a manner as to ravish your soul! Whisper, my son"--he bent
towards me confidentially--"once a year the merchant Mohammed
er-Rahmân prepares for the Lady Zuleyka a quantity of the perfume
which impious tradition has called 'Breath of Allah.' The father of
Mohammed er-Rahmân prepared it for the mother of the Lady Zuleyka and
his father before him for the lady of that day who held the
secret--the secret which has belonged to the women of this family
since the reign of the Khalîf el-Hakîm from whose favorite wife they
are descended. To her, the wife of the Khalîf, the first _dirhem_
(drachm) ever distilled of the perfume was presented in a gold vase,
together with the manner of its preparation, by the great wizard and
physician Ibn Sina of Bokhara" (Avicenna).

  [B] A _nûss-faddah_ equals a quarter of a farthing.

"You are well called Abdul the Know-all!" I cried in admiration. "Then
the secret is held by Mohammed er-Rahmân?"

"Not so, my son," replied Abdul. "Certain of the essences employed are
brought, in sealed vessels, from the house of the Lady Zuleyka, as is
also the brass coffer containing the writing of Ibn Sina; and
throughout the measuring of the quantities, the secret writing never
leaves her hand."

"What, the Lady Zuelyka attends in person?"

Abdul the Porter inclined his head serenely.

"On the eve of the birthday of the Prophet, the Lady Zuelyka visits
the shop of Mohammed er-Rahmân, accompanied by an _imám_ from one of
the great mosques."

"Why by an _imám_, Father Abdul?"

"There is a magical ritual which must be observed in the distillation
of the perfume, and each essence is blessed in the name of one of the
four archangels; and the whole operation must commence at the hour of
midnight on the eve of the Molid en-Nebi."

He peered at me triumphantly.

"Surely," I protested, "an experienced _attár_ such as Mohammed
er-Rahmân would readily recognize these secret ingredients by their
smell?"

"A great pan of burning charcoal," whispered Abdul dramatically, "is
placed upon the floor of the room, and throughout the operation the
attendant _imám_ casts pungent spices upon it, whereby the nature of
the secret essences is rendered unrecognizable. It is time you depart,
my son, to the shop of Mohammed, and I will give you a writing making
you known to him. Your task will be to carry the materials necessary
for the secret operation (which takes place to-night) from the
magazine of Mohammed er-Rahmân at Shubra, to his shop in the Sûk
el-Attârin. My eyesight is far from good, Saïd. Do you write as I
direct and I will place my name to the letter."


II

The words "well worth your while" had kept time to my steps, or I
doubt if I should have survived the odious journey from Shubra. Never
can I forget the shape, color, and especially the weight, of the
locked chest which was my burden. Old Mohammed er-Rahmân had accepted
my service on the strength of the letter signed by Abdul, and of
course, had failed to recognize in "Saïd" that Hon. Neville Kernaby
who had certain confidential dealings with him a year before. But
exactly how I was to profit by the fortunate accident which had led
Abdul to mistake me for someone called "Saïd" became more and more
obscure as the box grew more and more heavy. So that by the time that
I actually arrived with my burden at the entrance to the Street of the
Perfumers, my heart had hardened towards Abdul the Know-all; and,
setting my box upon the ground, I seated myself upon it to rest and
to imprecate at leisure that silent cause of my present exhaustion.

After a time my troubled spirit grew calmer, as I sat there inhaling
the insidious breath of Tonquin musk, the fragrance of attár of roses,
the sweetness of Indian spikenard and the stinging pungency of myrrh,
opoponax, and ihlang-ylang. Faintly I could detect the perfume which
I have always counted the most exquisite of all save one--that
delightful preparation of Jasmine peculiarly Egyptian. But the mystic
breath of frankincense and erotic fumes of ambergris alike left me
unmoved; for amid these odors, through which it has always seemed to
me that that of cedar runs thematically, I sought in vain for any hint
of "Breath of Allah."

Fashionable Europe and America were well represented as usual in the
Sûk el-Attârin, but the little shop of Mohammed er-Rahmân was quite
deserted, although he dealt in the most rare essences of all.
Mohammed, however, did not seek Western patronage, nor was there in
the heart of the little white-bearded merchant any envy of his
seemingly more prosperous neighbors in whose shops New York, London,
and Paris smoked amber-scented cigarettes, and whose wares were
carried to the uttermost corners of the earth. There is nothing more
illusory than the outward seeming of the Eastern merchant. The
wealthiest man with whom I was acquainted in the Muski had the aspect
of a mendicant; and whilst Mohammed's neighbors sold phials of essence
and tiny boxes of pastilles to the patrons of Messrs. Cook, were not
the silent caravans following the ancient desert routes laden with
great crates of sweet merchandise from the manufactory at Shubra? To
the city of Mecca alone Mohammed sent annually perfumes to the value
of two thousand pounds sterling; he manufactured three kinds of
incense exclusively for the royal house of Persia; and his wares were
known from Alexandria to Kashmîr, and prized alike in Stambûl and
Tartary. Well might he watch with tolerant smile the more showy
activities of his less fortunate competitors.

The shop of Mohammed er-Rahmân was at the end of the street remote
from the Hamzâwi (Cloth Bazaar), and as I stood up to resume my labors
my mood of gloomy abstraction was changed as much by a certain
atmosphere of expectancy--I cannot otherwise describe it--as by the
familiar smells of the place. I had taken no more than three paces
onward into the Sûk ere it seemed to me that all business had suddenly
become suspended; only the Western element of the throng remained
outside whatever influence had claimed the Orientals. Then presently
the visitors, also becoming aware of this expectant hush as I had
become aware of it, turned almost with one accord, and following the
direction of the merchants' glances, gazed up the narrow street
towards the Mosque of el-Ashraf.

And here I must chronicle a curious circumstance. Of the Imám Abû
Tabâh I had seen nothing for several weeks, but at this moment I
suddenly found myself thinking of that remarkable man. Whilst any
mention of his name, or nickname--for I could not believe "Tabâh" to
be patronymic--amongst the natives led only to pious ejaculations
indicative of respectful fear, by the official world he was tacitly
disowned. Yet I had indisputable evidence to show that few doors in
Cairo, or indeed in all Egypt, were closed to him; he came and went
like a phantom. I should never have been surprised, on entering my
private apartments at Shepheard's, to have found him seated therein,
nor did I question the veracity of a native acquaintance who assured
me that he had met the mysterious _imám_ in Aleppo on the same morning
that a letter from his partner in Cairo had arrived mentioning a visit
by Abû Tabâh to el-Azhar. But throughout the native city he was known
as the Magician and was very generally regarded as a master of the
_ginn_. Once more depositing my burden upon the ground, then, I gazed
with the rest in the direction of the mosque.

It was curious, that moment of perfumed silence, and my imagination,
doubtless inspired by the memory of Abû Tabâh, was carried back to the
days of the great _khalîfs_, which never seem far removed from one in
those mediæval streets. I was transported to the Cairo of Harûn al
Raschîd, and I thought that the Grand Wazîr on some mission from
Baghdad was visiting the Sûk el-Attârin.

Then, stately through the silent group, came a black-robed,
white-turbaned figure outwardly similar to many others in the bazaar,
but followed by two tall muffled negroes. So still was the place that
I could hear the tap of his ebony stick as he strode along the centre
of the street.

At the shop of Mohammed er-Rahmân he paused, exchanging a few words
with the merchant, then resumed his way, coming down the Sûk towards
me. His glance met mine, as I stood there beside the box; and, to my
amazement, he saluted me with smiling dignity and passed on. Had he,
too, mistaken me for Saïd--or had his all-seeing gaze detected beneath
my disguise the features of Neville Kernaby?

As he turned out of the narrow street into the Hamzâwi, the commercial
uproar was resumed instantly, so that save for this horrible doubt
which had set my heart beating with uncomfortable rapidity, by all the
evidences now about me his coming might have been a dream.


III

Filled with misgivings, I carried the box along to the shop; but
Mohammed er-Rahmân's greeting held no hint of suspicion.

"By fleetness of foot thou shalt never win Paradise," he said.

"Nor by unseemly haste shall I thrust others from the path,"
I retorted.

"It is idle to bandy words with any acquaintance of Abdul the
Porter's," sighed Mohammed; "well do I know it. Take up the box
and follow me."

With a key which he carried attached to a chain about his waist,
he unlocked the ancient door which alone divided his shop from the
outjutting wall marking a bend in the street. A native shop is usually
nothing more than a double cell; but descending three stone steps,
I found myself in one of those cellar-like apartments which are not
uncommon in this part of Cairo. Windows there were none, if I except
a small square opening, high up in one of the walls, which evidently
communicated with the narrow courtyard separating Mohammed's
establishment from that of his neighbor, but which admitted scanty
light and less ventilation. Through this opening I could see what
looked like the uplifted shafts of a cart. From one of the rough
beams of the rather lofty ceiling a brass lamp hung by chains, and
a quantity of primitive chemical paraphernalia littered the place;
old-fashioned alembics, mysterious looking jars, and a sort of
portable furnace, together with several tripods and a number of large,
flat brass pans gave the place the appearance of some old alchemist's
den. A rather handsome ebony table, intricately carved and inlaid with
mother-o'-pearl and ivory, stood before a cushioned _dîwan_ which
occupied that side of the room in which was the square window.

"Set the box upon the floor," directed Mohammed, "but not with such
undue dispatch as to cause thyself to sustain an injury."

That he had been eagerly awaiting the arrival of the box and was now
burningly anxious to witness my departure, grew more and more apparent
with every word. Therefore--

"There are asses who are fleet of foot," I said, leisurely depositing
my load at his feet; "but the wise man regulateth his pace in
accordance with three things: the heat of the sun; the welfare of
others; and the nature of his burden."

"That thou hast frequently paused on the way from Shubra to reflect
upon these three things," replied Mohammed, "I cannot doubt; depart,
therefore, and ponder them at leisure, for I perceive that thou art
a great philosopher."

"Philosophy," I continued, seating myself upon the box, "sustaineth
the mind, but the activity of the mind being dependent upon the
welfare of the stomach, even the philosopher cannot afford to labor
without hire."

At that, Mohammed er-Rahmân unloosed upon me a long pent-up torrent of
invective--and furnished me with the information which I was seeking.

"O son of a wall-eyed mule!" he cried, shaking his fists over me, "no
longer will I suffer thy idiotic chatter! Return to Abdul the Porter,
who employed thee, for not one _faddah_ will I give thee, calamitous
mongrel that thou art! Depart! for I was but this moment informed that
a lady of high station is about to visit me. Depart! lest she mistake
my shop for a pigsty."

But even as he spoke the words, I became aware of a vague disturbance
in the street, and--

"Ah!" cried Mohammed, running to the foot of the steps and gazing
upwards, "now am I utterly undone! Shame of thy parents that thou
art, it is now unavoidable that the Lady Zuleyka shall find thee
in my shop. Listen, offensive insect--thou art Saïd, my assistant.
Utter not one word; or with this"--to my great alarm he produced
a dangerous-looking pistol from beneath his robe--"will I blow a
hole through thy vacuous skull!"

Hastily concealing the pistol, he went hurrying up the steps, in time
to perform a low salutation before a veiled woman who was accompanied
by a Sûdanese servant-girl and a negro. Exchanging some words with her
which I was unable to detect, Mohammed er-Rahmân led the way down into
the apartment wherein I stood, followed by the lady, who in turn was
followed by her servant. The negro remained above. Perceiving me as
she entered, the lady, who was attired with extraordinary elegance,
paused, glancing at Mohammed.

"My lady," he began immediately, bowing before her, "it is Saïd my
assistant, the slothfulness of whose habits is only exceeded by the
impudence of his conversation."

She hesitated, bestowing upon me a glance of her beautiful eyes.
Despite the gloom of the place and the _yashmak_ which she wore, it
was manifest that she was good to look upon. A faint but exquisite
perfume stole to my nostrils, whereby I knew that Mohammed's charming
visitor was none other than, the Lady Zuleyka.

"Yet," she said softly, "he hath the look of an active young man."

"His activity," replied the scent merchant, "resideth entirely in
his tongue."

The Lady Zuleyka seated herself upon the _dîwan_, looking all about
the apartment.

"Everything is in readiness, Mohammed?" she asked.

"Everything, my lady."

Again the beautiful eyes were turned in my direction, and, as their
inscrutable gaze rested upon me, a scheme--which, since it was never
carried out, need not be described--presented itself to my mind.
Following a brief but eloquent silence--for my answering glances were
laden with significance:--

"O Mohammed," said the Lady Zuleyka indolently, "in what manner doth
a merchant, such as thyself, chastise his servants when their conduct
displeaseth him?"

Mohammed er-Rahmân seemed somewhat at a loss for a reply, and stood
there staring foolishly.

"I have whips for mine," murmured the soft voice. "It is an old custom
of my family."

Slowly she cast her eyes in my direction once more.

"It seemed to me, O Saïd," she continued, gracefully resting one
jeweled hand upon the ebony table, "that thou hadst presumed to cast
love-glances upon me. There is one waiting above whose duty it is to
protect me from such insults. Miska!"--to the servant girl--"summon
El-Kimri (The Dove)."

Whilst I stood there dumbfounded and abashed the girl called up the
steps:

"El-Kimri! Come hither!"

Instantly there burst into the room the form of that hideous negro
whom I had glimpsed above; and--

"O Kimri," directed the Lady Zuleyka, and languidly extended her hand
in my direction, "throw this presumptuous clown into the street!"

My discomfiture had proceeded far enough, and I recognized that, at
whatever risk of discovery, I must act instantly. Therefore, at the
moment that El-Kimri reached the foot of the steps, I dashed my left
fist into his grinning face, putting all my weight behind the blow,
which I followed up with a short right, utterly outraging the
pugilistic proprieties, since it was well below the belt. El-Kimri bit
the dust to the accompaniment of a human discord composed of three
notes--and I leaped up the steps, turned to the left, and ran off
around the Mosque of el-Ashraf, where I speedily lost myself in the
crowded Ghurîya.

Beneath their factitious duskiness my cheeks were burning hotly: I was
ashamed of my execrable artistry. For a druggist's assistant does not
lightly make love to a duchess!


IV

I spent the remainder of the forenoon at my house in the Darb el-Ahmar
heaping curses upon my own fatuity and upon the venerable head of
Abdul the Know-all. At one moment it seemed to me that I had wantonly
destroyed a golden opportunity, at the next that the seeming
opportunity had been a mere mirage. With the passing of noon and the
approach of evening I sought desperately for a plan, knowing that if
I failed to conceive one by midnight, another chance of seeing the
famous prescription would probably not present itself for twelve
months.

At about four o'clock in the afternoon came the dawn of a hazy idea,
and since it necessitated a visit to my rooms at Shepheard's, I washed
the paint off my face and hands, changed, hurried to the hotel, ate a
hasty meal, and returned to the Darb el-Ahmar, where I resumed my
disguise.

There are some who have criticized me harshly in regard to my
commercial activities at this time, and none of my affairs has
provoked greater acerbitude than that of the perfume called "Breath
of Allah." Yet I am at a loss to perceive wherein my perfidy lay; for
my outlook is sufficiently socialistic to cause me to regard with
displeasure the conserving by an individual of something which,
without loss to himself, might reasonably be shared by the community.
For this reason I have always resented the way in which the Moslem
veils the faces of the pearls of his _harêm_. And whilst the success
of my present enterprise would not render the Lady Zuleyka the poorer,
it would enrich and beautify the world by delighting the senses of men
with a perfume more exquisite than any hitherto known.

Such were my reflections as I made my way through the dark and
deserted bazaar quarter, following the Shâria el-Akkadi to the Mosque
of el-Ashraf. There I turned to the left in the direction of the
Hamzâwi, until, coming to the narrow alley opening from it into the
Sûk el-Attârin, I plunged into its darkness, which was like that of
a tunnel, although the upper parts of the houses above were silvered
by the moon.

I was making for that cramped little courtyard adjoining the shop of
Mohammed er-Rahmân in which I had observed the presence of one of
those narrow high-wheeled carts peculiar to the district, and as the
entrance thereto from the Sûk was closed by a rough wooden fence I
anticipated little difficult in gaining access. Yet there was one
difficulty which I had not foreseen, and which I had not met with had
I arrived, as I might easily have arranged to do, a little earlier.
Coming to the corner of the Street of the Perfumers, I cautiously
protruded my head in order to survey the prospect.

Abû Tabâh was standing immediately outside the shop of Mohammed
er-Rahmân!

My heart gave a great leap as I drew back into the shadow, for I
counted his presence of evil omen to the success of my enterprise.
Then, a swift revelation, the truth burst in upon my mind. He was
there in the capacity of _imám_ and attendant magician at the mystical
"Blessing of the perfumes"! With cautious tread I retraced my steps,
circled round the Mosque and made for the narrow street which runs
parallel with that of the Perfumers and into which I knew the
courtyard beside Mohammed's shop must open. What I did not know was
how I was going to enter it from that end.

I experienced unexpected difficulty in locating the place, for the
height of the buildings about me rendered it impossible to pick up
any familiar landmark. Finally, having twice retraced my steps, I
determined that a door of old but strong workmanship set in a high,
thick wall must communicate with the courtyard; for I could see no
other opening to the right, or left through which it would have been
possible for a vehicle to pass.

Mechanically I tried the door, but, as I had anticipated, found it to
be securely locked. A profound silence reigned all about me and there
was no window in sight from which my operations could be observed.
Therefore, having planned out my route, I determined to scale the
wall. My first foothold was offered by the heavy wooden lock which
projected fully six inches from the door. Above it was a crossbeam and
then a gap of several inches between the top of the gate and the arch
into which it was built. Above the arch projected an iron rod from
which depended a hook; and if I could reach the bar it would be
possible to get astride the wall.

I reached the bar successfully, and although it proved to be none too
firmly fastened, I took the chance and without making very much noise
found myself perched aloft and looking down into the little court. A
sigh of relief escaped me; for the narrow cart with its
disproportionate wheels stood there as I had seen it in the morning,
its shafts pointing gauntly upward to where the moon of the Prophet's
nativity swam in a cloudless sky. A dim light shone out from the
square window of Mohammed er-Rahmân's cellar.

Having studied the situation very carefully, I presently perceived to
my great satisfaction that whilst the tail of the cart was wedged
under a crossbar, which retained it in its position, one of the
shafts was in reach of my hand. Thereupon I entrusted my weight to the
shaft, swinging out over the well of the courtyard. So successful was
I that only a faint creaking sound resulted; and I descended into the
vehicle almost silently.

Having assured myself that my presence was undiscovered by Abû Tabâh,
I stood up cautiously, my hands resting upon the wall, and peered
through the little window into the room. Its appearance had changed
somewhat. The lamp was lighted and shed a weird and subdued
illumination upon a rough table placed almost beneath it. Upon this
table were scales, measures, curiously shaped flasks, and odd-looking
chemical apparatus which might have been made in the days of Avicenna
himself. At one end of the table stood an alembic over a little pan in
which burnt a spirituous flame. Mohammed er-Rahmân was placing
cushions upon the _dîwan_ immediately beneath me, but there was no one
else in the room. Glancing upward, I noted that the height of the
neighboring building prevented the moonlight from penetrating into the
courtyard, so that my presence could not be detected by means of any
light from without; and, since the whole of the upper part of the room
was shadowed, I saw little cause for apprehension within.

At this moment came the sound of a car approaching along the Shâria
esh-Sharawâni. I heard it stop, near the Mosque of el-Ashraf, and in
the almost perfect stillness of those tortuous streets from which by
day arises a very babel of tongues I heard approaching footsteps. I
crouched down in the cart, as the footsteps came nearer, passed the
end of the courtyard abutting on the Street of the Perfumers, and
paused before the shop of Mohammed er-Rahmân. The musical voice of
Abû Tabâh spoke and that of the Lady Zuleyka answered. Came a loud
rapping, and the creak of an opening door: then--

"Descend the steps, place the coffer on the table, and then remain
immediately outside the door," continued the imperious voice of the
lady. "Make sure that there are no eavesdroppers."

Faintly through the little window there reached my ears a sound as of
some heavy object being placed upon a wooden surface, then a muffled
disturbance as of several persons entering the room; finally, the
muffled bang of a door closed and barred ... and soft footsteps in
the adjoining street!

Crouching down in the cart and almost holding my breath, I watched
through a hole in the side of the ramshackle vehicle that fence to
which I have already referred as closing the end of the courtyard
which adjoined the Sûk el-Attârin. A spear of moonlight, penetrating
through some gap in the surrounding buildings, silvered its extreme
edge. To an accompaniment of much kicking and heavy breathing, into
this natural limelight arose the black countenance of "The Dove."
To my unbounded joy I perceived that his nose was lavishly decorated
with sticking-plaster and that his right eye was temporarily off duty.
Eight fat fingers clutching at the top of the woodwork, the bloated
negro regarded the apparently empty yard for a space of some three
seconds, ere lowering his ungainly bulk to the level of the street
again. Followed a faint "pop" and a gurgling quite unmistakable. I
heard him walking back to the door, as I cautiously stood up and
again surveyed the interior of the room.


V

Egypt, as the earliest historical records show, has always been a land
of magic, and according to native belief it is to-day the theater of
many super-natural dramas. For my own part, prior to the episode which
I am about to relate, my personal experiences of the kind had been
limited and unconvincing. That Abû Tabâh possessed a sort of uncanny
power akin to second sight I knew, but I regarded it merely as a form
of telepathy. His presence at the preparation of the secret perfume
did not surprise me, for a belief in the efficacy of magical
operations prevailed, as I was aware, even among the more cultured
Moslems. My scepticism, however, was about to be rudely shaken.

As I raised my head above the ledge of the window and looked into the
room, I perceived the Lady Zuleyka seated on the cushioned _dîwan_,
her hands resting upon an open roll of parchment which lay upon the
table beside a massive brass chest of antique native workmanship. The
lid of the chest was raised, and the interior seemed to be empty, but
near it upon the table I observed a number of gold-stoppered vessels
of Venetian glass and each of which was of a different color.

Beside a brazier wherein glowed a charcoal fire, Abû Tabâh stood;
and into the fire he cast alternately strips of paper bearing writing
of some sort and little dark brown pastilles which he took from
a sandalwood box set upon a sort of tripod beside him. They were
composed of some kind of aromatic gum in which benzoin seemed to
predominate, and the fumes from the brazier filled the room with
a blue mist.

The _imám_, in his soft, musical voice, was reciting that chapter of
the Korân called "The Angel." The weird ceremony had begun. In order
to achieve my purpose I perceived that I should have to draw myself
right up to the narrow embrasure and rest my weight entirely upon the
ledge of the window. There was little danger in the maneuver, provided
I made no noise; for the hanging lamp, by reason of its form, cast no
light into the upper part of the room. As I achieved the desired
position I became painfully aware of the pungency of the perfume with
which the apartment was filled.

Lying there upon the ledge in a most painful attitude, I wriggled
forward inch by inch further into the room, until I was in a position
to use my right arm more or less freely. The preliminary prayer
concluded, the measuring of the perfumes had now actually commenced,
and I readily perceived that without recourse to the parchment, from
which the Lady Zuleyka never once removed her hands, it would indeed
be impossible to discover the secret. For, consulting the ancient
prescription, she would select one of the gold-stoppered bottles,
unscrew it, direct that so many grains should be taken from it, and
never removing her gaze from Mohammed er-Rahmân whilst he measured
out the correct quantity, would restopper the vessel and so proceed.
As each was placed in a wide-mouthed glass jar by the perfumer, Abû
Tabâh, extending his hands over the jar, pronounced the names:

"Gabraîl Mikaîl, Israfîl, Israîl."

Cautiously I raised to my eyes the small but powerful opera-glasses
to procure which I had gone to my rooms at Shepheard's. Focussing them
upon the ancient scroll lying on the table beneath me, I discovered,
to my joy, that I could read the lettering quite well. Whilst Abû
Tabâh began to recite some kind of incantation in the course of which
the names of the Companions of the Prophet frequently occurred, I
commenced to read the writing of Avicenna.

"In the name of God, the Compassionate, the Merciful, the High, the
Great...."

So far had I proceeded and no further when I became aware of a curious
change in the form of the Arabic letters. They seemed to be moving, to
be cunningly changing places one with another as if to trick me out of
grasping their meaning!

The illusion persisting, I determined that it was due to the unnatural
strain imposed upon my vision, and although I recognized that time
was precious I found myself compelled temporarily to desist, since
nothing was to be gained by watching these letters which danced from
side to side of the parchment, sometimes in groups and sometimes
singly, so that I found myself pursuing one slim Arab A (_'Alif_)
entirely up the page from the bottom to the top where it finally
disappeared under the thumb of the Lady Zuleyka!

Lowering the glasses I stared down in stupefaction at Abû Tabâh. He
had just cast fresh incense upon the flames, and it came home to me,
with a childish and unreasoning sense of terror, that the Egyptians
who called this man the Magician were wiser than I. For whilst I could
no longer hear his voice, I now could _see_ the words issuing from his
mouth! They formed slowly and gracefully in the blue clouds of vapour
some four feet above his head, revealed their meaning to me in letters
of gold, and then faded away towards the ceiling!

Old-established beliefs began to totter about me as I became aware of
a number of small murmuring voices within the room. They were the
voices of the perfumes burning in the brazier. Said one, in a guttural
tone:

"I am Myrrh. My voice is the voice of the Tomb."

And another softly: "I am Ambergris. I lure the hearts of men."

And a third huskily: "I am Patchouli. My promises are lies."

My sense of smell seemed to have deserted me and to have been
replaced by a sense of hearing. And now this room of magic began to
expand before my eyes. The walls receded and receded, until the
apartment grew larger than the interior of the Citadel Mosque; the
roof shot up so high that I knew there was no cathedral in the world
half so lofty. Abû Tabâh, his hands extended above the brazier, shrank
to minute dimensions, and the Lady Zuleyka, seated beneath me, became
almost invisible.

The project which had led me to thrust myself into the midst of this
feast of sorcery vanished from my mind. I desired but one thing: to
depart, ere reason utterly deserted me. But, to my horror, I
discovered that my muscles were become rigid bands of iron! The figure
of Abû Tabâh was drawing nearer; his slowly moving arms had grown
serpentine and his eyes had changed to pools of flame which seemed to
summon me. At the time when this new phenomenon added itself to the
other horrors, I seemed to be impelled by an irresistible force to
jerk my head downwards: I heard my neck muscles snap metallically: I
_saw_ a scream of agony spurt forth from my lips ... and I saw upon a
little ledge immediately below the square window a little _mibkharah_,
or incense burner, which hitherto I had not observed. A thick, oily
brown stream of vapor was issuing from its perforated lid and bathing
my face clammily. Sense of smell I had none; but a chuckling,
demoniacal voice spoke from the _mibkharah_, saying--

"I am _Hashish_! I drive men mad! Whilst thou hast lain up there like
a very fool, I have sent my vapors to thy brain and stolen thy senses
from thee. It was for this purpose that I was set here beneath the
window where thou couldst not fail to enjoy the full benefit of my
poisonous perfume...."

Slipping off the ledge, I fell ... and darkness closed about me.


VI

My awakening constitutes one of the most painful recollections of a
not uneventful career; for, with aching head and tortured limbs, I sat
upright upon the floor of a tiny, stuffy, and uncleanly cell! The only
light was that which entered by way of a little grating in the door.
I was a prisoner; and, in the same instant that I realized the fact
of my incarceration, I realized also that I had been duped. The weird
happenings in the apartment of Mohammed er-Rahmân had been
hallucinations due to my having inhaled the fumes of some preparation
of _hashish_, or Indian hemp. The characteristic sickly odor of the
drug had been concealed by the pungency of the other and more
odoriferous perfumes; and because of the position of the censer
containing the burning _hashish_, no one else in the room had been
affected by its vapor. Could it have been that Abû Tabâh had known
of my presence from the first?

I rose, unsteadily, and looked out through the grating into a narrow
passage. A native constable stood at one end of it, and beyond him I
obtained a glimpse of the entrance hall. Instantly I recognized that
I was under arrest at the Bâb el-Khalk police station!

A great rage consumed me. Raising my fists I banged furiously upon
the door, and the Egyptian policeman came running along the passage.

"What does this mean, _shawêsh_?" I demanded. "Why am I detained here?
I am an Englishman. Send the superintendent to me instantly."

The policeman's face expressed alternately anger, surprise, and
stupefaction.

"You were brought here last night, most disgustingly and speechlessly
drunk, in a cart!" he replied.

"I demand to see the superintendent."

"Certainly, certainly, _effendim_!" cried the man, now thoroughly
alarmed. "In an instant, _effendim_!"

Such is the magical power of the word "Inglîsi" (Englishman).

A painfully perturbed and apologetic native official appeared almost
immediately, to whom I explained that I had been to a fancy dress ball
at the Gezira Palace Hotel, and, injudiciously walking homeward at a
late hour, had been attacked and struck senseless. He was anxiously
courteous, sending a man to Shepheard's with my written instructions
to bring back a change of apparel and offering me every facility for
removing my disguise and making myself presentable. The fact that he
palpably disbelieved my story did not render his concern one whit the
less.

I discovered the hour to be close upon noon, and, once more my outward
self, I was about to depart from the Place Bâb el-Khalk, when, into
the superintendent's room came Abû Tabâh! His handsome ascetic face
exhibited grave concern as he saluted me.

"How can I express my sorrow, Kernaby Pasha," he said in his soft
faultless English, "that so unfortunate and unseemly an accident
should have befallen you? I learned of your presence here but a few
moments ago, and I hastened to convey to you an assurance of my
deepest regret and sympathy."

"More than good of you," I replied. "I am much indebted."

"It grieves me," he continued suavely, "to learn that there are
footpads infesting the Cairo streets, and that an English gentleman
may not walk home from a ball safely. I trust that you will provide
the police with a detailed account of any valuables which you may have
lost. I have here"--thrusting his hand into his robe--"the only item
of your property thus far recovered. No doubt you are somewhat
short-sighted, Kernaby Pasha, as I am, and experience a certain
difficulty in discerning the names of your partners upon your dance
programme."

And with one of those sweet smiles which could so transfigure his
face, Abû Tabâh handed me my opera-glasses!




VI

THE WHISPERING MUMMY


I

Felix Bréton and I were the only occupants of the raised platform
at the end of the hall; and the inartistic performance of the bulky
dancer who occupied the stage promised to be interminable. From
motives of sheer boredom I studied the details of her dress--a white
dress, fitting like a vest from shoulder to hip, and having short,
full sleeves under which was a sort of blue gauze. Her hair, wrists,
and ankles glittered with barbaric jewelery and strings of little
coins.

A deafening orchestra consisting of tambourines, shrieking Arab viols,
and the inevitable _daràbukeh_, surrounded the performer in a
half-circle; and three other large-sized _ghawâzi_ mingled their
shrill voices with the barbaric discords of the musicians. I yawned.

"As a quest of local color, Bréton," I said, "this evening's
expedition can only be voted a dismal failure."

Felix Bréton turned to me, with a smile, resting his elbows upon the
dirty little marble-topped table. He looked sufficiently like an
artist to have been merely a painter; yet his gruesome picture "Le
Roi S'Amuse" had proved the salvation of the previous Salon.

"Have patience," he said; "it is Shejeret ed-Durr (Tree of Pearls)
that we have come to see, and she has not yet appeared."

"Unless she appears shortly," I replied, stifling another yawn,
"I shall disappear."

But even as I spoke, there arose a hum of excitement throughout
the crowded room; the fat dancer, breathless from her unpleasing
exertions, resumed her seat; and all the performers turned their
heads towards a door at the side of the stage. A veiled figure
entered, with slow, lithe step; and her appearance was acclaimed
excitedly. Coming to the centre of the stage, she threw off her
veil with a swift movement, and confronted the audience, a slim,
barbaric figure. I glanced at Felix Bréton. His eyes were glittering
with excitement. Here at last was the _ghazîyeh_ of romance, the
_ghazîyeh_ of the Egyptian monuments; a true daughter of that
mysterious tribe who, in the remote past of the Nile-land, wove
spells of subtle moon-magic before the golden Pharaoh.

A monstrous crash from the musicians opened the music of the
dance--the famous Gazelle dance--which commenced to a measure of
long, monotonous cadences. Shejeret ed-Durr began slowly to move her
arms and body in that indescribable manner which, like the stirring
of palm fronds, speaks the veritable language of the voluptuous Orient.
The attendant dancers clashing their miniature cymbals, the measure
quickened, and swift passion informed the languorous body, which
magically became transformed into that of a leaping nymph, a
bacchante, a living illustration of Keats' wonder-words:

    "Like to a moving vintage, down they came,
     Crown'd with green leaves, and faces all aflame;
     All madly dancing through the pleasant valley,
         To scare thee, Melancholy!"

At the conclusion of her dance, Shejeret ed-Durr, resuming her veil,
descended to the floor of the hall and passed from table to table,
exchanging light badinage with those patrons known to her.

"Do you think you could induce her to come up here, Kernaby?" said
Bréton excitedly; "she is simply the ideal model for my 'Danse
Funébre.'"

"Any inducement other than our presence in this select part of the
establishment," I replied, offering him a cigarette, "is unnecessary.
She will present herself with all reasonable despatch."

Indeed, I had seen the dark eyes glance many times towards us, as we
sat there in distinguished isolation; and, even as I spoke, the girl
was ascending the steps, from whence she approached our table, smiling
in friendly fashion. Bréton's surprise was rather amusing when she
confidently seated herself, giving an order to the cross-eyed waiter
in close attendance. It would be our privilege, of course, to pay the
bill. Of its being a privilege, no one could doubt who had observed
the envious glances cast in our direction by less favored patrons.

As Bréton spoke no Arabic, the task of interpreter devolved upon me;
and I was carrying on quite mechanically when my attention was drawn
to a peculiarly sinister-looking person seated alone at a table close
beside the corner of the stage. I remembered having observed him
address some remark to Shejeret ed-Durr, and having noted that she
seemed to avoid him. Now, he was directing upon us a glare so
electrically baleful that when I first detected it I was conscious of
a sort of shock. The man was rather oddly dressed, wearing a black
turban and a sort of loose robe not unlike the _burnûs_ of the desert
Arabs. I concluded that he belonged to some religious order, and that
his bosom was inflamed with a hatred of a most murderous character
towards myself, Felix Bréton, and the dancer.

I endeavored, without attracting the girl's notice to indicate to
Bréton the presence of the Man of the Glare; but the artist was so
engrossed in contemplation of Shejeret ed-Durr and kept me so busy
interpreting, that I abandoned the attempt in despair. Having made his
wishes evident to her, the girl readily consented to pose for him; and
when next I glanced at the table near the stage, the Man of the Glare
had disappeared.

What induced me to look towards the rear of the platform upon which
we were seated I know not, unless I did so in obedience to a species
of hypnotic suggestion; but something prompted me to glance over my
shoulder. And, for the second time that night, I encountered the gaze
of mysterious eyes. From a little square window these compelling eyes
regarded me fixedly, and presently I distinguished the outline of a
head surmounted by a white turban.

The second watcher was Abû Tabâh!

What business could have brought the mysterious _imám_ to such a place
was a problem beyond my powers of conjecture, but that he was silently
directing me to depart with all speed I presently made out. Having
signified, by a gesture, that I had grasped the purport of his
message, I turned again to Bréton, who was struggling to carry on a
conversation with Shejeret ed-Durr in his native French.

I experienced some difficulty in inducing him to leave, but my
arguments finally prevailed, and we passed out into the dimly lighted
street. About us in the darkness pipes wailed, and there was the dim
throbbing of the eternal _darábukeh_. We were in that part of El-Wasr
adjoining the notorious Square of the Fountain. Discordant woman
voices filled the night, and strange figures flitted from the shadows
into the light streaming from the open doorways. It was the centre of
secret Cairo, the midnight city; and three paces from the door of the
dance hall, a slim, black-robed figure suddenly appeared at my elbow,
and the musical voice of Abû Tabâh spoke close to my ear:

"Be on the terrace of Shepheard's in half an hour."

The mysterious figure melted again into the shadows about us.


II

On the deserted hotel balcony, Abû Tabâh awaited me.

"It was indeed fortunate, Kernaby Pasha," he said, "that I observed
you this evening."

"I am greatly obliged to you," I replied, "for watching over me with
such paternal solicitude. May I inquire what danger I have incurred?"

I was angrily conscious of feeling like a schoolboy suffering reproof.

"A very great danger," Abû Tabâh assured me, his gentle, musical voice
expressing real concern. "Ahmad es-Kebîr is the lover of the dancer
called Shejeret ed-Durr, although she who is of the _ghawâzi_, of
Keneh does not return his affections."

"Ahmad es-Kebîr?--do you refer to a malignant looking person in a
black turban?" I inquired.

Abû Tabâh gravely inclined his head.

"He is one of the _Rifa'îyeh_, the Black _Darwîshes_. They practise
strange rites and are by some accredited with supernatural powers. For
you the danger is not so great as for your friend, who seemed to be
speaking words of love to the _ghazîyeh_."

I laughed shortly.

"You are mistaken, Abû Tabâh," I replied; "his interest was not of the
character which you suppose. He is an artist and merely desired the
girl to pose for him."

Abû Tabâh shrugged his shoulders.

"She is an unveiled woman," he said contemptuously, "but love in the
heart of such a one as Ahmad is a terrible passion, consuming the
vitals and rendering whom it afflicts either a partaker of Paradise
or as one of the evil _ginn_."

"In the particular case under consideration," I said, "it would seem
distinctly to have produced the latter and less agreeable symptoms."

"Let your friend step warily," advised Abû Tabâh; "for some who have
aroused the enmity of the Black _Darwîshes_ have met with strange
ends, nor has it been possible to fix responsibility upon any member
of the order."

"You think my poor friend, Felix Bréton, may be discovered some
morning in an unpleasantly messy condition?"

"The Black _Darwîshes_ do not employ the knife," answered Abû Tabâh;
"they employ strange and more subtle weapons."

I stared hard at him in the darkness. I thought I knew my Cairo, but
this sounded unpleasantly mysterious. However--

"I am indebted to you, Abû Tabâh," I said, "for your timely warning.
As you know, I always personally avoid any possibility of
misunderstanding in regard to my relations with Egyptian womenfolk."

"With some rare exceptions," agreed Abû Tabâh, "particulars of which
escape my memory at the moment, you have always been a model of
discretion, Kernaby Pasha."

"I will warn my friend," I said hastily, "of the view of his conduct
mistakenly taken by the gentleman in the black turban."

"It is well," replied Abû Tabâh; "we shall meet again ere long."

With that and the customary dignified salutations he departed, leaving
me wondering what hidden significance lay in his words, "we shall meet
again ere long."

Experience had taught me that Abû Tabâh's warnings were not to be
lightly dismissed, and I knew enough of the fanaticism of those
strange Eastern sects whereof the _Rifa'îyeh_, or Black _Darwîshes_,
was one, to realize that it would prove an unhealthy amusement to
interfere with their domestic affairs. Felix Bréton, who possessed the
rare gift of capturing and transferring to canvas the atmosphere of
the East with the opulent colorings and vivid contrasts which
constitute its charm, had nevertheless but little practical experience
of the manners and customs of the golden Orient. He had leased a large
studio situated on the roof of a fine old Cairene palace hidden away
behind the Street of the Booksellers and almost in the shadow of the
Mosque of el-Azhar. His romantic spirit had prompted him after a time
to give up his rooms at the Continental and to take up his abode in
the apartment adjoining the studio; that is to say, completely to cut
himself off from European life and to become an inhabitant of the
Oriental city. With his imperfect knowledge of the practical side of
native life in the East, I did not envy him; but I was fully alive to
his danger, isolated as he was from the European community, indeed
from modernity; for out of the boulevards of modern Cairo into the
streets of the _Arabian Nights_ is but a step, yet a step that bridges
the gulf of centuries.

As I entered his studio on the following morning, I discovered him at
work upon the extraordinary picture "Danse Funébre." Shejeret ed-Durr
was posing in the dress of an ancient priestess of Isis. Bréton
briefly greeted me, waving his hand towards a cushioned _dîwan_ before
which stood a little coffee-table bearing decanters, siphons,
cigarettes, and other companionable paraphernalia. Making myself
comfortable, I studied the picture and the model.

"Danse Funébre" was an extraordinary conception, representing an
elaborately furnished modern room, apparently that of an antiquary
or Egyptologist; for a multitude of queer relics decorated the walls,
cabinets, and the large table at which a man was seated. Boldly
represented immediately to the left of his chair stood a mummy in an
ornate sarcophagus, and forth from the swathed figure into the light
cast downwards from an antique lamp, floated a beautiful spirit
shape--that of an Egyptian priestess. Upon her face was an expression
of intense anger, as, her fingers crooked in sinister fashion, she
bent over the man at the table.

The mummy and sarcophagus depicted on the canvas stood before me
against the wall of the studio, the lid resting beside the case. It
was moulded, as is sometimes seen, to represent the face and figure of
the occupant and was as fine an example of the kind as I had met with.
The mummy was that of a priestess and dancer of the Great Temple at
Philæ, and it had been lent by the museum authorities for the purpose
of Bréton's picture.

His enthusiasm at first seeing Shejeret ed-Durr was explainable by the
really uncanny resemblance which the girl bore to the modeled figure.
Studying her, from my seat on the _dîwan_, as she posed in that gauzy
raiment depicted upon the lid of the sarcophagus, it seemed indeed
that the ancient priestess was reborn in the form of Shejeret ed-Durr
the _ghazîyeh_. Bréton had evidently tabooed make-up, with the
exception of the characteristic black bordering to the eyes (which
appeared in the presentment of the servant of Isis); and seen now in
its natural coloring the face of the dancing-girl had undoubted
beauty.

Presently, whilst the model rested, I informed Bréton of my
conversation with Abû Tabâh; but, as I had anticipated, he was
sceptical to the point of derision.

"My dear Kernaby," he said, "is it likely that I am going to interrupt
my work now that I have found such an inspiring model, because some
ridiculous _darwîsh_ disapproves?"

"It is highly unlikely," I admitted; "but do not make the mistake of
treating the matter lightly. You are right off the map here, and Cairo
is not Paris."

"It is a great deal safer!" he cried in his boisterous fashion, "and
infinitely more interesting."

But my mind was far from easy; for in the dark eyes of the model, when
their glance rested upon Felix Bréton, there was that to have aroused
poisonous sentiments in the bosom of the Man of the Glare.


III

During the course of the following month I saw Felix Bréton two or
three times, and he was enthusiastic about the progress of his picture
and the beauty of his model. The first hint that I received of the
strange idea which was to lead to stranger happenings came one
afternoon when he had called upon me at Shepheard's.

"Do you believe in reincarnation, Kernaby?" he asked suddenly.

I stared at him in surprise.

"Regardless of my personal views on the matter," I replied, "in what
way does the subject interest you?"

Momentarily he hesitated; then--

"The resemblance between Yâsmîna" (this was the real name of Shejeret
ed-Durr) "and the priestess of Isis," he said, "appears to me too
marked to be explainable by mere coincidence. If the mummy were my
personal property I should unwrap it----"

"Do you seriously desire me to believe that you regard Yâsmîna as a
reincarnation of the elder lady?"

"That or a lineal descendant," he answered. "The tribe of the
_Ghawâzi_ is of unknown antiquity and may very well be descended from
those temple dancers of the days of the Pharaohs. If you have studied
the ancient wall paintings, you cannot have failed to observe that the
dancing girls represented have entirely different forms from those of
any other women depicted and from those of the ordinary Egyptian women
of to-day."

His enthusiasm was tremendous; he was one of those uncomfortable
fanatics who will ride a theory to the death.

"I cannot say that I have noticed it," I replied. "Your knowledge
of the female form divine is doubtless more extensive than mine."

"My dear Kernaby," he cried excitedly, "to the trained eye the
difference is extraordinary. Until I saw Yâsmîna I had believed the
peculiar form to which I refer to be extinct like the blue enamel and
the sacred lotus. If it is not reincarnation it is heredity."

I could not help thinking that it more closely resembled insanity than
either; but since Bréton had made no reference to the wearer of the
black turban, I experienced less anxiety respecting his physical than
his mental welfare.

Three days later there was a dramatic development. Drifting idly into
Bréton's studio one morning I found him pacing the place in despair
and glaring at his unfinished canvas like a man distraught.

"Where is Shejeret ed-Durr?" I inquired.

"Gone!" he replied. "She disappeared yesterday and I can find no trace
of her."

"Surely the excellent Suleyman, proprietor of the dancing
establishment, can assist you?"

"I tell you," cried Bréton savagely, "that she has disappeared. No one
knows what has become of her."

I looked at him in dismay. He presented a mournful spectacle. He was
unshaven and his dark hair was wildly disordered. His despair was more
acute than I should have supposed possible in the circumstances; and
I concluded that his interest in Yâsmîna was deeper than I had assumed
or that I was incapable of comprehending the artistic temperament. I
suppose the Gallic blood in him had something to do with it, but I was
unspeakably distressed to observe that the man was on the verge of
tears.

Consolation was impossible, and I left him pacing his empty studio
distractedly. That night at an unearthly hour, long after I had
retired to my own apartments, he came to Shepheard's. Being shown into
my room, and the servant having departed--

"Yâsmîna is dead!" he burst out, standing there, a disheveled figure,
just within the doorway.

"What!" I exclaimed, standing up from the table at which I had been
writing and confronting him. "Dead? Do you mean----"

"He has murdered her!" said Bréton, in a dull monotonous voice--"that
fiend of whom you warned me."

I was appalled; for I had been utterly unprepared for such a tragedy.

"Who discovered her?"

"No one discovered her; she will never be discovered! He has buried
her body in some secret spot in the desert."

My amazement grew with every word that he uttered, and presently--

"Then how in Heaven's name did you learn of her murder?" I asked.

Felix Bréton, who had begun to pace up and down the room, a truly
pitiable figure, paused and looked at me wildly.

"You will think that I am mad, Kernaby," he said; "but I must tell
you--I must tell someone. I could see that you were incredulous when
I spoke to you of reincarnation, but I was right, Kernaby, I was right!
Either that or my reason is deserting me."

My opinion inclined distinctly in the direction of the latter theory,
but I remained silent, watching Bréton's haggard face.

"To-night," he continued, "as I sat looking at my unfinished picture
and trying to imagine what could have become of Yâsmîna, the
mummy--the mummy of the priestess--_spoke to me_!"

I slowly sank back into my chair. I was now assured that Felix Bréton
had formed a sudden and intense infatuation for Yâsmîna and that her
mysterious disappearance had deranged his sensitive mind. Words failed
me; I could think of nothing to say; and bending towards me his
haggard face--

"It whispered to me," he said, "in _her_ voice--in my own language,
French, as I have taught it to her; just a few imperfect words, but
sufficient to convey to me the story of the tragedy. Kernaby, what
does it mean? Is it possible that her spirit, released from the body
of Yâsmîna, has returned to that which I firmly believe it formerly
inhabited?..."

I had had the misfortune to be a party to some distressing scenes, but
few had affected me so unpleasantly as this. That poor Felix Bréton
was raving I could not doubt, but having persuaded him to spend the
night at Shepheard's and having seen him safely to bed, I returned to
my own room to endeavor to work out the problem of what steps I should
take regarding him on the morrow.

In the morning, however, he seemed more composed, having shaved and
generally rendered himself more presentable; but the wild look still
lingered in his eyes and I could see that the strange obsession had
secured a firm hold upon him. He discussed the matter quite calmly
during breakfast, and invited me to visit the scene of this
supernatural happening. I assented, and hailing _arabîyeh_ we drove
together to the studio.

There was nothing abnormal in the appearance of the place, but I
examined the mummy and the mummy case with a new curiosity; for if
Felix Bréton was not mad (and this was a point upon which I recognized
my incompetence to decide) the phantom voice was clearly the product
of some trick. However, I was unable to discover anything to account
for it. The sarcophagus stood against the outer wall of the studio and
near to a large lattice window before which was draped a heavy
tapestry curtain for the purpose of excluding undesirable light upon
that side of the model's throne. There was no balcony outside the
window, which was fully, thirty feet from the street below; therefore
unless someone had been hiding in the window recess beside the
sarcophagus, trickery appeared to be out of the question. Turning to
Bréton, who was watching me haggardly--

"You searched the recess last night?" I said.

"I did--immediately. There was no one there. There was no one anywhere
in the studio; and when I looked out of the open window, the street
below was deserted from end to end."

Naturally, I took it for granted that he would avoid the place, at any
rate by night; and I said as much, as we passed along the Mûski
together. I can never forget the wildness in his eyes as he turned to
me.

"I _must_ go back, Kernaby," he said. "It seems like desertion, base
and cowardly."


IV

Bréton did not join me at dinner that evening as we had arranged that
he should do, and towards the hour of ten o'clock, growing more and
more uneasy on his behalf, I set out for the studio, half hoping that
I should meet him. I saw nothing of him, however, as I crossed the
Ezbekîyeh Gardens and the Atabet el-Khadrâ into the Mûski. From thence
onward to the Rondpoint the dark and narrow streets were almost
deserted, and from the corner of the Shâria el-Khordâgîya to the
Street of the Bookbinders I met with no living thing save a lean and
furtive cat.

My footsteps echoed hollowly from wall to wall of the overhanging
buildings, as I approached the door giving access to the courtyard
from which a stair communicated with the studio above. The moonlight,
slanting down into the ancient place, left more than half of it in
densest shadow, but just touched the railing of the balcony and the
lower part of the _mushrabîyeh_ screen masking what once had been the
_harêm_ apartments from the view of one entering the courtyard. Far
above me, through an open lattice, a dim light shone out, though
vaguely. This part of the house was bathed in the radiance of the
moon, which dimmed that of the studio lamp; for the open window was
the window of Bréton's studio.

The door at the foot of the stairs was partly open, and I ascended
slowly, since the place was quite dark and I was forced to feel my way
around the eccentric turnings introduced by an Arab architect to whom
simplicity had evidently been an abomination.

A modern door had been fitted to the studio; and although this door
was also unfastened, I rapped loudly, but, receiving no answer,
entered the studio. It was empty. The lamp was lighted, as I had
observed from below, and a faint aroma of Turkish tobacco smoke hung
in the air. Clearly, Bréton had left but a few moments earlier; and
I judged it probable that he would be returning very shortly, for had
he set out for Shepheard's he would not have left his door unlocked,
and in any event I should have met him on the way. Therefore, having
glanced into the inner room, which, latterly, Bréton had been using as
a bedroom, I sat down on the _dîwan_ and prepared to await his return.

The lamp whose light I had seen shining through the window was that
which hung before the model's throne, and the curtain which usually
draped the window recess had been partially pulled aside, so that from
where I sat I could see part of the centre lattice, which was open.
My mind at this time was entirely occupied with uneasy speculations
regarding Bréton, and although I had glanced more than once at the
large unfinished picture on the easel, from which the face of Shejeret
ed-Durr peered out across the shoulder of the seated man, and several
times had looked at the mummy set upright in its painted sarcophagus,
no sense of the uncanny had touched me or in any way prepared me for
the amazing manifestation which I was about to witness.

How long I had sat there I cannot say exactly; possibly for ten
minutes or a quarter of an hour: when, suddenly, an eerie whisper
crept through the stillness of the big room!

Since I had more than once been temporarily tricked into belief in the
supernatural, by means of certain ingenious devices, I did not readily
fall a victim to the mysterious nature of the present occurrence. Yet
I must confess that my heart gave a great leap and I was forced to
exert all my will to control my nerves. I sat quite still, listening
intently for a repetition of that evil whisper. Then, in the
stillness, it came again.

"Felix," it breathed, "because of you I lie dead in a grave in the
desert.... I died for you, Felix, and now I am so lonely...."

The whispering voice offered no clue to the age or the sex of the
speaker; for a true whisper is toneless. But the words, as Bréton had
declared, were uttered in broken French and spoken with a curious
accent.

It ceased, that ghostly whispering; and I realized that my nerves
could stand no more of it; for that it came or seemed to come from the
mummy of the priestess was a fact as undeniable as it was horrible.

Resorting to action, I sprang up and leaped across the room, grasping
first at the curtain draped in the window on the right of the
sarcophagus. I jerked it fully aside. The recess was empty. All three
lattices were open, on the right, left, and in the centre of the
window; but, craning out from the latter, I saw the street below to
be vacant from end to end.

Stepping back into the room, and metaphorically clutching my courage
with both hands, I approached the sarcophagus, peered behind it, all
around it, and, finally, into the swathed face of the mummy itself.
Nothing rewarded my search. But the studio of Felix Bréton seemed to
have become icily cold; at any rate I found myself to be shivering;
and walking deliberately, although it cost me a monstrous effort to do
so, I descended the dark winding stairway into the courtyard, and, on
regaining the street, discovered to my intense annoyance that my brow
was wet with cold perspiration.

I had taken no more than ten paces in the direction of the Sûk
es-Sûdan when I heard the sound of approaching footsteps, and for some
reason (I can only suppose as a result of my highly strung condition)
I stepped into the shelter of a narrow gateway, where I could see
without being seen, and there awaited the appearance of the one who
approached.

It was Felix Bréton, his face showing ghastly in the moonlight as he
turned the corner. I could not be certain if a mere echo had deceived
me, but I thought I could detect faintly the softer footfalls of
someone who was following him. From my cover I had an uninterrupted
view of the entrance to the house which I had just left; and without
showing myself I watched Bréton approach the door. At its threshold
he seemed to hesitate; and in that brief hesitancy were illustrated
the conflicting emotions driving the man. I recalled the words he had
spoken to me that morning. "I must go back, Kernaby; it seems like
desertion, base and cowardly." He opened the door and disappeared.

As he did so, a second figure crossed from the shadows on the opposite
side of the street--that is, the side upon which I was concealed; and
in turn advanced towards the door. As he passed my hiding-place I
acted. Without an instant's hesitation I hurled myself upon him.

How he avoided that furious attack--if he did avoid it--or whether in
the darkness I miscalculated my spring, I do not know to this day: I
only know that I missed my objective, stumbled, recovered myself ...
and turned with clenched fists to find _Abû Tabâh_ confronting me!

"Kernaby Pasha!" he cried.

"Abû Tabâh!" said I dazedly.

"I perceive that I am not alone in my anxiety for the welfare of M.
Felix Bréton."

"But why were you following him? I narrowly missed assaulting you."

"Very narrowly," he agreed in his gentle manner; "but you ask me why
I was following M. Bréton. I was following him because I have seen so
many of those who have crossed the path of the Black _Darwîshes_ meet
with violent and inexplicable deaths."

"Murder?" I whispered.

"Not murder--suicide. Therefore, observing, as I had anticipated,
a strangeness in your friend's behavior, I have watched him."

"The strangeness of his behavior is easily accounted for," I said.
And excitedly, for the horror of the episode in the studio was still
strongly upon me, I told him of the whispering mummy.

"These are very dreadful things of which you speak, Kernaby Pasha," he
admitted, "but I warned you that it was ill to incur the enmity of the
Black _Darwîshes_. That there is a scheme afoot to compass the
self-destruction or insanity of your friend is now evident to me; and
he has brought this calamity upon himself; for the words which he
believed to be spoken by the spirit of the girl Yâsmîna would not have
affected him so unpleasantly if his attitude towards her had been
marked by proper restraint and the affair confined within suitable
limitations."

"Quite so. But although the Black _Darwîshes_ may be both malignant
and clever, that uncanny whispering is beyond the control of natural
forces."

"Such is not my opinion," replied Abû Tabâh. "A spirit does not
mistake one person for another; and the whispering voice addressed
itself to 'Felix' when Felix was not present. I believe, Kernaby
Pasha, that you are the possessor of a pair of excellent
opera-glasses? May I suggest that you return to Shepheard's and
procure them."


V

The platform of the minaret seemed very cold to the touch of my
stockinged feet; for I had left my shoes at the entrance to the mosque
below in accordance with custom; and now, from the wooden balcony, I
overlooked the neighboring roofs of Cairo, and Abû Tabâh, beside me,
pointed to where a vague patch of light broke the darkness beneath us
to the left.

"The window of M. Felix Bréton's studio," he said.

Raising the glasses to his eyes, he gazed in that direction, whilst
I also peered thither and succeeded in making out the well of the
courtyard and the roofs of the buildings to right and left of it.
It was not evident to me for what Abû Tabâh was looking, and when
presently he lowered the glasses and turned to me I expressed my
doubts in words.

"It is surely evident," I said, speaking, as I now almost invariably
did to the _imám_, in English, of which he had a perfect mastery,
"that we have little chance of discovering anything from here, since
nothing was visible from the studio window. Furthermore, who save
Yâsmîna could have spoken in the manner which I have related and in
broken French?"

"An eavesdropper," he replied, "might have profited by the lessons
which Yâsmîna received from M. Bréton; and all vocal characteristics
are lost in a whisper. In the second place, Yâsmîna is not dead."

"What!" I cried.

Although, when Bréton had informed me of her death, I myself had
doubted him, for some reason the ghostly whisper had convinced me as
it had convinced him.

"She has been kept a prisoner during the past week in a house
belonging to one of the Black _Darwîshes_," continued Abû Tabâh; "but
my agents succeeded in tracing her this morning. By my orders,
however, she has not been allowed to return to her home."

"And what was the object of those orders?"

"That I might learn for what purpose she had been made to disappear,"
replied Abû Tabâh; "and I have learned it to-night."

"Then you think that the whispering mummy----"

He suddenly clutched my arm.

"Quick! raise your glasses!" he said softly. "On the roof of the house
to the left of the light. There is the whispering mummy!"

Strung up to a high pitch of excitement, I gazed through the glasses
in the direction indicated by my companion. Without difficulty I
discerned him--a man wearing a black turban--who crept like some
ungainly cat along the flat roof, carrying in his hand what looked
like one of those sugar canes which pass for a delicacy among the
natives, but which to European eyes appear more suitable for
curtain-poles than sweetmeats. Springing perilously across a yawning
gulf, the wearer of the black turban gained the roof of the studio,
crept along for some little distance further, and then, lying prone,
began slowly to lower the bamboo rod in the direction of the lighted
window.

I found that unconsciously I had suspended my respiration, and now,
breathlessly, as the truth came home to me--

"It is a speaking-tube!" I cried, "I cannot see the end of it, but no
doubt it is curved so as to protrude through the side of the lattice
window. Do you look, Abû Tabâh: _I_ propose to act."

Thrusting the glasses into the _imám's_ hand, I took my Colt repeater
from my pocket, and, having peered for some seconds steadily in the
direction of the dimly visible _Darwîsh_, I opened fire! I had fired
five shots in the heat of my anger at that sinister crouching figure,
ere Abû Tabâh seized my wrist.

"Stop!" he cried; "do you forget where you stand?"

Truly I had forgotten in my indignation, or I should not have outraged
his feelings by firing from the minaret of a mosque. But sufficient of
my wrath remained to occasion me a thrill of satisfaction, when,
peering through the dusk, I saw the _Darwîsh_ throw up his arms and
disappear from view.

       *       *       *       *       *

"There is blood in the courtyard," said Abû Tabâh; "but Ahmad es-Kebîr
has fled. Therefore he still lives, and his anger will be not the less
but the greater. Depart from Cairo, M. Bréton: it is my counsel to
you."

"But," cried Felix Bréton, glaring wildly at the big canvas on the
easel, "I must finish my picture. As Yâsmîna is alive, she must
return, and I must finish my picture!"

"Yâsmîna cannot return," replied Abû Tabâh, fixing his weird eyes upon
the speaker. "I have caused her to be banished from Cairo." He raised
his hand, checking Bréton's hot words ere they were uttered.
"Recriminations are unavailing. Her presence disturbs the peace of the
city, and the peace of the city it is my duty to maintain."




PART II

OTHER TALES




I

LORD OF THE JACKALS


In those days, of course (said the French agent, looking out across
the sea of Yûssuf Effendis which billowed up against the balcony to
where, in the moonlight, the minarets of Cairo pointed the way to
God), I did not occupy the position which I occupy to-day. No, I was
younger, and more ambitious; I thought to carve in the annals of Egypt
a name for myself such as that of De Lesseps.

I had a scheme--and there were those who believed in it--for extending
the borders of Egypt. Ah! my friends, Egypt after all is but a double
belt of mud following the Nile, and terminated east and west by the
desert. The desert! It was the dream of my life to exterminate that
desert, that hungry gray desert; it was my plan--a foolish plan as I
know now--to link the fertile Fáyûm to the Oases! How was this to be
done? Ah!

Why should I dig up those buried skeletons? It was not done; it never
could be done; therefore, let me not bore you with how I had proposed
to do it. Suffice it that my ambitions took me far off the beaten
tracks, far, even, from the caravan roads--far into the gray heart of
the desert.

But I was ambitious, and only nineteen--or scarcely twenty. At
nineteen, a man who comes from St. Rémy fears no obstacle which Fate
can place in his way, and looks upon the world as a grape-fruit to be
sweetened with endeavor and sucked empty.

It was in those days, then, that I learned as your Rudyard Kipling has
also learned that "East is East"; it was in those days that I came
face to face with that "mystery of Egypt" about which so much is
written, has always been written, and always will be written, but
concerning which so few people, so very few people, know anything
whatever.

Yes, I, René de Flassans, saw with my own eyes a thing that I knew to
be magic, a thing whereat my reason rebelled--a thing which my poor
European intelligence could not grapple, could not begin to explain.

It was this which you asked me to tell you, was it not? I will do so
with pleasure, because I know that I speak to men of honor, and
because it is good for me, now that I cannot count the gray hairs in
my beard, to confess how poor a thing I was when I could count every
hair upon my chin--and how grand a thing I thought myself.

One evening, at the end of a dreadful day in the saddle--beneath a sky
which seemed to reflect all the fires of hell, a day passed upon sands
simply smoking in that merciless sun--I and my native companions came
to an encampment of Arabs.

They were Bedouins[C]--the tribe does not matter at the moment--and,
as you may know, the Bedouin is the most hospitable creature whom God
has yet created. The tent of the Sheikh is open to any traveller who
cares to rest his weary limbs therein. Freely he may partake of all
that the tribe has to offer, food and drink and entertainment; and to
seek to press payment upon the host would be to insult a gentleman.

  [C] This incorrect but familiar spelling is retained throughout.

That is desert hospitality. A spear that stands thrust upright in the
sand before the tent door signifies that whosoever would raise his
hand against the guest has first to reckon with the Sheikh. Equally
it would be an insult to erect one's own tent in the neighborhood of
a Bedouin encampment.

Well, my friends, I knew this well, for I was no stranger to the
nomadic life, and accordingly, without fear of the fierce-eyed throng
who came forth to meet us, I made my respects to the Sheikh Saïd
Mohammed, and was reckoned by him as a friend and a brother. His tent
was placed at my disposal and provisions were made for the suitable
entertainment of those who were with me.

You know how dusk falls in Egypt? At one moment the sky is a brilliant
canvas, glorious with every color known to art, at the next the
curtain--the wonderful veil of deepest violet--has fallen; the stars
break through it like diamonds through the finest gauze; it is night,
velvet, violet night. You see it here in this noisy modern Cairo. In
the lonely desert it is ten thousand times grander, ten thousand times
more impressive; it speaks to the soul with the voice of the silence.
Ah, those desert nights!

So was the night of which I speak; and having partaken of the fare
which the Sheikh caused to be set before me--and Bedouin fare is not
for the squeamish stomach--I sipped that delicious coffee which,
though an acquired taste, is the true nectar, and looked out beyond
the four or five palm trees of this little oasis to where the gray
carpet of the desert grew black as ebony and met the violet sweep of
the sky.

Perhaps I was the first to see him; I cannot say; but certainly he was
not perceived by the Bedouins, although one stood on guard at the
entrance to the camp.

How can I describe him? At the time, as he approached in the moonlight
with a shambling, stooping gait, I felt that I had never seen his like
before. Now I know the reason of my wonder, and the reason of my
doubt. I know what it was about him which inspired a kind of horror
and a revulsion--a dread.

Elfin locks he had, gray and matted, falling about his angular face,
shading his strange, yellow eyes. His was dressed in rags, in tatters;
he was furtive, and he staggered as one who is very weak, slowly
approaching out of the vastness.

Then it appeared as though every dog in the camp knew of his coming.
Out from the shadows of the tents they poured, those yapping mongrels.
Never have I seen such a thing. In the midst of the yellowish,
snarling things, at the very entrance to the camp, the wretched old
man fell, uttering a low cry.

But now, snatching up a heavy club which lay close to my hand, I
rushed out of the tent. Others were thronging out too, but, first of
them all, I burst in among the dogs, striking, kicking, and shouting.
I stooped and raised the head of the stranger.

Mutely he thanked me, with half-closed eyes. A choking sound issued
from his throat, and he clutched with his hands and pointed to his
mouth.

An earthenware jar, containing cool water, stood beside a tent but
a few yards away. Hurling my club at the most furious of the dogs,
which, with bared fangs, still threatened to attack the recumbent man,
I ran and seized the _dorak_, regained his side, and poured water
between his parched lips.

The throng about me was strangely silent, until, as the poor old man
staggered again to his feet, supported by my arm, a chorus arose about
me--one long, vowelled word, wholly unfamiliar, although my Arabic was
good. But I noted that all kept a respectful distance from myself and
the man whom I had succored.

Then, pressing his way through the throng came the Sheikh Saïd
Mohammed. Saluting the ragged stranger with a sort of grim respect,
he asked him if he desired entertainment for the night.

The other shook his head, mumbling, pointed to the water jar, and by
dint of gnashing his yellow and pointed teeth, intimated that he
required food.

Food was brought to him hurriedly. He tied it up in a dirty cloth,
grasped the water jar, and, with never a glance at the Arabs, turned
to me. With his hand he touched his brow, his lips, and his breast in
salute; then, although tottering with weakness, he made off again with
that queer, loping gait.

The camp dogs began to howl, and a strange silence fell upon the Arabs
about me. All stood watching the departing figure until it was lost in
a dip of the desert, when the watchers began to return again to their
tents.

Saïd Mohammed took my hand, and in a few direct and impressive words
thanked me for having spared him and his tribe from a grave dishonor.
Need I say that I was flattered? Had you met him, my friends, that
fine Bedouin gentleman, polished as any noble of old France, fearless
as a lion, yet gentle as a woman, you would know that I rejoiced in
being able to serve him even so slightly.

Two of the dogs, unperceived by us, had followed the weird old man
from the camp; for suddenly in the distance I heard their savage
growls. Then, these growls were drowned in such a chorus of
howling--the howling of jackals--as I had never before heard in all my
desert wanderings. The howling suddenly subsided ... but the dogs did
not return.

I glanced around, meaning to address the Sheikh, but the Sheikh was
gone.

Filled with wonder, then, respecting this singular incident, I entered
the tent--it was at the farther end of the camp--which had been placed
at my disposal, and lay down, rather to reflect than to sleep. With my
mind confused in thoughts of yellow-eyed wanderers, of dogs, and of
jackals, sleep came.

How long I slept I cannot say; but I was awakened as the cool fingers
of dawn were touching the crests of the sand billows. A gray and
dismal light filled the tent, and something was scratching at the
flap.

I sat up immediately, quite wide awake, and taking my revolver, ran
to the entrance and looked out.

A slinking shape melted into the shadows of the tent adjoining mine,
and I concluded that a camp dog had aroused me. Then, in the early
morning silence, I heard a faint call, and peering through the gloom
to the east saw, in black silhouette, a solitary figure standing near
the extremity of the camp.

In those days, my friends, I was a brave fellow--we are all brave at
nineteen--and throwing a cloak over my shoulders I strode intrepidly
towards this figure. I was within ten paces when a hand was raised to
beckon me.

It was the mysterious stranger! Again he beckoned to me, and I
approached yet nearer, asking him if it was he who had aroused me.

He nodded, and by means of a grotesque kind of pantomime ultimately
made me understand that he had caused me to be aroused in order to
communicate something to me. He turned, and indicated that we were to
walk away from the camp. I accompanied him without hesitation.

Although the camp was never left unguarded, no one had challenged us;
and, a hundred yards beyond the outermost tent, this strange old man
stopped and turned to me.

First, he pointed back to the camp, then to myself, then out along
the caravan road towards the Nile.

"Do you mean," I asked him--for I perceived that he was dumb or vowed
to silence--"that I am to leave the camp?"

He nodded rapidly, his strange yellow eyes gleaming.

"Immediately?" I demanded.

Again he nodded.

"Why?"

Pantomimically he made me understand that death threatened me if I
remained--that I must leave the Bedouins before sunrise.

I cannot convey to you any idea of the mad earnestness of the man.
But, alas! youth regards the counsels of age with nothing but
contempt; moreover, I thought this man mad, and I was unable to choke
down a sort of loathing which he inspired in me.

I shook my head then, but not unkindly; and, waving my hand, prepared
to leave him. At that, with a sorrow in his strange eyes which did not
fail to impress me, he saluted me with gravity, turned, and passed out
of sight.

Although I did not know it at the time, I had chosen of two paths the
one that led through fire.

I slept little after this interview--if it was a real interview and
not a dream--and feeling tired and unrefreshed, I saw the sun rise
purple and angry over the distant hills.

You know what _khamsîn_ is like, my friends? But you cannot know what
_simoom_ is like--_simoom_ in the heart of the desert! It came that
morning--a wall of sand so high as to shut out the sunlight, so dense
as to turn the day into night, so suffocating that I thought I should
never live through it!

It was apparent to me that the Bedouins were prepared for the storm.
The horses, the camels and the asses were tethered in an enclosure
specially strengthened to exclude the choking dust, and with their
cloaks about their heads the men prepared for the oncoming of this
terror of the desert.

My God! it was a demon which sought to blind me, to suffocate me,
and which clutched at my throat with strangling fingers of sand! This,
I told myself, was the danger which I might have avoided by quitting
the camp before sunrise.

Indeed, it was apparent to me that if I had taken the advice so
strangely offered, I might now have been safe in the village of the
Great Oasis for which I was bound. But I have since seen that the
_simoom_ was a minor danger, and not the real one to which this weird
being had referred.

The storm passed, and every man in the encampment praised the merciful
God who had spared us all. It was in the disturbance attendant upon
putting the camp in order once more that I saw her.

She came out from the tent of Saïd Mohammed, to shake the sand from
a carpet; the newly come sunlight twinkled upon the bracelets which
clasped her smooth brown arms as she shook the gaily colored mat at
the tent door. The sunlight shone upon her braided hair, upon her
slight robe, upon her silver anklets, and upon her tiny feet.
Transfixed I stood watching--indeed, my friends, almost holding my
breath. Then the sunlight shone upon her eyes, two pools of mysterious
darkness into which I found myself suddenly looking.

The face of this lovely Arab maiden flushed, and drawing the corner of
her robe across those bewitching eyes, she turned and ran back into
the tent.

One glance--just one glance, my friends! But never had Ulysses' bow
propelled an arrow more sure, more deadly. I was nineteen, remember,
and of Provence. What do you foresee! You who have been through the
world, you who once were nineteen.

I feigned a sickness, a sickness brought about by the sandstorm, and
taking base advantage of that desert hospitality which is unbounded,
which knows no suspicion, and takes no count of cost, I remained in
the tent which had been vacated for me.

In this voluntary confinement I learned little of the doings of the
camp. All day I lay dreaming of two dark eyes, and at night when the
jackals howled I thought of the wanderer who had counseled me to
leave. One day, I lay so; a second; a third again; and the women of
Saïd Mohammed's household tended me, closely veiled of course. But in
vain I waited for that attendant whose absence was rendering my
feigned fever a real one--whose eyes burned like torches in my dreams
and for the coming of whose little bare feet across the sand to my
tent door I listened hour by hour, day by day, in vain--always in
vain.

But at nineteen there is no such thing as despair, and hope has
strength to defy death itself. It was in the violet dusk of the fourth
day, as I lay there with a sort of shame of my deception struggling
for birth in my heart, that she came.

She came through the tent door bearing a bowl of soup, and the rays of
the setting sun outlined her fairy shape through the gossamer robe as
she entered.

At that my poor weak little conscience troubled me no more. How my
heart leaped, leaped so that it threatened to choke me, who had come
safe through a great sandstorm.

There is fire in the Southern blood at nineteen, my friends, which
leaps into flame beneath the glances of bright eyes.

With her face modestly veiled, the Bedouin maid knelt beside me,
placing the wooden bowl upon the ground. My eager gaze pierced the
_yashmak_, but her black lashes were laid upon her cheek, her glorious
eyes averted. My heart--or was it my vanity?--told me that she
regarded me at least with interest, that she was not at ease in my
company; and as, having spoken no word, having ventured no glance,
she rose again to depart, I was emboldened to touch her hand.

Like a startled gazelle she gave me one rapid glance, and was gone!

She was gone--and my very soul gone with her! For hours I lay, not so
much as thinking of the food beside me--dreaming of her eyes. What
were my plans? Faith! Does one have plans at nineteen where two bright
eyes are concerned?

Alas, my friends, I dare not tell you of my hopes, yet upon those
hopes I lived. Oh, it is glorious to be nineteen and of Provence; it
is glorious when all the world is young, when the fruit is ripe upon
the trees and the plucking seems no sin. Yet, as we look back, we
perceive that at nineteen we were scoundrels.

The Bedouin girl is a woman when a European woman is but a child, and
Sakîna, whose eyes could search a man's soul, was but twelve years of
age--twelve! Can you picture that child of twelve squeezing a lover's
heart between her tiny hands, entwining his imagination in the coils
of her hair?

You, my friend, may perhaps be able to conceive this thing, for you
know the East, and the women of the East. At ten or eleven years of
age many of them are adorable; at twenty-one most of them are _passé_;
at twenty-six all of them--with rare exceptions--are shrieking hags.

But to you, my other friends, who are strangers to our Oriental ways,
who know not that the peach only attains to perfect ripeness for one
short hour, it may be strange, it may be horrifying, that I loved,
with all the ardor which was mine, this little Arab maiden, who, had
she been born in France, would not yet have escaped from the nursery.
But I digress.

The Arabs were encamped, of course, in the neighborhood of a spring.
It lay in a slight depression amid the tiny palm-grove. Here, at
sunset, came the women with their pitchers on their heads, graceful
of carriage, veiled, mysterious.

Many peaches have ripened and have rotted since those days of which
I speak, but now--even now--I am still enslaved by the mystery of
Egypt's veiled women. Untidy, bedraggled, dirty, she may be, but the
real Egyptian woman when she bears her pitcher upon her head and
glides, stately, sinuously, through the dusk to the well, is a figure
to enchain the imagination.

Very soon, then, the barrier of reserve which, like the screen of the
_harêm_, stands between Eastern women and love, was broken. My trivial
scruples I had cast to the winds, and feigning weakness, I would sally
forth to take the air in the cool of the evening; this two days later.

My steps, be assured, led me to the spring; and you who are men of the
world will know that Sakîna, braving the reproaches of the Sheikh's
household, neglectful of her duties, was last of all the women who
came to the well for water.

I taught her to say my name--René! How sweet it sounded from her lips,
as she strove in vain to roll the 'R' in our Provençal fashion. Some
_ginnee_ most certainly presided over this enchanted fountain, for
despite the nearness of the camp our rendezvous was never discovered,
our meetings were never detected.

With her pitcher upon the ground beside her, she would sit with those
wistful, wonderful eyes upraised to mine, and sway before the ardor of
my impassioned words as a young and tender reed sways in the Nile
breeze. Her budding soul was a love lute upon which I played in
ecstasy; and when she raised her red lips to mine.... Ah! those nights
in the boundless desert! God is good to youth, and harsh to old age!

Next to Saïd Mohammed, her father, Sakîna's brother was the finest
horseman of the tribe, and his white mare their fleetest steed. I
had cast covetous eyes upon this glorious creature, my friends, and
secretly had made such overtures as were calculated to win her
confidence.

Within two weeks, then, my plans were complete--up to a point. Since
they were doomed to failure, like my great scheme, I shall not trouble
you with their details, but an hour before dawn on a certain night I
cut the camel-hair tethering of the white mare, and, undetected, led
the beautiful creature over the silent sands to a cup-like depression,
a thousand yards distant from the camp.

The Bedouin who was upon guard that night had with him a gourd of
_'erksoos_. This was customary, and I had chosen an occasion when the
duty of filling the sentinel's gourd had fallen upon Sakîna; to his
_'erksoos_ I had added four drops of dark brown fluid from my medicine
chest.

It was an hour before dawn, then, when I stood beside the white mare,
watching and listening; it was an hour before dawn when she for whom
my great scheme was forgotten, for whom I was about to risk the anger,
the just anger, of men amongst the most fierce in the known world,
came running fleetly over the hillocks down into the little valley,
and threw herself into my arms....

When dawn burst in gloomy splendor over the desert, we were still five
hours' ride from the spot where I had proposed temporarily to conceal
myself, with perhaps an hour's start of the Arabs. I knew the desert
ways well enough, but the ghostly and desolate place in which I now
found myself nevertheless filled me with foreboding.

A seam of black volcanic rock split the sands for a great distance,
forming a kind of natural wall of forbidding aspect. In places this
wall was pierced by tunnel-like openings; I think they may have been
prehistoric tombs. There was no scrap of verdure visible, north,
south, east or west; only desolation, sand, grayness, and this place,
ghostly and wan with that ancient sorrow, that odor of remote
mortality which is called "the dust of Egypt."

Seated before me in the saddle, Sakîna looked up into my face with a
never-changing confidence, having her little brown fingers interlocked
about my neck. But her strength was failing. A short rest was
imperative.

Thus far I had detected no evidence of pursuit and, descending from
the saddle, I placed my weary companion upon a rock over which I had
laid a rug, and poured out for her a draught of cool water.

Bread and dates were our breakfast fare; but bread and dates and water
are nectar and ambrosia when they are sweetened with kisses. Oh! the
glorious madness of youth! Sometimes, my friends, I am almost tempted
to believe that the man who has never been wicked has never been
happy!

Picture us, then, if you can, set amid that desolation, which for us
was a rose-garden, eating of that unpalatable food--which for us was
the food of the gods!

So we remained awhile, deliriously happy, though death might terminate
our joys ere we again saw the sun, when something ... _something_
spoke to me....

Understand me, I did not say that _someone_ spoke, I did not say that
anything _audible_ spoke. But I know that, unlocking those velvet arms
which clung to me, I stood up slowly--and, still slowly, turned and
looked back at the frowning black rocks.

Merciful God! My heart beats wildly now when I recall that moment.

Motionless as a statue, but in a crouching attitude, as if about to
leap down, he who had warned me so truly stood upon the highest point
of the rocks watching us!

How long did I remain thus?

I cannot pretend to say; but when I turned to Sakîna--she lay
trembling on the ground, with her face hidden in her hands.

Then, down over the piled-up rocks, this mysterious and ominous being
came leaping. Old man though he was, he descended with the agility of
a mountain goat--and sometimes, in the difficult places, _he went on
all fours_.

Crossing the intervening strip of sand, he stood before me. You have
seen the reproach in the eyes of a faithful dog whose master has
struck him unjustly? Such a reproach shone out from the yellow eyes of
this desert wanderer. I cannot account for it; I can say no more....

It was impossible for me to speak; I trembled violently; such a fear
and such a madness of sorrow possessed me that I would have welcomed
any death--to have freed me from that intolerable reproach.

He suddenly pointed towards the horizon where against the curtain of
the dawn black figures appeared.

I fell upon my knees beside Sakîna. I was a poor, pitiable thing; the
madness of my passion had left me, and already I was within the great
Shadow; I could not even weep; I knew that I had brought Sakîna out
into that desolate place--to die.

And now the man whose ways were unlike human ways began to babble
insanely, gesticulating and plucking at me. I cannot hope to make you
feel one little part of the emotion with which those instants were
laden. Sakîna clung to me trembling in a way I can never
forget--never, never forget. And the look in her eyes! even now I
cannot bear to think of it, I cannot bear----

Those almost colorless lizards which dart about in the desert places
with incredible swiftness were now coming forth from their nests; and
all the while the black figures, unheard as yet, were approaching
along the path of the sun.

My mad folly grew more apparent to me every moment. I realized that
this which so rapidly was overtaking me had been inevitable from the
first. The strange wild man stood watching me with that intolerable
glare, so that my trembling companion shrank from him in horror.

But evidently he was seeking to convey some idea to me. He
gesticulated constantly, pointing to the approaching Arabs and then
over his shoulder to the frowning rock behind. Since it was too late
for flight--for I knew that the white mare with a double burden could
never outpace our pursuers--it occurred to me at the moment when the
muffled beat of hoofs first became audible, that this hermit of the
rocks was endeavoring to induce me to seek some hiding-place with
which no doubt he was acquainted.

How I cursed the delay which had enabled the Arabs to come up with us!
I know, now, of course, that even had I not delayed, our ultimate
capture was certain. But at the moment, in my despair, I thought
otherwise.

And now I cursed the stupidity which had prevented me from following
this weird guide; I even thought wrathfully of the poor frightened
child, whose weakness had necessitated the delay and whose fears had
contributed considerably to this later misunderstanding.

The pursuing party, numbering four, and led by Saïd Mohammed, was no
more than five hundred yards away when I came to my senses. The hermit
now was tugging at my arm with frightful insistence; his eyes were
glaring insanely, and he chattered in an almost pitiable manner.

"Quick!" I cried, throwing my arm about Sakîna, "up to the rocks. This
man can hide us!"

"No, no!" she whispered, "I dare not----"

But I lifted her, and signing to the singular being to lead the way,
staggered forward despairingly.

The distance was greater than it appeared, the climb incredibly
difficult. My guide held out his hand to me to assist me to mount the
slippery rocks; but I had much ado to proceed and also to support
Sakîna.

Her terror of the man and of the place to which he was leading us
momentarily increased. Indeed, it seemed that she was becoming mad
with fear. When the man paused before an opening in the rocks not
more than fifteen or sixteen inches in height, and wildly waving
his arms in the air, his elfin locks flying about his shoulder, his
eyes glassy, intimated that we were to crawl in--Sakîna writhed free
of my grasp and bounded back some three or four paces down the slope.

"Not in there!" she cried, holding out her little hands to me
pitifully. "I dare not! He would devour us!"

At the foot of the slope, Saïd Mohammed, who had dismounted from his
horse, and who, far ahead of the others, was advancing towards us,
at that moment raised his gun and fired....

Can I go on?

It is more years ago than I care to count, but it is fresher in my
mind than the things of yesterday. A lonely old age is before me, my
friends--for I have been a solitary man since that shot was fired. For
me it changed the face of the world, for me it ended youth, revealing
me to myself for what I was.

Something more nearly resembling human speech than any sound he had
yet uttered burst from the lips of the wild man as the report of Saïd
Mohammed's shot whispered in echoes through the mysterious labyrinths
beneath us.

Fate had stood at the Sheikh's elbow as he pulled the trigger.

With a little soft cry--I hear it now, gentle, but having in it a
world of agony--Sakîna sank at my feet ... and her blood began to
trickle over the black rocks on which she lay.

       *       *       *       *       *

The man who professes to describe to you his emotions at such a
frightful moment is an impostor. The world grew black before my eyes;
every emotion of which my being was capable became paralysed.

I heard nothing, I saw nothing but the little huddled figure, that red
stream upon the black rock, and the agonized love in the blazing eyes
of Sakîna. Groaning, I threw myself down beside her, and as she sighed
out her life upon my breast, I knew--God help me--that what had been
but a youthful amour, was now a life's tragedy; that for me the light
of the world had gone out, that I should never again know the warmth
of the sun and the gladness of the morning....

The cave man, with a dog-like fidelity, sought now to drag me from my
dead love, to drag me into that gloomy lair which she had shrunk from
entering. His incoherent mutterings broke in upon my semi-coma; but I
shook him off, I shrieked curses at him....

Now the Bedouins were mounting the slope, not less than a hundred
yards below me. In the growing light I could see the face of Saïd
Mohammed....

The man beside me exerted all his strength to drag me back into the
gallery or cave--I know not what it was; but with my arms locked about
Sakîna I lay watching the pursuers coming closer and closer.

Then, those persistent efforts suddenly ceased, and dully I told
myself that this weird being, having done his best to save me, had
fled in order to save himself.

I was wrong.

You have asked me for a story of the magic of Egypt, and although,
as you see, it has cost me tears--oh! I am not ashamed of those tears,
my friends!--I have recounted this story to you. You say, where is the
magic? and I might reply: the magic was in the changing of my false
love to a true. But there was another magic as well, and it grew up
around me now at this moment when I lay inert, waiting for death.

From behind me, from above me, arose a cry--a cry. You may have heard
of the Bedouin song, the 'Mizmûne':

    "Ya men melek ana dêri waat sa jebb,
     Id el' ish hoos' a beb hatsa azât ta lebb."

You may have heard how when it is sung in a certain fashion, flowers
drop from their stalks? Also, you may have doubted this, never having
heard a magical cry.

_I_ do not doubt it, my friends! For I _have_ heard a magical
cry--this cry which arose from behind me! It started some chord in my
dulled consciousness which had never spoken before. I turned my
head--and there upon the highest point of the rocks stood the cave
man. He suddenly stretched forth his hands.

Again he uttered that uncanny, that indescribable cry. It was not
human. It was not animal. Yet it was nearer to the cry of an animal
than to any sound made by the human species. His eyes gleamed with an
awful light, his spare body had assumed a strange significance; he was
transfigured.

A third time he uttered the cry, and out from one of those openings in
the rock which I have mentioned, crept a jackal. You know how a jackal
avoids the day, how furtive, how nocturnal a creature it is? but there
in the golden glory which proclaimed the coming of the sun, black
silhouettes moved.

A great wonder possessed me, as the first jackal was followed by a
second, by a third, by a fourth, by a fifth. Did I say a fifth?...
By five hundred--by five thousand!

From every visible hole in the rocks, jackals poured forth in packs.
Wonder left me, fear left me; I forgot my sorrow, I became a numbed
intelligence amid a desert of jackals. Over a sea of moving furry
backs, I saw that upstanding crag and the weird crouching figure upon
it. Right and left, above and below, jackals moved ... and all turned
their heads towards the approaching Bedouins!

Again--again I heard that dreadful cry. The jackals, in a pack,
thousands strong, began to advance upon the Bedouins!...

Not east or west, north or south, could you hope to find a braver man
than was the Sheikh Saïd Mohammed; but--he fled!

I saw the four horsemen riding like furies into the morning sun. The
white mare, riderless, galloped with them--and the desert behind was
yellow with jackals! For the last time I heard the cry.

The jackals began to return!

Forgive me, dear friends, if I seem an emotional fool. But when I
recovered from the swoon which blotted out that unnatural spectacle,
the wizard--for now I knew him for nothing less--had dug a deep
trench--and had left me, alone.

Not a jackal was in sight; the sun blazed cruelly upon the desert.
With my own hands I laid my love to rest in the sands. No cross, no
crescent marks her resting-place; but I left my youth upon her grave,
as a last offering.

You may say that, since I had sinned so grievously, since I had
betrayed the noble confidence of Saïd Mohammed, my host, I escaped
lightly.

Ah! you do not know!

And what of the strange being whose gratitude I had done so little to
merit but yet which knew no bounds? It is of him that I will tell you.

Years later--how many it does not matter, but I was a man with no
illusions--my restless wanderings (I being still a desert
bird-of-passage) brought me one night to a certain well but rarely
visited. It lay in a depression, like another well that I am fated
often to see in my dreams, and, as one approached, the crowns of the
palm trees which grew there appeared above the mounds of sand.

I was alone and tired out; the next possible camping-place--for I had
no water--was many miles away. Yet it was written that I should press
on to that other distant well, weary though I was.

First, then, as I came up, I perceived numbers of vultures in the air;
and I began to fear that someone near to his end lay at the well. But
when, from the top of a mound, I obtained a closer view, I saw a sight
that, after one quick glance, caused me to spur up my tired horse and
to fly--fly, with panic in my heart.

The brilliant moon bathed the hollow in light and cast dense shadows
of the palm stems upon the slope beyond. By the spring, his fallen
face ghastly in the moonlight, in a clear space twenty feet across,
lay a dead man.

Even from where I sat I knew him; but, had I doubted, other evidence
was there of his identity. As I mounted the slope, thousands of fiery
eyes were turned upon me.

God! that arena all about was alive with jackals--jackals, my friends,
eaters of carrion--which, silent, watchful, guarded the wizard dead,
who, living, had been their lord!




II

LURE OF SOULS


I

This is the story which Bernard Fane told me one afternoon as we sat
sipping China tea in the Heliopolis Palace Hotel, following a round
upon the neighboring links.

The life of a master at the training college (said Fane) is beastly
uneventful, taken all around; not even _your_ keen sense of the
romantic could long survive it. The duties are not very exacting,
certainly, and in our own way I suppose we are Empire builders of a
sort; but when you ask me for a true story of Egyptian life, I find
myself floored at once.

We all come out with the idea of the mystic East strong upon us, but
it is an idea that rarely survives one summer in Cairo. Personally, I
made a more promising start than the average; an adventure came my way
on the very day I landed in Port Said, in fact it began on the way
out. But alas! it was not only the first, but the last adventure which
Egypt has offered me.

I have not related the story more than five hundred times, so that you
will excuse me if I foozle it in places. I will leave you to do the
polishing.

On my first trip out, then, I joined the ship at Marseilles, and saw
my cabin trunk placed in a nice deck berth, with the liveliest
satisfaction. Walking along the white promenade deck, I felt no end of
a man of the world. Every Anglo-Indian that I met seemed a figure from
the pages of Kipling, and when I accidentally blundered into the
_ayahs'_ quarters, I could almost hear the jangle of the temple bells,
so primed was I with traditions of the Orient--the traditions one
gathers from books of the lighter sort, I mean.

You will see that in those days I was not a bit _blasé_; the glamour
of the East was very real to me. For that matter, it is more real than
ever, now; Near or Far, the East has a call which, once heard, can
never be forgotten, and never be unheeded. But the call it makes to
those who have never been there is out of tune, I have learned; or
rather, it is not in the right key.

Well, I had a most glorious bath--I am sybarite enough to love the
luxuriance of your modern liner--got into blue serge, and felt no end
of an adventurer. There was a notice on the gangway that the steamer
would not leave Marseilles until ten o'clock at night, but I was far
too young a traveller to risk missing the boat by going ashore again.
You know the feeling? Consequently I took my place in the saloon for
dinner, and vaguely wondered why nobody else had dressed for the
function. I was a proper Johnny Raw, no end of a Johnny Raw, but I
enjoyed it all immensely, nevertheless. I personally superintended the
departure of the ship, and believed that every deck-hand took me for a
hardened globe-trotter; and when at last I sought my cosy cabin, all
spotlessly white, with my trunk tucked under the bunk, and, drawing
the little red curtain, I sat down to sum up the sensations of the
day, I was thoroughly satisfied with it all.

Gad! novelty is the keynote of life, don't you think? When one is
young, one envies older and more experienced men, but what has the
world left of novelty to offer them? The simple matter of joining a
steamboat, and taking possession of my berth, had afforded me thrills
which some of my fellow-passengers--those whom I envied the most for
the stories of life written upon their tanned features--could only
hope to taste by means of big-game hunting, now, or other far-fetched
methods of thrill-giving.

It wore off a bit the next day, of course, and I found that once one
has settled down to it, ocean traveling is merely floating hotel life.
But many of my fellow-passengers (the boat was fairly full) still
appealed to me as books of romance which I longed to open. And before
the end of that second day, I became possessed of the idea that there
was some deep mystery aboard. Since this was my first voyage,
something of that sort was to be expected of me; but it happened that
I stood by no means alone in this belief.

In the smoking-room, after dinner, I got into conversation with a
chap of about my own age who was bound for Colombo--tea-planting. We
chatted on different topics for half an hour, and discovered that we
had mutual friends, or rather, the other fellow discovered it.

"Have you noticed," he said, "a distinguished-looking Indian
personage, who, with three native friends, sits at the small corner
table on our left?"

Hamilton--that was my acquaintance's name--was my right-hand neighbor
at the chief officer's table, and I recollected the group to which he
referred immediately.

"Yes," I replied; "who are they?"

"I don't know," answered Hamilton, "but I have a suspicion that they
are mysterious."

"Mysterious?"

"Well, they joined at Marseilles, just before yourself. They were
received by the skipper in person, and two of them were closeted in
his cabin for twenty minutes or more."

"What do you make of that?"

"Can't make anything of it, but their whole behavior strikes me as
peculiar, somehow. I cannot quite explain myself, but you say that you
have noticed something of the sort, yourself?"

"They certainly keep very much to themselves," I said. Hamilton
glanced at me quickly.

"Naturally," he replied.

Not desiring to appear stupid, I did not ask him to elucidate this
remark, although at the time it meant nothing to me. Of course I have
learned since, as everyone learns whose lines are cast among
Orientals, that iron barriers divide the races. But at the time I knew
nothing of this--as will shortly appear.

During breakfast on the following morning, I glanced several times at
the mysterious quartette. They had been placed at a separate table and
were served with different courses from the rest of the passengers. I
was not the only member of the company who found them interesting, but
the Anglo-Indians on board, to a man, left the native party severely
alone. You know the icy aloofness of the Anglo-Indians?

My second day at sea wore on, uneventfully enough; the bugle had
already announced the hour for dressing, and the boat-deck outside
my berth, where I had had my chair placed, was practically deserted,
when something occurred to turn my thoughts from the four Indians.
It was a glorious evening, with the sun setting out across the
Mediterranean in such a red blaze of glory that I sat watching it
fascinatedly, my book lying unheeded on the deck beside me. Right
and left of me men occupying the other deck cabins had lighted up,
and were busily dressing. Right aft was a corner cabin, larger than
the others, and suddenly I observed the door of this to open.

A slim figure glided out on to the deck, and began to advance toward
me. It proved to be that of a woman or girl dressed in clinging black
silk, and wearing a _yashmak_! She had a richly embroidered shawl
thrown over her head and shoulders, and in that coy half-light she
presented a dazzlingly beautiful picture.

It was my first sight of a _yashmak_, and, because it was worn by a
marvelously pretty woman, the thousands seen since have never entirely
lost their charm for me. I could detect the lines of an exquisitely
chiseled nose, and the long dark eyes of the apparition were entirely
unforgettable. The hand with which she held her shawl about her was of
ivory smoothness, and, like a little red lamp, a great ruby blazed
upon the index finger.

With her high-heeled shoes tapping daintily upon the deck she
advanced; then, suddenly perceiving that the promenade was not
entirely deserted, she turned, but not hastily or rudely, and glided
back to her cabin.

I have endeavored to outline for your benefit the state of my mind at
this period, hinting how keenly alive I was to romance of any sort,
provided it wore the guise of the Orient; so that it will be
unnecessary for me to explain how strong an impression this episode
made upon me. The Indian party was forgotten, and as I hastily dressed
and descended to dinner, I scarcely listened to Hamilton when he bent
toward me and whispered something about the "Strong Room."

My gaze was roaming about the spacious saloon. Even in those days I
might have known better; I might have known that no Mohammedan woman
would take her meals in a public saloon. But I was too dazzled by my
memories to summon to my aid such fragments of knowledge respecting
Eastern customs as were mine.

       *       *       *       *       *

Well, some little time elapsed before I saw or heard anything further
of the houri. I began to settle down to the routine of the trip, and
(you know how news circulates through a ship?) it was not very long
before I knew as much as any of the other passengers knew.

Hamilton was a sort of filter through which it all came to me, and of
course it was not undiluted, but colored with his own views. The lady
of the _yashmak_, he informed me, was a member of the household of a
wealthy Moslem in the neighborhood of Damascus. She was travelling via
Port Said, and taking a Khedivial boat from there to Beyrût. He was a
perfect mine of information, but his real interest was centered all
the time on the party of four Indians.

"They are emissaries of the Rajah of Bhotana," he informed me
confidentially. "The mystery begins to clear up. You must have read
about a month ago that Lola de l'Iris was selling some of her jewelery
and devoting the proceeds to the founding of an orphanage or something
of the kind; quite a unique advertisement. Well, the famous Indian
diamond presented to her by one of the crowned heads of Europe was
amongst the bunch which she sold; and after staying in the West for
over fifty years, it is again on its way back to the East where it
came from."

I began to recollect the circumstances, now; the historic Indian
diamond--I do not know Hindustanî, but its name translated means "Lure
of Souls"--had been in the possession of the dancer for many years,
and its sale for such a purpose had turned the limelight upon her most
enviably. It was a new idea in advertising, and had proved an
admirable success.

So the four reticent gentlemen were the guardians of the diamond.
Under normal circumstances this might have been interesting, but, as
I have tried to make clear, another matter engrossed my attention. In
fact, I was living in a dream-world.

Of course, my opportunity came, in due course. One evening, as I
mooned on the shadowy deck--which was quite deserted, because an
extempore dance was taking place on the deck below--_she_ came gliding
along towards me. I could see her eyes sparkling in the moonlight.

At first I feared that she was going to turn back. She hesitated, in
a wildly alluring manner, when first she saw me sitting there watching
her. Then, turning her head aside, she came on, and passed me. I never
took my eyes off that graceful figure for a moment.

Coming to the rail, she leaned and looked out toward the coast of
Crete, where silver tracings in the blue marked the mountain peaks;
then, shivering slightly, and wrapping her embroidered shawl more
closely about her shoulders, she retraced her steps.

Not a yard from where I sat, she dropped a little silk handkerchief
on the deck!

How my heart leapt at that! the rest was a magical whirl; and ten
seconds later I was chatting with her.

She spoke fluent French, but little English.

She appealed to me in a way that was new and almost irresistible; it
was an appeal quite Oriental, sensuous--indescribable. I just wanted
to take her in my arms and kiss those tantalizing lips; talking seemed
a waste of time. Of course, I cannot hope to make you understand; but
it was extraordinary. I felt that I was losing my head; the glances of
those long dark eyes were setting me on fire.

Suddenly, she terminated this, our first _tête-à-tête_. She raised her
finger to her veiled lips and glided away into the shadows like a
phantom. A sentence died, unfinished, on my tongue. I turned, and
looked over my shoulder.

Gad! I got a fright! A most hideous Oriental of some kind, having only
one eye but that afire with malignancy, was watching me from where he
stood half concealed by a boat.

My lily of Damascus was guarded!

Humming, with an assumption of unconcern, I strolled away and joined
the dancers below.


II

That was the beginning, then. I cursed to think how short a time was
at my disposal; but since, the very next morning, I found myself
enjoying a second delicious little stolen interview, I perceived that
my company was not unacceptable.

What? oh, I had lost my head entirely; I admit it.

It was an effort to speak of matters ordinary, topics of the ship; my
impulse was to whisper delicious nonsense into those tiny ears.
However, I forced myself to talk about things in general, and told her
that the famous diamond, Lure of Souls, was aboard.

This was news to her, and she seemed to be tremendously interested.
Her interest was of such a childish sort, so naïve, that the project
grew up in my mind at that very moment--the project that was to
terminate so disastrously. It was hardly a matter of so many words;
there was nothing definite about the thing at all, and this, our
second interview, was cut short in much the same manner as the first.

"_Ssh! Mustapha!_"

With those whispered words, and a dazzling smile, this jewel of
Damascus who interested me so much more deeply than the Rajah's
diamond, departed hurriedly--and I turned to meet again the malignant
gaze of the wall-eyed guardian.

The sort of romance in which I was steeped at that time flourishes and
grows fat upon incidents of this kind. I have searched my memory many
a time since then, for some word or hint to prove that the
conversation about the diamond was opened and guided in a desired
direction by the lady of the _yashmak_; but excluding transmission of
thought, I could never find any evidence of the kind--have never been
able to do so.

Certainly my memories of that period are hazy except in regard to
Nahèmah. If I were an artist, I could paint her portrait from memory
without the slightest error, I think. She occupied my thoughts to the
exclusion of all else. But the project was formed and carried out.
Hamilton was one of those popular men who seem born to occupy the
chair at any kind of meeting at which they may be present; he
organized almost every entertainment that took place on board. At
first he was not at all keen on the idea.

"There are all sorts of difficulties," he said; "and one doesn't care
to ask a favor of a native. At any rate one doesn't care to be
refused."

But I had set my heart upon gratifying Nahèmah's curiosity, and, with
the aid of Hamilton, it was all arranged satisfactorily. The native
guardians of the diamond were rather flattered than otherwise, and a
select little party of the "best" people on board met in the chief
officer's cabin to view Lure of Souls.

The difficulty in regard to Nahèmah was readily overcome by Hamilton
the energetic, and Dr. Patterson's wife "took her up" for the occasion
in a delightfully patronizing manner. The four swarthy, polite
Orientals were there, of course; several other ladies in addition to
Mrs. Patterson and Nahèmah, the chief officer, myself, Hamilton, and a
sepulchral Scotch curate, the Rev. Mr. Rawlingson, whom I had scarcely
noticed hitherto, and whose presence at this "select" gathering rather
surprised me.

The sea was like a sheet of glass, and this was the hottest day which
I had yet experienced. It was about an hour before lunch-time when we
gathered to view the diamond; and Mr. Brodie, the chief officer,
exercised his pawky humor in a series of elaborate pantomimic
precautions, locking the door with labored care, and treating the
ladies of the company to Bluebeard glances of frightful intensity.

Phew! if we had only known!...

Finally one of the Indians took out the diamond from its case--which
had been brought from the strong-room a few minutes before. It was a
wonderful thing, I suppose, of quite unusual size, and it sparkled and
gleamed in the sunlight streaming through the open porthole in an
absolutely dazzling fashion. I had ranged myself close beside Nahèmah.
Each of us was permitted to handle the stone. It was I who passed it
to her, Mr. Rawlingson having passed it to me. She held it in the palm
of her little hand, and her eyes sparkled with childish delight as she
bent to examine the gem. Then a very strange thing happened.

From somewhere behind me--I was sitting with my back to the
porthole--a dull gray object came leaping and twirling; and a
scorpion--I have never seen a larger specimen--fell upon Nahèmah's
wrist!

She uttered a piercing cry, dropped the diamond and brushed the horrid
insect from her wrist; then fell swooning into my arms....

A scene of incredible confusion followed. The four Indians, ignoring
the presence of the scorpion, dropped like cats upon the floor,
seeking for Lure of Souls. Mrs. Patterson and I carried Nahèmah to
the sofa hard by and laid her upon it. Just as we did so the scorpion
darted from between the end of the sofa and the wardrobe, and the
chief officer put his foot upon it.

Ensuing events were indescribable. Since the diamond had not yet been
picked up, obviously the cabin door could not be unlocked; so in the
stuffy atmosphere of the place it was a matter of some difficulty to
revive Nahèmah. Meanwhile, four wild-eyed Indians were creeping about
amongst our feet--like cats, as I have said before.

In the end, just as the girl began to revive somewhat, it became
evident that Lure of Souls was missing. A pearl shirt button, the
ownership of which we were unable to establish, was picked up, but no
diamond.

The chief officer showed himself a man of priceless tact. He rang for
the stewardess, and the ladies were shepherded to a neighboring,
vacant cabin. Then the door was relocked, and Mr. Brodie proceeded to
strip, placing his garments one by one upon the little folding table
for examination. He was not satisfied until every man present had
overhauled them. We all followed his example, the Rev. Mr. Rawlingson
last of all ... and Lure of Souls was still on the missing list!

Then we gave the chief officer's cabin such a turnout as it had never
had before, I should assume. Our quest was unrewarded. Meanwhile, the
ladies had been submitted to a similar search in the adjoining cabin;
same result.

With great difficulty we succeeded in hushing up the matter to a
certain extent; but the captain's language to the chief officer was
appalling, and the chief officer's remarks to Hamilton were equally
unparliamentary; whilst Hamilton seemed to consider that he was
justified in placing the whole blame upon me, which he did in terms
little short of insulting. The four Indians apparently regarded all
of us with equal suspicion and animosity.

I could not foresee the end. The thing was so sudden, so serious, that
at the time it banished even thoughts of Nahèmah from my mind. I
anticipated that we should all find ourselves arrested when we reached
Port Said.

Later in the day Hamilton walked into my cabin and placed a little
cardboard box upon the dressing-table. It contained the crushed body
of the scorpion.

"Where did that scorpion come from?" he demanded abruptly.

It was a question which already had been asked fully a thousand times,
yet no one had discovered an intelligent reply.

I shook my head.

"It came from the open porthole," he replied, "and as it's a thousand
to one against a scorpion being aboard, somebody was _carrying_ it for
this very purpose--somebody who was on the deck outside the chief
officer's cabin _and who threw the scorpion_ into the cabin."

"But such a deadly thing...."

"Have a good look," said Hamilton, turning the insect over with a
lead pencil; "this one isn't deadly at all. See!--his tail has been
cut off!"

I looked and stifled an exclamation. It was as Hamilton had said. The
scorpion was harmless.

       *       *       *       *       *

I never once set eyes upon Nahèmah again until we arrived at Port
Said. Then I saw her preparing to go ashore in one of the boats. I
managed to join her, ignoring the scowls of her one-eyed attendant,
and we arrived at the quay together. Right there by the water's edge
a most curious scene was being enacted. Surrounded by two or three
passengers and a perfect ring of uniformed officials, Hamilton, very
excited, watched his baggage being turned out upon the ground. He saw
me approaching.

"Hang it all, Fane," he cried, "this is disgraceful!--I don't know
upon whose orders they are acting, but the beastly police are
searching my baggage for the diamond...."

I thought it very extraordinary and said as much to the Rev. Mr.
Rawlingson, who was one of the onlookers.

"It is very strange indeed," he said mildly, turning his gold-rimmed
spectacles in my direction.

A moment later, to my horror and indignation, Nahèmah was submitted to
the same indignity! The crowd had been roped off from the part of the
quay upon which we stood, and I could see that the whole thing had
been arranged beforehand in some way, probably by wireless from the
ship. Curiously, as I thought at the time, my own baggage was not
examined in this way, but I was detained long enough to lose sight of
Nahèmah and her one-eyed guardian. When I got to the hotel I indulged
in some reflection. It occurred to me that Hamilton was bound for
Colombo, which made it rather singular that he should have had his
baggage put ashore at Port Said.

I should have liked to have searched the town for my lady of the
_yashmak_, but having no clue to her present whereabouts, realized
the futility of such a proceeding. My last thought before I fell
asleep that night was that some day in the near future I should
visit Damascus.


III

I saw very little of Port Said, for we had arrived in the early
morning and I was departing for Cairo by a train leaving shortly
before midday. I wandered about the quaint streets a bit, however,
and wondered if, from one of the latticed windows overhanging me,
the dark eyes of Nahèmah were peering out.

Although I looked up and down the train fairly carefully, I failed to
find among the passengers anyone whom I knew, and I settled down into
my corner to study the novel scenery uninterruptedly. The shipping in
the canal fascinated me for a long time as did the figures which moved
upon its shores. The ditches and embankments, aimlessly wandering
footpaths, and moving figures which seemed to belong to a thousand
years ago, seized upon my imagination as they seize upon the
imagination of every traveller when first he beholds them.

But, properly speaking, my story jumps now to Zagazig. The train
stopped at Zagazig; and, walking out into the corridor and lowering a
window, I was soon absorbed in contemplation of that unique town. Its
narrow, dirty, swarming streets; the millions of flies that boarded
the train; the noisy vendors of sugar cane, tangerine oranges and
other commodities; the throng beyond the barriers gazing open-mouthed
at me as I gazed open-mouthed at them--it was a first impression, but
an indelible one.

I was not to know it was written that I should spend the night in
Zagazig; but such was the case. Generally speaking, I have found the
service on the Egyptian State Railway very good, but a hitch of some
kind occurred on this occasion, and after an hour or so of delay, it
was definitely announced to the passengers that owing to an accident
to the permanent way, the journey to Cairo could not be continued
until the following morning.

Then commenced a rush which I did not understand at first, and in
which, feeling no desire to exert myself unduly, I did not
participate. Half an hour later I ascertained that the only two hotels
which the place boasted were full to overflowing, and realized what
the rush had meant. It was all part of the great scheme of things, no
doubt; but when, thanks to the kindly, if mercenary, offices of the
International Sleeping Car attendant, I found myself in possession of
a room at a sort of native _khân_ in the lower end of the town, I
experienced no very special gratitude towards Providence.

I have enjoyed the hospitality of less pleasing caravanserai since,
but this was my first experience of the kind, and I thought very
little of it.

My room boasted a sort of bed, certainly, but without entering into
details, I may say that there were earlier occupants who disputed its
possession. The plaster of the walls--the place apparently was built
of a mixture of straw and dried mud--provided residence not only for
mosquitoes, but also for ants, and the entire building was redolent
of an odor suggestive of dried bones. That smell of dried bones is
characteristic, I have learned, of the sites of ancient Egyptian
cities (Zagazig is close to the ruins of ancient Bubastis, of course);
one gets it in the temples and the pyramids, also. But it was novel to
me, then, and not pleasing.

I killed time somehow or other until the dinner hour; and the train,
which now reposed in a siding, became a rendezvous for those who
desired to patronize the dining-car. Evidently no sleeping-cars were
available (or perhaps that idea was beyond the imagination of the
native officials), and having left a trail of tobacco smoke along the
principal native street, I turned into my apartment which I shared
with the ants, mosquitoes--and the other things.

An examination of my rooms by candle-light revealed the presence of a
cupboard, or what I thought to be a cupboard, but opening the double
doors I saw that it was a window, latticed and overlooking a lower
apartment; so much I perceived by the light of an oil lamp which stood
upon the table. Then, stifling a gasp of amazement, I hastily snuffed
my candle and peered down eagerly at that incredible scene....

Nahèmah, longer veiled, was sitting at the table, and opposite to her
was seated the hideous wall-eyed attendant!

They were conversing in low tones, so that, strive as I would, I could
not overhear a word. You ask me why I spied upon the lady's privacy in
this manner? For a very good reason.

Midway between the two, upon the rough boards of the table, lay Lure
of Souls, twinkling and glittering like a thing of incarnate light.

I observed that there was a door to the room below, almost immediately
opposite the window through which I was peering ... and this door was
opening very slowly and noiselessly. At least, _I_ could hear no
noise, but the one-eyed man detected something, for suddenly he
started up and did a remarkable thing. Snatching up the diamond from
the table, he clapped it into the eyeless cavity of his skull and
turned in a twinkling to face the intruder.

Then the door was thrown open, and Hamilton leapt into the room.

I could scarcely credit my senses. Honestly, I thought I was dreaming.
Hamilton's whole face was changed: a hard, cunning look had come over
it, and he held a revolver in his hand. Nahèmah sprang to her feet as
he entered, but he covered the pair of them with his revolver, and
pointing to the one-eyed man muttered something in a low voice. Rage,
fear, rebellion chased in turn across the evil features of One-eye;
but there was something about Hamilton's manner that cowed.

Manipulating the sunken eyelids as though they had been of rubber,
the guardian of the veiled lady slipped the diamond into the palm
of his hand and tossed it, glittering, on to the table.

Hamilton's expression of triumph I shall never forget. One step
forward he took and was about to snatch up the gem when--out of the
dark cavity of the doorway behind him stepped a second intruder.

It was the Rev. Mr. Rawlingson!

The reverend gentleman's behavior was most unclerical. He leapt upon
the unsuspecting Hamilton like a panther and screwed the muzzle of a
revolver into that gentleman's right ear with quite unnecessary vigor.

"You have been wasting your time, Farland!" he snapped in a voice that
was quite new to me. "That is, unless you have turned amateur
detective."

He made no attempt to reach for the diamond, but just held out his
hand, and with his eyes fixed upon Hamilton, silently commanded the
latter to hand over the gem. This Hamilton did with palpable
reluctance. Mr. Rawlingson, who, though still clerically garbed, had
discarded his spectacles, slipped the stone into his pocket, snatched
the revolver from Hamilton's hand and jerked his thumb in the
direction of the open door. Hamilton shrugged his shoulders and walked
out of the room. For scarce a moment did Rawlingson's eyes turn to
follow the retreating figure, but the chance was good enough for the
wall-eyed man.

He launched himself through space like nothing so much as a kangaroo,
bearing Rawlingson irresistibly to the floor! With his lean hands at
the other's throat he turned his solitary eye upon Nahèmah, muttering
something gutturally. After a moment's hesitation she ran from the
room.

       *       *       *       *       *

Twenty seconds later I was downstairs, and ten seconds after that was
helping Rawlingson to his feet. He was considerably shaken and boasted
a very elegant design in bruises which was just beginning to reveal
itself upon his throat; but otherwise he was unhurt.

"I have lost her, Mr. Fane!" were his first words. "She knows this
part of the world inside out. I have no case against Farland, but I
am sorry to have lost the woman."

Was my mind in a whirl? Did I think that madness had seized me?
Replies both in the affirmative; I was simply staggered.

I always go to pieces with this part of the yarn, being an unpractised
narrator, as I have already explained; but I may relieve your mind
upon one point. I never saw Nahèmah and the one-eyed man again, nor
have I since set eyes upon Hamilton. Mr. Rawlingson, the last time I
heard from him, was in similar case.

The explanation of the whole thing was something of a blow to me, of
course. The lily of Damascus who had fascinated me so hopelessly was
no Eastern at all; you will have guessed as much. She was a
Frenchwoman, I believe; at any rate they had a long record up against
her in Paris. She had gone out after Lure of Souls, and very
ingeniously had made me her instrument. As Mr. Rawlingson explained
to me, what had probably taken place was this:

The harmless scorpion, specially brought along for some such purpose,
had been thrown into the chief officer's cabin from the open porthole
by the one-eyed villain. That had been the cue for Nahèmah to drop the
shirt button, and, whilst the occupants of the cabin were in
confusion, to toss the diamond out on to the deck where her accomplice
was waiting. The search of their effects had been futile, of course;
no one had thoughts of searching the eye-cavity of her Eastern
companion.

Where did Hamilton come in? Hamilton was one James Farland, an
American crook of the highest accomplishments, known to the police of
the entire civilized world. He, too, had gone out for Lure of Souls,
but the woman, his professional competitor, had proved too clever for
him.

The Rev. Mr. Rawlingson? He was Detective-Inspector Wexford of New
Scotland Yard. Yes, it's a rotten story, from a romantic point of
view.




III

THE SECRET OF ISMAIL


I

Mustapha Mirza knew it--Mustapha Mirza, the blind Persian who makes
shoes hard by the Bâb ez-Zuwêla and in the very shadow of the minarets
of Muayyâd; Hassan es-Sîwa of the Street of the Carpet-sellers in the
Mûski, Hassan, who, where another man has hands, has but hideous
stumps, knew it, and because of him it was that Abdûl Moharli sought
it--Abdûl the mendicant who crouches on the steps of the Blue Mosque
muttering, guttural, inarticulate, and pointing to the tongueless
cavity of his mouth. Now I know it; but not from Abdûl Moharli: may
Allah, the Great, the Compassionate, defend me!

I say "May Allah defend me," yet I am no Moslem; I have no spot of
Egyptian blood in my veins. No, I am a pure Greek of Cos, of Cos the
home of the loveliest women in the world; and my mother was one of
these, whilst my father was a Cretan, and a true descendent of Minos.
My story perhaps will not be believed, for always it has been my fate
to be maligned. You will ask, perhaps, what I was doing in the Mâzi
Desert between Beni Suêf and the Red Sea, but I reply that my cotton
interests--for I have cotton interests in the Delta--often lead me
far afield. You do not understand the cotton industry or this
explanation would be unnecessary. It is only those who do not
understand the cotton industry that speak of _hashish_. _Hashish!_
I leave it to the Egyptians and the Jews to deal in _hashish_; I am
neither a Jew nor an Egyptian, but a Greek of Cos, who would not soil
his hands with such a trade--no.

Upon my business, then, my legitimate business, I found myself with
a small company of servants encamped by the Wâdi Araba. At the Wâdi
Araba I had a commercial acquaintance, a sheikh of the Mâzi Arabs.
Those villains who say that he was a "go-between," that my business
was not with him, but through him with a port of the Red Sea, dare not
say as much to my face; for there is a law in the land--even in the
land of Egypt, now that the British hold power here.

I had reached the point, then, whereat it was my custom to meet my
business acquaintance and to discuss certain affairs in which we were
interested. My servants had erected the tent in which I was to sleep,
and the camels lay in a little limestone valley to the west, their
eyes mild because they knew that the day's work was ended; for it is
a foolish mistake to suppose that the eye of a camel is mild at any
other time. The camel knows the secret name of Allah--and that name is
Rest.

The violet after-glow, which is the most wonderful thing in Nature,
crowned the desert with glory right away to the porphyry mountains.
I stood at my tent door looking westward to the Nile. I stood looking
out upon the waste of the sands, the eternal sands which are a belt
about Egypt; and my thoughts running fleetly before me, crossed the
desert, crossed the Nile, and came to rest in the verdant, fertile
Fàyûm, its greenness sweet to look upon in the heat of such an
evening, its palms fashioned in ebony black against the wondrous sky.
Yes, I, who am a Greek, love the Fàyûm more than any spot on earth;
the modern clamor and dust of Cairo are hateful to me, although my
business often takes me there, and also to Alexandria, the most
European city in the East, and to me the most detestable. But my
business is in the Delta and it is a good business, so why should
I complain?

I stood at my tent door, and I thought of many things, though little
of the matters which had brought me there; a faint cool breeze fanned
my brow, and about me was that great peace which comes to Egypt with
the touch of night. My servants were silent in their encampment, and
the shrieking of the camels had ceased. About me, then, all was
sleeping; only I was awake, only I was there to receive Abdûl Moharli
and his secret--the secret of Ismail.

By the pattering of his bare feet upon the sand, I first learned of
his coming, but for a long time I could not see him, for his way led
him through the valley where the camels slept, and a mound obscured
my view. But presently I heard his panting breaths and his little
delirious cries of fear, which were like sobs, and presently, again,
I saw him staggering over the slope. At the sight of me he uttered one
last gasping cry and fell forward on his face unconscious--like a dead
man.

I hurried to him, stooped and raised him. His face was dreadful to
look upon. His eyes were sunken in his skull, and his flesh shrivelled
as by long fasting. His beard was filthy, knotted and unkempt, and his
hair a black mat streaked with dirty gray. He was thin as a mummy and
the bones protruded through his skin. He was as one who is dying from
excess of _hashish_.

Ah! I know how they look, those poor fools who poison themselves with
the Indian hemp. I wonder Allah does not strike down the villain who
places that poison within their reach. I use the term "Allah" because
my business brings me much in contact with the natives, but I am no
Moslem, as I have related. Father Pierre of Alexandria can tell you
how devoted a Christian I am.

Drink and food revived him somewhat; and as I sat beside him in my
tent that night he babbled to me, half deliriously; he raved, and to
another it might have seemed the fancies of a poor madman which he
poured into my ears. For he spoke of a secret oasis and of a sheikh
who had lived since the days of Sultan Kalaûn; of a treasure vast as
that of Suleyman--and of magic, black magic; of the transmuting of
gold and the making of diamonds.

But I, who am a Greek, and one who has lived all his life between
Alexandria and the Red Sea; I who know the Garden of Egypt as another
knows the palm of his hand--I detected in this delirium the shadow of
a truth. To me it became evident that this wretched being who had
fled, a hunted thing, over the trackless desert for many days and
nights--it became evident to me, I say, that he spoke of the far-famed
secret of Ismail.

You would ask: What is the secret of Ismail? I would tell you, ask it
of Hassan the Handless, of Mustapha, the blind Persian of the Bâb
ez-Zuwêla; better still, ask it of any son of the Fàyûm, of any man of
the Mâzi. None of them will answer you, for none save Hassan and
Mustapha knows the strange truth--Hassan and Mustapha, and Abdûl
Moharli ... and no one of these three knows all, nor will reveal what
he knows.

Ah! how my heart leapt and how my eyes must have gleamed in the
darkness of the tent, yet how cold a fear clutched at the life within
me. The night seemed suddenly to become a thin curtain veiling eyes
that watched, the empty desert a hiding-place for unseen multitudes
that listened; the faint breeze raising the flap of the tent, ever so
gently, ever so softly, assumed the shape of a malignant hand that
reached for my throat, that sought to stifle me ere the secret, the
deathly secret of Ismail should be mine.

Abdûl Moharli was the name of this wanderer; and as he spoke to me,
gulping down great draughts of water between the words, ever he
glanced to right and left, over his shoulder and all about him.

"It is four days from here," he whispered hoarsely; "due south in the
direction of the porphyry quarries and the Mountain of Smoke. There
is a tiny village and all the inhabitants are of the race of Saïd Ebn
al As, being descendants of the companion of the prophet. I had long
supposed that this race of heretics was extinct; but it is not so, O
my benefactor; with these eyes, have I seen the houses wherein they
dwell. By the strategy of which I have spoken did I penetrate to their
secret dwelling-place and win their unsuspecting love."

And then, clutching me to him with his bony hands, he spoke in hushed
and fearful tones of the house of the Sheikh Ismail Ebn al As. It was
the fabled treasure of this holy man which had been the lodestone
drawing Abdûl Moharli out into the desert. Something of his fear, of
his constant apprehension seized upon me too; and as he glanced
tremblingly first over this shoulder and then over that, so likewise
did _I_ glance, until I seemed to crouch in a world of spies listening
to a secret greater than that of the Universe.

I pronounced the _Takbîr_, "Great is the Lord!"--a superstitious
custom which I have acquired from my business acquaintances. I made
the sign of the Cross and called upon the name of the Holy Virgin.
Almost I feared to listen further, yet I lacked the courage to
abstain.

"Not with mine eyes have I beheld the treasure of Ismail," he
whispered to me, this shadow of a man, this living mummy, those same
eyes rolling in their sunken sockets; "nor with mine ears have I
heard it named. These hands have never touched it; yet the secret of
Ismail is _my_ secret."

So far he had proceeded and no further, when a slight noise, that was
not of my imagination, came from immediately outside the tent. On the
instant I sprang forth ... but no one was there and nothing now
disturbed the solitude of the desert about me. A moment I stood,
peering to left and right, into the void of the velvet dusk; no more
than a moment, I can swear, yet long enough for that dreadful thing
to happen--that thing which sometimes haunts my dreams.

Shrill and awful upon the silence it burst; the scream of a stricken
man. It stabbed me like a knife; and as a creature of clay I stood,
unable to stir or think. It died away, in a long wail of pain, that
gave place to a guttural, inarticulate babbling--a choking, sobbing
sound indescribable, but that may not be forgotten once it has been
heard.

No living thing, as I can testify, entered or left the tent; so far
the evidence of my senses bears me. But that one had entered and left
it, unseen, I learned, when, throwing off this palsy of horror, I
staggered back to the side of the one who knew the secret of Ismail.

He lay writhing upon the ground; blood issued from his mouth. The
tongue of Abdûl Moharli had been torn out!


II

Three weeks later I had my first sight of the secret oasis. The fate
from which Abdûl had fled had overtaken him as I have related, in my
tent, and from that moment until we parted company--for this poor
wretch survived his mutilation--not another hint could I glean from
him respecting the discovery for which he had paid so terrible a
price.

In the first place, he lacked the accomplishment of writing and in
the second place his fear of the vengeance of Ismail had become a
veritable madness. I left him at Beni Suêf, filled with a
determination to probe this mystery for myself. Suitably prepared for
such an undertaking I set out alone from Dér Byâd, and undertook the
four days' journey which I had planned.

In a little gorge, arid, shadeless, in which only a few stunted
tamarisks grew, but affording a sort of hiding-place for myself and
my camel, I made my base of operations. Provisions of a sort I had
plenty, but for water I must depend on the secret oasis, which I
estimated to be not more than four miles distant. In the dead of night
I set out, making for a series of mounds or hillocks rising up from
the rocky face of the plateau. Cautiously I ascended their slopes,
ever watchful and with eagerly beating heart; and it was lying prone
upon the crest of the greatest of these that I first saw the village
and the oasis.

There was nothing extraordinary in the appearance of the village; it
presented to the eye the usual group of small, squat houses clinging
to the trunks of the palm trees and surrounding a shrine or mosque
boasting a wooden minaret. There were tilled fields and palm groves
to the left of the village and a large house surrounded by white walls
embracing extensive gardens. My spirits rose high. Within that house
lay the secret of Ismail.

I determined to approach from the left, where I should be able to
take advantage of the far-cast shadows of the palm groves and of
the direction of the faint breeze; for most of all I feared the dogs,
without which no Arab village is complete. Sure enough, although I
had elected to approach the left of the village and although I crawled
laboriously upon hands and knees, the accursed brutes apparently
scented me or heard me and made night hideous with their clamor.

Flat upon the ground I lay, awaiting the dogs who bore down upon me
snarling, their fangs bared. I had come prepared for this; but,
mysteriously, at a point by the end of the palm grove and some twenty
yards away from me, the pack halted, and after a time became silent.
This was unaccountable but fortunate; and after waiting a while longer
to learn if anyone had been aroused by the outcry, I advanced towards
the wall of the garden, passing stealthily from palm to palm.

I observed that the mosque was a more important building than I had
supposed, with a tomb on the right of the entrance surmounted by a
white dome. A passage leading to the courtyard, which presented a
charming picture in the moonlight, its fountain overshadowed by
acacias, reminded me very much of that in the Mosque of Muayyâd in
Cairo. As in the latter, a double arcade surrounded it on three sides
and the columns were of some kind of marble and sculptured with
inscriptions in Arabic. I had a glimpse of a blue-tiled sanctuary,
through a fine _mushrabîyeh_ screen beneath the pointed arches.
Arabesques in colored glass rendered the windows very beautiful to
look upon. Nothing stirred within the village, as I crept along the
narrow lane separating the mosque from the wall of the garden. Beyond
prospecting the ground, I had no definite plans for to-night; but Fate
had willed it that I was to become more deeply involved in the affair
than I had designed or intended.

A side door opened from the garden at a spot nearly opposite the
little wooden platform which served as the minaret of the mosque; and
the mud bricks of the porch were so broken and decayed by time that I
perceived here an opportunity of mounting to the top of the wall, an
opportunity of which I instantly availed myself.

Yes, in spite of my peaceful calling (I have explained that I have
cotton interests in the Delta) my life has not been unadventurous nor
have I ever hesitated to incur risk where profit might be gained.
Therefore, having climbed to the top of the wall, unmolested, and
perceiving at a spot some little distance to the right a sort of
trellis overgrown with purple blossom, I did not hesitate to make for
it and to descend into the garden. I had just completed the descent,
and stood looking cautiously about me, when a sound disturbed the
silence--a sound so entirely unexpected, in that place, at such an
hour, that it turned my blood cold, bringing to my mind all those
stories of the black magic for which the people of this oasis were
famed.

It was the sound of a woman singing; and although the song she sang
was a familiar Arab love song and the voice of the singer was sweet,
if very mournful, the effect, as I have said, was weird to a degree.

    _Ashik yekul l'il hammám hát le genáhak yom_
    (A lover said to a dove, "Lend me your wings for a day," etc.)

Overcoming the fear and astonishment which momentarily had deprived
me of action, I advanced with the utmost caution in the direction from
whence this mysterious singing seemed to proceed. Passing an angle of
the house, where the stucco wall ran sheerly up to a _mushrabîyeh_
window, I perceived before me a smaller, detached building in the form
of a sort of pavilion. Some fine acacias overhung its white and
glistening dome, in which were little windows of colored glass.
Concealed in the shadow of the house, I stood looking towards this
smaller building, observing with astonishment that it possessed a
massive, bronze-mounted door.

Indeed, in many respects, and in spite of the charming picture which
its jeweled appearance presented, it might well have been the tomb of
some holy Sheikh. But seated on an old-fashioned _mastabah_ before
the entrance were two huge negroes of most ferocious aspect, armed
with scimitars which glittered evilly in the light of the moon!

I drew back sharply into the shelter of the projecting wall. One of
the negroes seemed to slumber, but the wicked black eyes of his
companion were widely open and he revealed his ivory teeth in a
frightful leer. The beating of my heart almost suffocated me, for I
ascribed that ghastly grimace to the fact that the negro had detected
my presence and was already gloating over the pleasing prospect of my
swift and bloody despatch. For many agonized moments I lurked there,
one hand clutching the stucco wall and the other resting upon the butt
of a new Colt magazine pistol which I had taken the precaution to
purchase in Alexandria a week earlier.

When again I ventured to protrude my head, I learned how groundless
my fears had been; I realized that the loathsome contortion of the
negro's countenance represented a smile of appreciation. He was
listening to the unseen singer whose voice now stole again upon the
silence of the night! His blubber lips drooped open cavernously and
his fierce little eyes blinked in stupid rapture.

It appeared to me, now, that the sweet voice proceeded from some
subterranean place: I thought that I was listening to the song of a
_ginneyeh_. I remembered how the Sheikh Ismail was reputed to be the
son of an _Efreet_ and an Arabian princess, and to have lived in that
oasis for generations, since the reign of the Sultan Mohammed Nâsir
ibn-Kalaûn, who had expelled him from Cairo as a magician. He was said
to possess the secrets of Geber and of Avicenna--the great Ibn Sina of
Bokhara; to possess the Philosophers' Stone and the _Elixir Vitæ_. In
this pavilion with the bronze door I beheld the magician's
treasure-house, guarded, within, by a _ginneyeh_ and, without, by
ghouls or black _Efreets_!

You will understand that these childish superstitions sometimes
overcome me, because I have lived so long among those who believe
them; but to me, a Greek, possessing the consolation of the true
religion, it was only momentary, this cold fear which belongs to
ignorance and is bred in the blood of the Moslem but finds no place
in the heart of a true Christian.

And now the Fates again took a hand in the game. The pack of curs in
the distant palm grove set up a sudden tempest of sound, so that they
seemed to have become possessed of a million devils. It was a
disturbance infinitely louder and more prolonged than that with which
the dogs had greeted my appearance, and I had barely time to throw
myself flat in the depths of a black and friendly shadow ere the two
negroes, monstrous in the moonlight, passed me silently and trotted
off in the direction from whence the uproar proceeded. You will say,
no doubt, that a madness as great as that of the dogs possessed me;
but because what I tell you is true, you must not be surprised to find
it strange.

Allowing the negroes time to reach the gate for which I divined them
to be making, I ran across the moon-bathed garden to the door of the
pavilion.

You must understand that my madness was not entirely without method;
for I had a vague plan in my mind: it was to ascertain the character
of the lock upon the bronze door (for you must know that I am skilled
in the craft of the locksmith), and then, passing beyond the pavilion,
which I was assured was the treasure-house of Ismail, to make my
escape over the garden wall at some point to the west and return to
my base in the desert ravine armed with a knowledge of the enemy's
dispositions.

But, as I have said, the Fates took a hand. The sweet-voiced singer
ceased her song as I approached the pavilion; and, at the moment that
I set foot upon the lower step, her voice--by Allah! whose Name be
exalted, it was sweet as honey!--addressed to me these words:

"O my master, at last thou art come! Here is the key! enter ere they
return."

Whilst I stared blankly upward to the open lattice from whence the
invisible speaker thus addressed me, an antique key wrapped in a piece
of perfumed silk, fell almost upon my head!


III

Dazed though I was by the complete unexpectedness of this happening
I doubt if I should have had the temerity to pursue the matter further
that night but for the sound of fleetly running footsteps of which at
this moment I became aware.

My escape was cut off! If I endeavored to pass around the pavilion in
accordance with my original plan I should undoubtedly be perceived. My
only hope lay in accepting the invitation so singularly given. With
trembling hands I fitted the key to the cumbersome lock, opened the
door, and entered the pavilion. My presence of mind had not completely
deserted me and before closing the door I withdrew the key.

I found myself in a saloon of extraordinary magnificence, furnished
with mattresses covered with silk and lighted by hanging lamps and by
candles, and having at its upper end a couch of alabaster decorated
with pearls and canopied by curtains of satin peacock-blue. From a
carved wooden archway draped with cloth of gold there leaped forth a
girl of such surpassing loveliness that her image must forever reside
in my heart together with those of the saints.

Conceive all the dark-eyed beauties of Oriental poetry, of Hafiz, of
Omar, of Attâr, and from each distil the very essence of female
loveliness; though you combine them all in one rapturous vision of
delight you will have conceived but a feeble shadow of shadows of this
wondrous reality who now stood panting before me, her red lips parted
and her bosom tumultuous.

I think if the light in her eyes had been for me I could gladly have
died for her and found death sweet; but as her gaze met mine a
pitiful change took place in that lovely countenance. Her color fled
and she swayed and almost fell.

"Oh," she whispered, "thou art not my beloved! O Allah! this is some
snare that Ismail hath set for my feet! Who art thou? who art thou?"

But because of the excess of the loveliness of the speaker, from whom
I could not remove my eyes, and because as I stood in that perfumed
apartment it seemed to me that I was no longer a real man, but a
figment of some _Efreet's_ dream, I found myself incapable of both
speech and action.

Yet I was speedily to know that the Fates, which had thrust me into
that saloon--nay, which had brought me across the desert to that
secret oasis--were not yet wearied of their sport.

A soft call, a lover's signal (for no true Believer will whistle at
night, since to do so is to summon the evil _ginn_) sounded from
immediately outside the bronze door, followed by a muffled rapping
upon the door itself!

"Saîd, my beloved!" cried the girl wildly, and ran towards the door.

At that very moment, and whilst I stood there like a man of clay,
I heard the negro guardians returning to their posts; I heard the
clatter of their sandals and I heard their guttural cries of rage!
Uttering a long tremulous sigh, the beautiful occupant of the pavilion
fell swooning upon the floor.

A loud imperious voice now rose above the sounds of conflict which had
commenced outside the pavilion; I heard the sound of many running
feet, and--my blood turned to ice--that of a key being inserted in the
lock of the bronze door! Power of action returned to me, though I
confess that I now grew sick with dread. Only one hiding-place was
possible: the first I could reach.

I leaped across the lovely form extended upon the floor and dropped,
almost choking with emotion, behind the alabaster couch. I had barely
gained this cover when the door was hurled open and a tall,
excessively gaunt, and hawk-faced old man entered, his eyes blazing,
his thin nostrils quivering, and his lean hands opening and closing at
his sides in a sort of clutching movement horribly suggestive and
terrifying.

He was followed by the two negroes, who were dragging between them a
young Egyptian of prepossessing appearance down whose pale face blood
was pouring from a wound in the brow.

Several other persons, principally servants of the _harêm_, brought
up the rear.

Towering over the recumbent body of the girl, the terrible old man--in
whom I could not fail to recognize the Sheikh Ismail--glared down at
her for some moments in passionate silence; then he made as if to
spurn her with his foot; then he clutched his long white beard with
both hands and plucked at it frenziedly, whilst tears began to course
down his furrowed cheeks, which had the frightful appearance of those
of a mummy.

"O light of mine eyes!" he exclaimed; "O shame of my house! O
reproach of my white hairs!"

He recovered himself by dint of a stupendous effort and turning a
fiery glance upon the captive:

"Cast him down upon the floor," he cried, "that I may spit upon him,
who is a scorn among swine and the son of a disease!"

To my unspeakable horror, the Sheikh then strode across the saloon and
seated himself upon the alabaster couch! I almost choked with fear; I
felt my teeth beginning to chatter and the beating of my heart sounded
in my ears like the throb of a _darâbukeh_. The Sheikh, fortunately
ignorant of my proximity, thus addressed the unfortunate young man who
lay at his feet:

"Know, O disgrace of thy mother, that thy death hath been decided
upon, and it shall visit thee in a most painful and unfortunate
manner. O thou spawn of offal, learn that I have been aware of thy
malevolent intentions since first thou didst seek to penetrate into my
secret. What! am I heir to all the wisdom of the ages, that I should
remain ignorant of the presence of such as thee, O thou gnat's egg, in
my house? When the partner in thine infamy didst steal the key of the
door from me, thinkest thou that mine eyes were blind to the theft, O
thou foredoomed carrion? It was in order that thy culpability should
be made manifest that I permitted thee to enter. Thy double stratagem
for quelling and then exciting the dogs, in order that the guards
might be drawn from their posts, was known to me, and the negroes had
received my orders to run to the gate in seeming accordance with
thine accursed desires, O filthy insect!"

Throughout the time that this dreadful old man thus addressed his
victim, the latter crouched upon the floor, apparently paying no heed
to his words but keeping an agonized glance fixed upon the lovely form
of the girl. I was now in a condition of such profound and dejected
fear as I had never known before and trust I may never know again. The
Sheikh continued:

"Learn of the fate of some of those who sought the secret of Ismail
before thee. One there was, Mustapha Mirza, a Persian, who came hither
to despoil me. With his eyes did he behold my treasure. To-day _he
hath no eyes_! And there was one Hassan of the Khân Khalîl. He dared
to lay violent hands upon the treasure of my house--the 'treasure' not
of gold nor jewels but of fairest flesh and blood. To-day _he hath no
hands_! Wouldst like to know of Abdûl Moharli, who learned much of
this "secret" of mine, and would have spoken of it? His tongue I threw
to the carrion crows! _Thou_, O sink of iniquity, hast not only seen
with thine eyes, heard with thine ears and laid thy filthy hands upon
the treasure of Ismail: thou hast approached thy foul lips to this
peach of Allah's garden! thou hast...."

He choked in his utterance and seemed upon the point of hurling
himself upon the young man before him: but again he recovered his
composure after a great effort and proceeded:

"The unpleasant punishments visited upon those others shall likewise
fall to thy portion, since thou hast committed like crimes; but this
shall only be in order to prepare thee for a most protracted and
painful death. Bear him forth into the courtyard."

As one who dreams an evil dream, I saw the company stream out of the
saloon, the wretched prisoner in their midst. When at last the bronze
door was reclosed and I found myself alone with the swooning girl, I
could scarce believe that even this respite was mine.

I offered a prayer to St. Antony of the Thebäid--_my_ patron saint--as
I listened to the sound of their receding footsteps; when I was
aroused from the lethargy of fear into which I had fallen by a distant
scream--a long wailing cry....

       *       *       *       *       *

I have often asked myself: How did I make my escape from that dreadful
village? You will remember that I had the purloined key of the bronze
door in my possession? Then it was to this in the first place that I
owed my preservation. To regain the garden was a simple matter, for
the Sheikh and his bloodthirsty following were engaged in the
courtyard of the house, but to St. Antony be all praise for the
circumstance that the little door opposite the mosque had been left
open--possibly by the unhappy Saïd,--and to St. Antony be all praise
that a second time I avoided the dogs....

Dawn found me staggering down into that friendly ravine which
sheltered my camel. I was utterly exhausted, for I bore a burden, but
triumphant, delirious with joy and rapture, because my burden was so
sweet. You may question me of these matters, and I shall reply: As well
as my cotton interests I have now another interest in the Delta--the
lovely "Secret" of the Sheikh Ismail Ebn al As![D]

  [D] Readers of _Tales of Abû Tabâh_ will recognize Mizmûna,
      "The Lady of the Lattice," the story of whose recovery
      by the bereaved Sheikh has already been related.




IV

HARÛN PASHA


I

I will tell you this story (said Ferrier of the Egyptian Civil) with
one reservation; comments are to be reserved for some future time. I
can only tell you what I saw with my own eyes and heard with my own
ears; I offer no explanation; I pass on the story; you can take it or
leave it.

Some of you will remember Dunlap--I don't mean Robert Dunlap, who is
chief officer of the _Pekin_, but Jack Dunlap his cousin, the
irrigation man who used to be stationed at Assuan.

You remember the build of the beggar?--the impression of scaffolding
his figure conveyed? I always used to think of him as an iron
framework, and he had the most hard-bitten head-piece I have ever
struck; steel blue eyes and a mouth that was born shut. The dash of
ginger in his hair, complexion, and constitution made up a Scotch brew
that was very strongly flavored.

He came down to Cairo one spring, and a lot of us got together in the
club--on a Sunday night, I remember, it was. The conversation got
along that silly line; what we were all doing, and why we were doing
it, what we had really intended to do, and how Fate had butted in and
made sailors of those that had meant to be parsons, engineers of the
poets, and tramps of the chaps who had proposed to become financiers.

Well, we had traveled up and down this blind alley for hours, I should
think, when Dunlap mounted on his hind legs and took the rug with the
proposition that nothing--_nothing_--was impossible of achievement to
the man of single purpose. Someone put up an extreme case; asking
Dunlap how he should handle the business of the son of a respectable
greengrocer who, with singleness of purpose, proposed to become king
of England.

He said it was not a fair case, but he accepted the challenge; and the
way this junior greengrocer, under Dunlap's guidance, plunged into
politics, got elected M.P., wormed himself into the confidence of the
entire Empire by a series of brilliant campaigns conducted from John
o' Groats to Van Diemen's Land; induced the reigning monarch, publicly,
to advocate his own abdication; established a sort of commonwealth with
his ex-Majesty on the board and Dunlap occupying a post between that of
a protector and a Roman Cæsar--well, it was wonderful.

Of course, you can judge of the lateness of the hour from the fact
that a group of moderately intelligent men tolerated, and contributed
to, a chat of this nature. But what brings me down to the story is the
few words which I exchanged with Dunlap at the break-up of the party,
when he was leaving.

His cousin Robert, as you know, is well on the rippity side; but Jack,
with all his fine capacity for heather-dew, had always struck me as
something of a psalmster. I've heard that Bacchus holds the keys of
truth, and it may be right; for out on the steps of the club, I said
to Jack Dunlap:

"It seems you don't practise what you preach?"

"Don't I?" he snapped hardly. "What do you suppose I am doing here?"

"Engineering, I take it. Do you aspire to a pedestal beside De
Lesseps?"

"De Lesseps be damned!" he retorted sourly. "Look at these."

He held out his hands, hardened with manual toil--the hands of a
grinder.

"Clearly you are a glutton for work," I said.

"I am aiming at never doing another hand's stroke in my life," he
replied, with an odd glint in his blue eyes. "My idea of life--_life_,
mind you, not mere existence--is to be a pasha--one of the old school,
with gate porters, orange trees, fountains, slaves, mosaic pavements,
a marble bath."

He mixed his ambitions oddly.

"Someone to do all the shifting for me, and even the thinking; to hold
a book in front of me if I wanted to read, to poke my pipe in my mouth,
and to take it out when I wanted to blow smoke rings--and to _know_
when I wanted it taken out without being told."

"On your showing, you are traveling by the wrong road."

"Am I?" he snapped viciously. "Just wait awhile."

That was all the indication I had of Dunlap's ideas, and remembering
the time of night and other circumstances, I did not count upon it
worth a brass farthing; putting it down to the heather-dew rather than
to any innate viciousness of the man. But listen to the sequel, which
shifts us up just about twelve months, to the spring of the following
year, in fact.


II

I had seen no more of Dunlap, and concluded that he was back in
Assuan, or somewhere on the river, foozling with his irrigation again.
I never had the clearest conception of the work of his department, by
the way. An irrigation man once started to explain to me about his
section, mixing up surveying paraphernalia in his talk, telling me
something about an allowance of half an inch variation in half a mile
of bank, or chat to that effect; but I couldn't quite make it out. My
impression of Dunlap at business was very hazy; I pictured him
measuring the bank of the Nile with a six-foot rule, and periodically
kneeling down in the smelly mud to footle with a spirit-level. But he
was a Senior Wrangler, as you remember, and a man, too, of more
substantial accomplishments, and he drew five hundred a year from the
Egyptian Government; so that probably I underestimated his usefulness.

At any rate, I had forgotten his iron framework and mahogany
countenance, together with his response (under the afflatus of
heather-dew) at the time of which I am now speaking.

A little matter had cropped up which touched me on a weak spot; and
with a mob of jabbering Egyptians and one very placid Bedouin flooding
my room, I found myself thinking again of Dunlap and envying him his
intimate acquaintance with Arabic.

Although I had been in the country quite twice as long as Dunlap, my
Arabic was far from perfect, for I have always been a rotten linguist.
Dunlap, as I now remembered, might have passed for a native (excepting
his Scottish headpiece), and I ascribed his proficiency to an inherent
trick of mimicry. There was something of the big ape about him; and
after one function at which we both were present, I remember how he
convulsed the entire club with an imitation of a certain highly placed
Egyptian dignitary, voice and gesture being equal in comic effect to
Cyril Maude at his best. In fact, if you notice, you will find that
the best linguists, as a rule, have a marked apish streak in their
composition.

Well, here was I at my wits' ends to grasp twenty points of view at
one and the same time; no two expressed in quite the same dialect,
and each orator more excited than another. You know the brutes?

That got me thinking of Dunlap, and even after the incident was
closed, I found myself thinking of him. Some friends from home were
staying at Shepheard's, and of course they had claimed me as dragoman;
not that I objected in the least, for one of the party--when it was
possible to dodge her mother--was, well, a very agreeable companion,
you understand.

On this particular morning we were doing the bazaars. I have found by
comparison that the average tourist knows far more of the Mûski than
the average resident; in the same way, I suppose that for information
regarding the Tower of London or the British Museum, one must go, not
to a Cockney, but to an American visitor. At any rate, my party told
me more than I could tell them, and my job degenerated into that of
a mere interpreter. In the matter of purchases, I possibly saved them
money, but their knowledge of the wares was miles ahead of my own.
These up-to-date guide books must be very useful reading, I think.

Although I had tried hard to rush them past that dangerous quarter,
the _Gôhargîya_, the ladies of the party had discovered a shop where
little trays of loose gems, turquoises, rubies, bits of lapis-lazuli,
and so forth, were displayed snarefully.

After that I knew where I could find them up to any time before lunch;
I knew they were safe enough for the rest of the morning; and accepting
my defeat at the hands of the jewel merchant who turned his slow eyes
upon me and shrugged apologetically, I drifted off, after a decent
interval (leaving young Forrest, who, mysteriously, had turned up, to
do the cavalierly), intending to visit my acquaintance, Hassan, in the
_Sûk el-Attârin_ (Street of the Perfumers), not twenty yards away.

You know Hassan? A large, mysterious figure in the shadows of his
little shop, smoking amber-scented cigarettes as though he liked them,
and turning his sleepy eyes slowly upon each passer-by. Well, I
drifted around in his direction.

Right at the corner of the street, a big limousine was standing; an
up-to-date car, fawn cushions, silver-plated fittings, and simply
stuffed with fresh-cut flowers. A useful-looking Nubian was chauffeur,
and on the step squatted a fat and resplendent being in all the glory
of much gold braid.

These _harêm_ guards are rarely seen in Cairo nowadays--they belong
to the other picturesque Oriental institutions which have begun to
fade with the crescent of Islâm. There was something startlingly
incongruous about this full-grown specimen, that bloated
representative of Eastern despotism squatting on the step of an
up-to-date French car.

It was a kind of all-round shock; I cannot describe how it struck me.
It was something like running into Martin Luther at the Grand National
or Nero, say, at an aviation meeting.

This was a frightfully hot morning, and the adipose object on the car
step was slumbering blissfully. A moment later I spotted the charge
which he was guarding with such sedulous care. She was seated in
Hassan's shop--well back in the shadows--a gauzy white vision, all
eyes and _yashmak_. A confidential female servant accompanied her.
They made a pleasing picture enough, and a more suitable setting could
not well be found. It was an illustrated page of the _Arabian Nights_,
and it appealed strongly even to my jaded perceptions.

Of course, I was not going to interrupt the _tête-à-tête_; but from
where I stood I could observe the group very well whilst remaining
myself unobserved. It presently became evident that the lady of the
_yashmak_, under the pretence of purchasing perfumes, was merely
killing time, and my interest increased as the hour of noon grew near
and the artistic group remained unbroken. You know the Mosque of
El-Ashraf by Hassan's shop? Its minaret almost overhung the place.
Well, in due course, out popped the _mueddin_.

"_La il aha illa Allah...._"

There he was a very sweet-voiced singer, as I noted at the time,
telling them there was no God but God, and all the rest of it; and
presently he worked round to the side of the gallery overlooking
Hassan's shop.

Then I could see which way the wind blew. He seemed to be deliberately
singing _at_ the picturesque trio--and the dark eyes of the lady of
the _yashmak_ were lifted upward--in reverence, perhaps; but I hardly
thought so.

There was no doubt about the _mueddin's_ final glance, as he turned
and retired from the gallery. I remained where I was until the
_yashmak_ left the shop; and as she had to pass quite close to me in
order to rejoin the waiting car, I had a good look at her.

It was just an impression, of course, an impression of red lips under
the white gauze, an oval Oriental outline, with very fine eyes--notably
fine, where fine eyes are common--and a little exquisitely chiseled
nose; a bewitching face. Just that one glimpse I had and a vague
impression of rustling silk with the tap of high heels. A faint breath
of musk still proclaimed itself above the less pleasing odors of the
street; then, the female attendant having cuffed the slumbering Silenus
into wakefulness, the car moved off and this _harêm_ lily vanished from
the bazaar.

I knew that my party was safe for another half an hour, at any rate,
so I nipped along to Hassan's shop. Of course, he began brazenly by
declaring that no ladies had been there that morning. I had expected
it, and the attitude confirmed my suspicions.

Presently, when his boy had made fresh coffee, and Hassan, from the
black cabinet, had produced some real cigarettes, we got more
intimate. There was a scarcity of European visitors that morning; and
excepting one interruption by a party of four American ladies, I had
Hassan to myself for half an hour.

He raised his fat finger to his lips when I pressed my question, and
rolled his eyes fearfully.

"She is from the palace of Harûn Pasha," he whispered with more
sidelong glances. "Ah! _effendim_, I fear...."

We smoked awhile; then--

"The Pasha's wife?" I inquired.

"It is the Lady Zohara," he said.

This did not add greatly to my information; but I continued: "And
the _mueddin_?"

"Ah!--do not whisper it.... That is my brother, Saïd!"

"He raises his eyes very high?"

"Not so, _effendim_; it is she who raises her eyes. I fear--I fear
for Saïd. The Pasha ... you have heard of him?"

"I may have heard his name," I replied; "but I am quite unfamiliar
with his reputation."

Hassan shook his head gloomily.

"He is the last of his race," he explained; "the race of the Khalîfs.
He inhabits the ancient palace--but much has been rebuilt, and much
added--in Old Cairo, close behind the Coptic Church...."

"I did not know that such a palace even existed."

Again Hassan raised his finger to his lips.

"He is not like the other pashas," he said; "in the house of Harûn
Pasha are observed to-day all the old customs as in the day of his
great ancestor Harûn al-Raschîd."

"But a motor-car!"

"Ah, _effendim_, he does not scorn to employ modern comforts, nor do
I mean that he is a strict Moslem. But you saw the one who sat upon
the step? The _harêm_ of the Pasha is well guarded; not only by such
as he, but by the Nubians and by the other mutes."

"Mutes!"

"He has many slaves. His agent in Mecca procures for him the pick of
the market."

"But there is no such thing as slavery in Egypt!"

"Do the slaves know that, _effendim_?" he asked simply. "Those who
have tongues are never seen outside the walls--unless they are guarded
by those who have no tongue!"

It was a curious sidelight upon a more curious possibility and I was
much impressed.

"Your brother----"

"Alas! I have warned him! I fear, most sincerely I fear, that one dark
night the same will befall him that befell the son of my cousin, Ali."

"And what was that?"

"He climbed the wall of the Pasha's garden. There is a fig tree
growing close beside it at one place. Someone assisted him to descend
on the other. But he had been betrayed; the Nubian mutes took him--and
they----"

He bent and whispered in my ear.

"Impossible!" I cried--"impossible! _báss! báss!_"

"Not so, _effendim_--nor was that all. After that they----"

"Enough, Hassan, enough!" I cried. "_Usbûr!_"

Hassan sighed, raising fearful eyes to the minaret.


III

There has been nothing you are likely to disbelieve so far; but
now--well, I specified at the beginning--no comments. Let me tell the
story in my own way, and you have permission to _think_ what you
please.

There was a dance at Shepheard's that night, and young Forrest rather
interfered with my plans again as to one of the members of the English
party; I think I have referred to her before? That sent me home in a
bad humor--at least not home; for as I was standing over by the
Ezbekîyeh Gardens, wondering whether to go along to "Jimmy's" or not,
I formed a sudden determination to go and have a look at the abode of
Harûn Pasha instead!

Mind you, I was not surprised to have lived in Cairo all these years
without having heard of the place; I had learned things about the
Mûski in the morning, from my tourist friends, which had revealed to
me something of my pitiable ignorance. But I was determined to mend my
ways, so to speak, and I thought I would turn my restless mood to good
purpose, by improving my knowledge of my neighbors.

I induced the torpid driver of an _arabîyeh_ to drive me out to Old
Cairo. He obviously considered me to be even more demented than the
rest of my countrymen, but since the fare would be a substantial one,
he tackled the job. Mad expedition? Quite so; but you appreciate the
mood?

After we had passed a certain quarter--a quarter which never
sleeps--there was nothing livelier than decayed tombs _en route_. In
the chill of the evening I began to weigh up my own foolishness
appreciatively, but having got so far as the Coptic Church--you know
the church I mean?--I was not going back unsatisfied; so I told my man
to wait, and started off to look for the famous palace.

I must say the scene was impressive; a sky full of diamonds and a moon
just bursting with light. The liquid night--sounds of the Nile alone
disturbed the silence, and the buildings might have been made of
mother-o'-pearl, so flawless and pure did they seem, gleaming there
under the moon.

Well, I wandered up some narrow streets--past ruins of former
important houses, and all that--until I found myself in the shadow of
a high wall which obviously was kept in good repair. I followed this
for some distance, and I could see trees on the other side; at one
place a perfect mat of those purple flowers hung over the top;
gorgeous things; the name begins with a B, but I can never remember
it. This seemed promising, and as there was not a soul in sight, nor,
on the visible evidences, a habitable building near me, I began to
fossick for a likely place to climb up.

Presently I found the spot, and at the same time confirmation of my
belief that these were the precincts of the Pasha. A fig tree grew
beside the wall, affording an admirable means of reaching the top--a
natural ladder. In a jiffy I was up ... and overlooking one of the
most glorious gardens I had ever seen or dreamt of!

It must have been planned by an artist simply soaked in the lore of
the Orient. It set me thinking of Edmond Dulac's illustrations to the
_Arabian Nights_. Apart from those pages, you never saw anything like
it, I swear. The position of each tree was a study; the arrangement of
the flowerbeds was poetic--that is the only word for it; there was a
pond with marble seats around and a flight of steps with big copper
urns filled with growing flowers, mosaic paths, and lesser pools with
fountains playing. I peered down into the water, and the moon rays
glittered magically upon the scales of the golden carp which darted
there. And all this fairy prospect was no more than an introduction,
as it were, a sort of lead-up, to the Aladdin's Palace beyond.

I saw now that what with palms and the natural rise of the land back
from the Nile, the wonderful palace, with its terraces and gleaming
domes, must actually be invisible from all points; a more secret
locality one could not well imagine.

As to this magician's abode, which lay before me, I shall not attempt
to describe it. But turn to the illustrations which I have mentioned,
or to those of Burton's big edition; I will leave it to the artist's
and your imagination to fill up the canvas.

Lights shone out from a hundred windows. Out of the ghostly, tomb-like
silence of Old Cairo, I had clambered into a sort of fairyland; I
stood there with the spray from a fountain wetting me, and rubbed my
eyes. Honestly, I should not have been surprised to find myself
dreaming. Well, you may be sure I was not going back yet; there was
not a living soul to be seen in the gardens, and I meant to have a
peep into the palace, whatever the chances.

The likeliest point, as I soon determined, was to the west--where a
long, low wing of the building extended, and was lost, if I may use
the term, in a great bank of verdure and purple blooms. I took full
advantage of the ample shadow cast by the trees, and came right up
under the white wall without mishap.

To my right, the wall was obviously modern, but to my left, although
in the distance and under the moon it had seemed uniform, it was built
of sandstone blocks and was evidently of great age. The palace proper,
you understand, was fully forty yards east; the place before me was a
sort of low extension and evidently had no real connection with the
residential part.

Just above my head was a square window, iron-barred, but this did not
look promising, and cautiously, for I was hampered by the creepers
which grew under the wall, I felt my way further west. Presently I
encountered a pointed door of black, time-seared wood, and heavily
iron-studded. Then, with alarming suddenness, the quietude of my
adventure was broken; things began to move with breathless rapidity.

A most dreadful screaming and howling split the stillness and made me
jump like a startled frog!

The sound of a lash on bare flesh reached me from some place behind
the pointed door. Screams for mercy in thick, guttural Arabic, mingled
and punctuated with horrifying shrieks of pain, informed my ignorance
unmistakably that mediæval methods yet ruled in the civilized Near
East.

Screams and supplications merged into a dull moaning; but the whistle
of the lash continued uninterruptedly. Then that too ceased, and dimly
came the sounds of a muffled colloquy; a sort of gurgling talk that
got me wondering.

I had just time to creep away and conceal myself behind a thick clump
of bushes, when the door was thrown open, and the most gigantic negro
I have ever set eyes upon appeared in the opening, outlined against
the smoky glare from within. He had one gleaming bare arm about the
neck of an insensible man, and he dragged him out into the garden as
one might drag a heavy sack; dropping him all in a quivering heap upon
the very spot which I had just vacated!

The negro, who was stripped to the waist and whose glistening body
reminded me of a bronze statue of Hercules, stood looking down at the
insensible victim, with a hideous leer. I ventured to raise myself
ever so slightly; and in the ghastly, sweat-bedaubed face of the
tortured man--whose bare shoulders were bloody from the lash--I
recognized the Silenus of the limousine!

In response to a guttural inarticulate muttering by the black giant, a
second Nubian, of scarcely lesser dimensions, emerged from the dungeon
with a jar of water. He drenched the swooning man, evidently in order
to revive him; and, when the wretched being ultimately fought his way
back to agonized consciousness--to my horror he was seized, dragged in
through the doorway again, and once more I heard the whistle of the
lash being applied to his lacerated back, the skin of which was
already in ribbons.

I suppose there are times when the most discreet man is snatched
outside himself by circumstances? The door of this beastly
torture-room had not been reclosed, and before I could realize what I
was about, I found myself inside!

The wretched victim had been hauled up to a beam by his bound wrists,
and the huge Nubian was putting all his strength into the wielding of
the cat-o'-nine-tails, drawing blood with every stroke; whilst his
assistant hung on to the rope running through a pulley-block in the
low ceiling.

All in a sort of whirl (I was raving mad with indignation) I got
amongst the trio, and landed a clip on the jaw of the son of Erebus
which made his teeth rattle like castanets.

Down came the fat sufferer all in a heap in his own blood. Down went
my man, and began to cough out broken molars. Then it was my turn;
and down _I_ went with the second mute on top of me, and the pair of
us were playing hell all about the blood-spattered floor--up, down,
under, over--straining, punching, kicking ... then my antagonist
introduced gouging, and I had to beat the mat.

It had been a stiff bout, and the stinking shambles were whirling
about me like a bloody maelstrom. When things settled down a bit, I
found myself lying in a small cell skewered up like a pullet, and with
a prospect of iron grating and stone-flagged passage before me. I was
more than a trifle damaged, and my head was singing like a kettle. If
I had thought that I dreamed before, it was a struggle now to convince
myself that this was not a nightmare.

Amid the rattling of chains and dropping of bars, a fantastic
procession was filing down the passage. First came a hideous,
crook-backed apparition, hook-nosed, and bearing a lantern. Behind him
appeared two guards with glittering scimitars. Behind the guards
walked a fourth personage, black-robed and white-turbaned--a sort of
dignified dragoman, carrying an enormous bunch of keys.

The iron grating of my dungeon was unlocked and raised, and I was
requested, in Arabic, to rise and follow. Realizing that this was no
time for funny business, I staggered to my feet, and between the two
Scimitars marched unsteadily through a maze of passages with doors
unlocked and locked behind us, stairs ascended and stairs descended.

From empty passages, our journey led us to passages richly carpeted
and softly lighted. By a heavy door opening on to the first of the
latter, we left the squinting man; and, with the two Scimitars and
Black Robe, I found myself crossing a lofty pavilion.

The floor was of rich mosaic, and priceless carpets were spread about
in artistic confusion. Above my head loomed a great dome, lighted by
stained glass windows in which the blue of lapis-lazuli predominated.
By golden chains from above swung golden lamps burning perfumed oil
and flooding the pavilion with a mellow blue light. There were inlaid
tables and cabinets; great blue vases of exquisite Chinese porcelain
stood in niches of the wall. The walls were of that faintly
amber-tinted alabaster which is quarried in the Mokattam Hills; and
there were fragile columns of some delicately azure-veined marble,
rising, graceful and slender, ethereal as pencils of smoke, to a
balcony high above my head; then, from this, a second series of fairy
columns crept in blue streaks up into the luminous shadows of the
dome.

We crossed this place, my heel taps echoing hollowly and before a
curtained door took pause. An impressive interval of perfumed silence;
then in response to the muffled clapping of hands, the curtain was
raised and I was thrust into a smaller apartment beyond.

I found myself standing before a long _dîwan_, amid an opulence of
Oriental appointment which surpassed anything which I could have
imagined. The atmosphere was heavy with the odor of burning perfumes,
and, whereas the lofty pavilion afforded a delicate study in blue,
this chamber was voluptuously amber--amber-shaded lamps, amber
cushions, amber carpets; everywhere the glitter of amber and gold.

Amid the amber sea, half immersed in the golden silks of the daïs,
reclined a large and portly Sheikh; full and patriarchal his beard,
wherein played amber tints, lofty and serene his brow, sweeping up to
the snowy turban. From a mouthpiece of amber and gold he inhaled the
scented smoke of a _narghli_. Behind him, upon a cushioned stool,
knelt a female whose beauty of face and form was unmistakable, since
it was undisguised by the filmy artistry of her attire. With a
gigantic fan of peacock's feathers, she cooled the Sheikh, and
dispersed the flies which threatened to disturb his serenity. A second
houri received in her hands the amber mouthpiece as it fell from her
lord's lips; a third, who evidently had been playing upon a lute, rose
and glided from the apartment like an opium vision, as I entered
between the guardian Scimitars.

I found myself thinking of Saint Saen's music to _Samson and Delilah_;
the barbaric strains of the exquisite _bacchanale_ were beating on my
brain.

Black Robe advanced and knelt upon the floor of the _dîwan_.

"We have brought the wretched malefactor into your glorious presence,"
he said.

The Pasha (for I knew, beyond doubt, that I stood before Harûn Pasha)
raised his eyes and fixed a stern gaze upon me. He gazed long and
fixedly, and an odd change took place in his expression. He seemed
about to address me, then, apparently changing his mind, he addressed
the recumbent figure at his feet.

"Have the slaves returned with the female miscreant and her partner in
Satan?" he demanded sternly.

"Lord of the age," replied the other, rising upon his knees, "they are
expected."

"Let them be brought before me," directed the Pasha, "upon the instant
of their arrival. Has Misrûn confessed his complicity?"

"He fainted beneath the lash, excellency, but confessed that he
slept--that pig who prayed without washing and whose birth was a
calamity--on several occasions when accompanying the lady Zohara."

"Leave us!" cried the Pasha. "But, first, unbind the prisoner."

He swept his arm around comprehensively, and everyone withdrew from
the apartment, including the Scimitars (one of whom cut my lashings)
and the lady of the fan. I found myself alone with Harûn Pasha.


IV

"Sit here beside me!" directed the Pasha.

Being yet too dazed for wonder or protest, I obeyed mechanically. My
exact situation was not clear to me at the moment and I was a long way
off knowing how to act.

"I am much disturbed in mind, and my bosom is contracted," continued
the Pasha, with a certain benignity, "by reason of a conspiracy in my
_harêm_, which came to a head this night, and which led to the loss of
the pearl of my household, a damsel who cost me her weight in gold,
who entangled me in the snare of her love and pierced me with anguish.
Know, O young Inglîsi, that love is difficult. Alas! she who had
captivated my reason by her loveliness fled with a shame of the
Moslems who defamed the sacred office of _mueddin_! In truth he is
naught but the son of a disease and a consort of camels. My soul
cries out to Allah and my mind is a nest of wasps. Relate to me your
case, that it may turn me from the contemplation of my sorrows. At
another time, it had gone hard with you, and penalties of a most
unfortunate description had been visited upon your head, O disturber
of my peace; but since this child of filth and progeny of mules has
shattered it forever, your lesser crime comes but as a diversion.
Relate to me the matters which have brought you to this miserable
pass."

There was some still little voice in my mind which was trying to speak
to me, if you understand what I mean. But what with the suffocating
perfume of ambergris (or it may have been frankincense), my incredible
surroundings, and the buzzing of my maltreated skull, I simply could
_not_ think connectedly.

A memory was struggling for identification in my addled brain; but
whether it was due to something I had seen, heard, or smelled, I could
not for the life of me make out. I heard myself spinning my own
improbable yarn as one listens to a dreary and boresome recitation;
_I_ didn't seem to be the raconteur; my mind was busy about that amber
room, furiously chasing that hare-like memory, which leaped and
doubled, dived under the silken cushions, popped up behind the Pasha,
and flicked its ears at me from amid the feathers of the peacock fan.

I driveled right on to the end of my story, mechanically, without
having got my mind in proper working order; and when the Pasha spoke
again--there was that wretched memory still dodging me, sometimes
almost within my grasp, but always just eluding it.

"Your amusing narrative has diverted me," said the Pasha; and he
clapped his hands three times.

It never occurred to me, you will note, to assert myself in any way; I
accepted the lordly condescensions of this singular personage without
protest. You will be wondering why I didn't kick up a devil of a
hullabaloo--declare that I had come in response to screams for
assistance--wave the dreaded name of the British Agent under the
Pasha's nose, and all that. I can only say that I didn't; I was
subdued; in fact I was down, utterly down and out.

Black Robe entered with eyes averted.

"Well, wretched vermin!" roared the Pasha in sudden wrath; "do you
tell me they are not here?"

The man, with his head bumping on the carpet, visibly trembled.

"Most noble," he replied hoarsely, "your lowly slave has exerted
himself to the utmost----"

"Out! son of a calamity!" shouted the Pasha--and before my astonished
eyes he raised the heavy _narghli_ and hurled it at the bowed head of
the man before him.

It struck the white turban with a resounding crack, and then was
shattered to bits upon the floor. It was a blow to have staggered a
mule. But Black Robe, without apparent loss of dignity, rose and
departed, bowing.

The Pasha sat rocking about, and plucking madly at his beard.

"O Allah!" he cried, "how I suffer." He turned to me. "Never since the
day that another of your race (but, this one, a true son of Satan)
came to my palace, have I tasted so much suffering. You shall judge
of my clemency, O imprudent stranger, and pacify your heart with the
spectacle of another's punishment."

He clapped his hands twice. This time there was a short delay, which
the Pasha suffered impatiently; then there entered the squint-eyed
man, together with the two Scimitars.

"I would visit the dungeon of the false Pasha," said my singular host;
and, rising to his feet, he placed his hand upon my shoulder and
indicated that we were to proceed from the apartment.

Led by Crook Back, in whose hand the gigantic bunch of keys rattled
unmelodiously, and followed by the Scimitars, we proceeded upon our
way; and it was beyond the powers of my disordered brain to dismiss
the idea that I was taking part in a Christmas pantomime. Many steps
were descended; many heavy doors unbolted and unbarred, bolted and
barred behind us; many stone-paved passages, reminding me of operatic
scenery, were traversed ere we came to one tunnel more gloomy than the
rest.

Upon the right was a blank stone wall, upon the left, a series of
doors, black with age and heavily iron-studded. The only illumination
was that furnished by the lantern which Crook Back carried.

Before one of the doors the Pasha paused.

"In which is Misrûn?" he demanded.

"In the next, excellency," replied the jailer--for such I took to be
the office of the hunchback.

As he spoke, he held the lantern to the grating.

I found myself peering into a filthy dungeon, the reek of which made
me ill; and there, upon the stone floor, lay poor Silenus! He raised
his eyes to the light.

"Lord of the age," he moaned, lifting his manacled wrists, "glory of
the universe, sun of suns! I have confessed my frightful sin, and most
dire misfortunes. Of your sublime mercy, take pity upon the meanest
thing that creeps upon the earth----"

"Proceed!" said the Pasha.

And with the moaning cries of Misrûn growing fainter behind us, we
moved along the passage. Before a second door, we halted again, and
the jailer raised the lantern.

"Look upon this!" cried the Pasha to me--"look well, and look long!"

Shudderingly I peered in between the bars. It had come home to me how
I was utterly at the mercy of this man's moods. If he had chosen to
have me hurled into one of his dungeons, what prospect of release
would have been mine? Who would ever know of my plight? No one! And
beyond doubt I was in the realm of an absolute monarch. I silently
thanked my lucky stars that my lot was not the lot of him who occupied
this second dungeon.

As the dim light, casting shadow bars across the filthy floor, picked
out the features of the prisoner, I gave a great start. Save that the
beard was more gray, longer, filthy and unkempt, and that, in place
of the nearly shaven skull, this unhappy being displayed dishevelled
locks, the captive might easily have passed for the Pasha.

I met the eye of this terrible despot.

"Look upon the false Pasha," he said; "look upon the one who thought
to dispossess me! For years, by his own miserable confession, he
studied me in secret. When I journeyed to my estates in Assuan" (I
started again) "he was watching--watching--always watching. His
scheme, which was whispered into his ear by the Evil One, was no plant
of sudden growth, but a tree, that, from a seed of Satan planted in
fertile soil, had flourished exceedingly, tended by the hand of
villainous ambition."

I clutched at the bars for support. The stench of the place was simply
indescribable; but it was neither the stench nor the bizarre incidents
of the night which accounted for my dizziness: it was the sudden
tangibility of that hitherto elusive memory.

In build, in complexion, in certain mannerisms underlying the
dignified assumption, Harûn Pasha might well have been the twin
brother of Jack Dunlap!

A frightful possibility burst upon me like a bomb; clutching the bars
with quivering hands, I stared and stared at the wretched impostor in
the cell. _Could_ it be? Had he been mad enough to make some attempt
upon the Pasha? And was this his end?

I looked around again. I searched the bearded features of the Pasha
with eager gaze. Good God! either I was going mad, or incredible
things had been done, were being done, in Cairo.

I had not seen Dunlap for a year, remember, and in the ordinary way
I did not see him more than half a dozen times in twelve months, so
that, all things considered, it was not so remarkable that I had
overlooked the resemblance. A full beard and mustache, artificially
darkened eyelashes, a shaven head and a white turban, are effectual
disguises; but if you can imagine Dunlap--the Dunlap you remember--so
arrayed, then you have Harûn Pasha. Imagine Harûn Pasha, dirty,
bedraggled, a hopeless captive ... and you have the prisoner who
crouched upon the straw in that noisome dungeon!

For the second time that night I was lifted out of myself. I turned
on the man beside me in a blazing fury.

"You villain!" I shouted at him, and clenched my fists--"do you _dare_
to confine a Britisher in your stinking cellars. By God! sir...."

Harûn Pasha clapped his hand over my mouth; the two guards had me by
the arms from behind. But my cries had aroused the man in the dungeon,
and, as I was dragged down the passage, these moaning words reached
me, spoken in Arabic:

"Help! help! Englishman! A crime has been committed! I appeal to
Lord----."

A door was slammed fast with a resounding bang, and the rest of the
captive's appeal was lost to me. One of my guards had substituted his
hand for that of the Pasha, but now it was removed; and, speechless
with rage, I found myself being thrust up stone stairs--and I realized
that by a moment's indiscretion, I had ruined everything.

Back in the amber apartment once more, with the two Scimitars at the
door and Harûn Pasha reclining upon the cushions, I found speech.

"What are you going to do with me?" I demanded.

"My son," replied the Pasha with benignity, "I pardon all! Your great
courage and address, together with the modesty of your deportment, and
the spirit of adventure which has brought you to your present
unfortunate case, plead for you in a manner which my clemency cannot
resist. It is my unhappy lot often to be called upon to punish.
To-night, those gloomy dungeons which you have seen will echo, alas,
with the howls of miserable wretches who are responsible for the loss
of the pearl of my soul; for I am persuaded that she has fled with the
son of offal who profaned the words of Allah from the minaret. This
being so, I would temper my proper severity with a merciful deed. You
shall never speak of what you have seen within these walls, save in
terms suitably disguised. You shall never seek to return, nor, by
speech with any man, to confirm whatsoever you may suspect. Upon this
warranty, you shall depart in peace."

He clapped his hands twice, and a houri of most bewitching aspect
glided into the _dîwan_.

"Bring sherbet!" ordered the Pasha.

The maiden departed; and whilst I was yet trying to come to a
decision (the Pasha had mentioned no alternative, but my imagination
was equal to the task of supplying one!) she returned with a tray upon
which were porcelain cups and two vessels of beautifully chased gold.

Harûn Pasha decocted a sparkling beverage, and, with his own hands,
passed the brimming cup to me.

       *       *       *       *       *

I knew you would not believe it; but I warned you, and I made a
stipulation. Your idea is that I must be a poor sort of animal to
accept so dishonorable a compromise? I agree. But the situation was
even more peculiarly difficult than is apparent to you at the moment.
Without _seeking_ the information, I learned from Hassan of the Scent
Bazaar that his brother had indeed fled with the beauteous Lady
Zohara, no one knew whither; and this confirmation of the Pasha's
sorrows touched a very tender spot in my heart!

Then there is another little point.

When the Pasha removed the elaborate stopper from the first of the
golden vessels to which I have just referred, _my_ eye alone perceived
that a bottle, bearing a familiar black and white label, was contained
in this golden casing. The flavor of the decoction with which we
sealed our infamous bargain clinched the matter.

I was absolutely thrust out of the presence chamber before I had time
for another word; but, looking back from the door and meeting the eye
of the Pasha, I encountered a most portentous wink. Therefore I have
stuck to my bargain.

Oh! I have not given much away. The Pasha is not called Harûn, and the
palace is nowhere near the Coptic Church in Old Cairo. Because, you
see, I only knew one man who winked in quite that elaborate
fashion--and his name was Jack Dunlap!




V

IN THE VALLEY OF THE SORCERESS


I

Condor wrote to me three times before the end (said Neville,
Assistant-Inspector of Antiquities, staring vaguely from his open
window at a squad drilling before the Kasr-en-Nîl Barracks). He dated
his letters from the camp at Deir-el-Bahari. Judging from these,
success appeared to be almost within his grasp. He shared my theories,
of course, respecting Queen Hatasu, and was devoting the whole of his
energies to the task of clearing up the great mystery of Ancient Egypt
which centres around that queen.

For him, as for me, there was a strange fascination about those
defaced walls and roughly obliterated inscriptions. That the queen
under whom Egyptian art came to the apogee of perfection should thus
have been treated by her successors; that no perfect figure of the
wise, famous, and beautiful Hatasu should have been spared to
posterity; that her very cartouche should have been ruthlessly removed
from every inscription upon which it appeared, presented to Condor's
mind a problem only second in interest to the immortal riddle of
Gîzeh.

You know my own views upon the matter? My monograph, "Hatasu, the
Sorceress," embodies my opinion. In short, upon certain evidences,
some adduced by Theodore Davis, some by poor Condor, and some
resulting from my own inquiries, I have come to the conclusion that
the source--real or imaginary--of this queen's power was an intimate
acquaintance with what nowadays we term, vaguely, magic. Pursuing her
studies beyond the limit which is lawful, she met with a certain end,
not uncommon, if the old writings are to be believed, in the case of
those who penetrate too far into the realms of the Borderland.

For this reason--the practice of black magic--her statues were
dishonored, and her name erased from the monuments. Now, I do not
propose to enter into any discussion respecting the reality of such
practices; in my monograph I have merely endeavored to show that,
according to contemporary belief, the queen was a sorceress. Condor
was seeking to prove the same thing; and when I took up the inquiry,
it was in the hope of completing his interrupted work.

He wrote to me early in the winter of 1908, from his camp by the Rock
Temple. Davis's tomb, at Bibân el-Mulûk, with its long, narrow
passage, apparently had little interest for him; he was at work on the
high ground behind the temple, at a point one hundred yards or so due
west of the upper platform. He had an idea that he should find there
the mummies of Hatasu--and another; the latter, a certain Sen-Mût, who
appears in the inscriptions of the reign as an architect high in the
queen's favor. The archæological points of the letter do not concern
us in the least, but there was one odd little paragraph which I had
cause to remember afterwards.

"A girl belonging to some Arab tribe," wrote Condor, "came racing to
the camp two nights ago to claim my protection. What crime she had
committed, and what punishment she feared, were far from clear; but
she clung to me, trembling like a leaf, and positively refused to
depart. It was a difficult situation, for a camp of fifty native
excavators, and one highly respectable European enthusiast, affords no
suitable quarters for an Arab girl--and a very personable Arab girl.
At any rate, she is still here; I have had a sort of lean-to rigged up
in a little valley east of my own tent, but it is very embarrassing."

Nearly a month passed before I heard from Condor again; then came a
second letter, with the news that on the eve of a great discovery--as
he believed--his entire native staff--the whole fifty--had deserted
one night in a body! "Two days' work," he wrote, "would have seen the
tomb opened--for I am more than ever certain that my plans are
accurate. Then I woke up one morning to find every man Jack of my
fellows missing! I went down into the village where a lot of them
live, in a towering rage, but not one of the brutes was to be found,
and their relations professed entire ignorance respecting their
whereabouts. What caused me almost as much anxiety as the check in my
work was the fact that Mahâra--the Arab girl--had vanished also. I am
wondering if the thing has any sinister significance."

Condor finished with the statement that he was making tremendous
efforts to secure a new gang. "But," said he, "I shall finish the
excavation, if I have to do it with my own hands."

His third and last letter contained even stranger matters than the two
preceding it. He had succeeded in borrowing a few men from the British
Archæological camp in the Fáyûm. Then, just as the work was
restarting, the Arab girl, Mahâra, turned up again, and entreated him
to bring her down the Nile, "at least as far as Dendera. For the
vengeance of her tribesmen," stated Condor, "otherwise would result
not only in her own death, but in mine! At the moment of writing I am
in two minds what to do. If Mahâra is to go upon this journey, I do
not feel justified in sending her alone, and there is no one here who
could perform the duty," etc.

I began to wonder, of course; and I had it in mind to take the train
to Luxor merely in order to see this Arab maiden, who seemed to occupy
so prominent a place in Condor's mind. However, Fate would have it
otherwise; and the next thing I heard was that Condor had been brought
into Cairo, and was at the English hospital.

He had been bitten by a cat--presumably from the neighboring village;
and although the doctor at Luxor dealt with the bite at once, traveled
down with him, and placed him in the hand of the Pasteur man at the
hospital, he died, as you remember, in the night of his arrival,
raving mad; the Pasteur treatment failed entirely.

I never saw him before the end, but they told me that his howls were
horribly like those of a cat. His eyes changed in some way, too, I
understand; and, with his fingers all contracted, he tried to
_scratch_ everyone and everything within reach.

They had to strap the poor beggar down, and even then he tore the
sheets into ribbons.

Well, as soon as possible, I made the necessary arrangements to finish
Condor's inquiry. I had access to his papers, plans, etc., and in the
spring of the same year I took up my quarters near Deir-el-Bahari,
roped off the approaches to the camp, stuck up the usual notices, and
prepared to finish the excavation, which, I gathered, was in a fairly
advanced state.

My first surprise came very soon after my arrival, for when, with the
plan before me, I started out to find the shaft, I found it,
certainly, but only with great difficulty.

It had been filled in again with sand and loose rock right to the very
top!


II

All my inquiries availed me nothing. With what object the excavation
had been thus closed I was unable to conjecture. That Condor had not
reclosed it I was quite certain, for at the time of his mishap he had
actually been at work at the bottom of the shaft, as inquiries from a
native of Suefee, in the Fáyûm, who was his only companion at the
time, had revealed.

In his eagerness to complete the inquiry, Condor, by lantern light,
had been engaged upon a solitary night-shift below, and the rabid cat
had apparently fallen into the pit; probably in a frenzy of fear, it
had attacked Condor, after which it had escaped.

Only this one man was with him, and he, for some reason that I could
not make out, had apparently been sleeping in the temple--quite a
considerable distance from Condor's camp. The poor fellow's cries had
aroused him, and he had met Condor running down the path and away from
the shaft.

This, however, was good evidence of the existence of the shaft at the
time, and as I stood contemplating the tightly packed rubble which
alone marked its site, I grew more and more mystified, for this task
of reclosing the cutting represented much hard labor.

Beyond perfecting my plans in one or two particulars, I did little on
the day of my arrival. I had only a handful of men with me, all of
whom I knew, having worked with them before, and beyond clearing
Condor's shaft I did not intend to excavate further.

Hatasu's Temple presents a lively enough scene in the daytime during
the winter and early spring months, with the streams of tourists
constantly passing from the white causeway to Cook's Rest House on the
edge of the desert. There had been a goodly number of visitors that day
to the temple below, and one or two of the more curious and
venturesome had scrambled up the steep path to the little plateau
which was the scene of my operations. None had penetrated beyond the
notice boards, however, and now, with the evening sky passing through
those innumerable shades which defy palette and brush, which can only
be distinguished by the trained eye, but which, from palest blue melt
into exquisite pink, and by some magical combination form that deep
violet which does not exist to perfection elsewhere than in the skies
of Egypt, I found myself in the silence and the solitude of "the Holy
Valley."

I stood at the edge of the plateau, looking out at the rosy belt which
marked the course of the distant Nile, with the Arabian hills vaguely
sketched beyond. The rocks stood up against that prospect as great
black smudges, and what I could see of the causeway looked like a gray
smear upon a drab canvas. Beneath me were the chambers of the Rock
Temple, with those wall paintings depicting events in the reign of
Hatasu which rank among the wonders of Egypt.

Not a sound disturbed my reverie, save a faint clatter of cooking
utensils from the camp behind me--a desecration of that sacred
solitude. Then a dog began to howl in the neighboring village.
The dog ceased, and faintly to my ears came the note of a reed
pipe. The breeze died away, and with it the piping.

I turned back to the camp, and, having partaken of a frugal supper,
turned in upon my campaigner's bed, thoroughly enjoying my freedom
from the routine of official life in Cairo, and looking forward to
the morrow's work pleasurably.

Under such circumstances a man sleeps well; and when, in an uncanny
gray half-light, which probably heralded the dawn, I awoke with a
start, I knew that something of an unusual nature alone could have
disturbed my slumbers.

Firstly, then, I identified this with a concerted howling of the
village dogs. They seemed to have conspired to make night hideous;
I have never heard such an eerie din in my life. Then it gradually
began to die away, and I realized, secondly, that the howling of the
dogs and my own awakening might be due to some common cause. This
idea grew upon me, and as the howling subsided, a sort of disquiet
possessed me, and, despite my efforts to shake it off, grew more
urgent with the passing of every moment.

In short, I fancied that the thing which had alarmed or enraged the
dogs was passing from the village through the Holy Valley, upward
to the Temple, upward to the plateau, and was approaching _me_.

I have never experienced an identical sensation since, but I seemed
to be audient of a sort of psychic patrol, which, from a remote
_pianissimo_, swelled _fortissimo_, to an intimate but silent clamor,
which beat in some way upon my brain, but not through the faculty of
hearing, for now the night was deathly still.

Yet I was persuaded of some _approach_--of the coming of something
sinister, and the suspense of waiting had become almost insupportable,
so that I began to accuse my Spartan supper of having given me
nightmare, when the tent-flap was suddenly raised, and, outlined
against the paling blue of the sky, with a sort of reflected elfin
light playing upon her face, I saw an Arab girl looking in at me!

By dint of exerting all my self-control I managed to restrain the cry
and upward start which this apparition prompted. Quite still, with my
fists tightly clenched, I lay and looked into the eyes which were
looking into mine.

The style of literary work which it has been my lot to cultivate fails
me in describing that beautiful and evil face. The features were
severely classical and small, something of the Bisharîn type, with a
cruel little mouth and a rounded chin, firm to hardness. In the eyes
alone lay the languor of the Orient; they were exceedingly--indeed,
excessively--long and narrow. The ordinary ragged, picturesque finery
of a desert girl bedecked this midnight visitant, who, motionless,
stood there watching me.

I once read a work by Pierre de l'Ancre, dealing with the Black
Sabbaths of the Middle Ages, and now the evil beauty of this Arab face
threw my memory back to those singular pages, for, perhaps owing to
the reflected light which I have mentioned, although the explanation
scarcely seemed adequate, those long, narrow eyes shone catlike in the
gloom.

Suddenly I made up my mind. Throwing the blanket from me, I leapt to
the ground, and in a flash had gripped the girl by the wrists.
Confuting some lingering doubts, she proved to be substantial enough.
My electric torch lay upon a box at the foot of the bed, and,
stooping, I caught it up and turned its searching rays upon the face
of my captive.

She fell back from me, panting like a wild creature trapped, then
dropped upon her knees and began to plead--began to plead in a voice
and with a manner which touched some chord of consciousness that I
could swear had never spoken before, and has never spoken since.

She spoke in Arabic, of course, but the words fell from her lips as
liquid music in which lay all the beauty and all the deviltry of the
"Siren's Song." Fully opening her astonishing eyes, she looked up at
me, and, with her free hand pressed to her bosom, told me how she had
fled from an unwelcome marriage; how, an outcast and a pariah, she
had hidden in the desert places for three days and three nights,
sustaining life only by means of a few dates which she had brought
with her, and quenching her thirst with stolen water-melons.

"I can bear it no longer, _effendim_. Another night out in the desert,
with the cruel moon beating, beating, beating upon my brain, with
creeping things coming out from the rocks, wriggling, wriggling, their
many feet making whisperings in the sand--ah, it will kill me! And I
am for ever outcast from my tribe, from my people. No tent of all the
Arabs, though I fly to the gates of Damascus, is open to me, save I
enter in shame, as a slave, as a plaything, as a toy. My
heart"--furiously she beat upon her breast--"is empty and desolate,
_effendim_. I am meaner than the lowliest thing that creeps upon the
sand; yet the God that made that creeping thing made me also--and you,
you, who are merciful and strong, would not crush any creature because
it was weak and helpless."

I had released her wrist now, and was looking down at her in a sort
of stupor. The evil which at first I had seemed to perceive in her was
effaced, wiped out as an artist wipes out an error in his drawing. Her
dark beauty was speaking to me in a language of its own; a strange
language, yet one so intelligible that I struggled in vain to
disregard it. And her voice, her gestures, and the witch-fire of her
eyes were whipping up my blood to a fever heat of passionate
sorrow--of despair. Yes, incredible as it sounds, despair!

In short, as I see it now, this siren of the wilderness was playing
upon me as an accomplished musician might play upon a harp, striking
this string and that at will, and sounding each with such full notes
as they had rarely, if ever, emitted before.

Most damnable anomaly of all, I--Edward Neville, archæologist, most
prosy and matter-of-fact man in Cairo, perhaps--_knew_ that this nomad
who had burst into my tent, upon whom I had set eyes for the first
time scarce three minutes before, held me enthralled; and yet, with
her wondrous eyes upon me, I could summon up no resentment, and could
offer but poor resistance.

"In the Little Oasis, _effendim_, I have a sister who will admit me
into her household, if only as a servant. There I can be safe, there
I can rest. O _Inglîsi_, at home in England you have a sister of your
own! Would you see her pursued, a hunted thing from rock to rock,
crouching for shelter in the lair of some jackal, stealing that she
might live--and flying always, never resting, her heart leaping for
fear, flying, flying, with nothing but dishonor before her?"

She shuddered and clasped my left hand in both her own convulsively,
pulling it down to her bosom.

"There can be only one thing, _effendim_," she whispered. "Do you not
see the white bones bleaching in the sun?"

Throwing all my resolution into the act, I released my hand from her
clasp, and, turning aside, sat down upon the box which served me as
chair and table, too.

A thought had come to my assistance, had strengthened me in the moment
of my greatest weakness; it was the thought of that Arab girl
mentioned in Condor's letters. And a scheme of things, an incredible
scheme, that embraced and explained some, if not all, of the horrible
circumstances attendant upon his death, began to form in my brain.

Bizarre it was, stretching out beyond the realm of things natural and
proper, yet I clung to it, for there, in the solitude, with this
wildly beautiful creature kneeling at my feet, and with her uncanny
powers of fascination yet enveloping me like a cloak, I found it not
so improbable as inevitably it must have seemed at another time.

I turned my head, and through the gloom sought to look into the long
eyes. As I did so they closed and appeared as two darkly luminous
slits in the perfect oval of the face.

"You are an impostor!" I said in Arabic, speaking firmly and
deliberately. "To Mr. Condor"--I could have sworn that she started
slightly at sound of the name--"you called yourself Mahâra. I know
you, and I will have nothing to do with you."

But in saying it I had to turn my head aside, for the strangest,
maddest impulses were bubbling up in my brain in response to the
glances of those half-shut eyes.

I reached for my coat, which lay upon the foot of the bed, and, taking
out some loose money, I placed fifty piastres in the nerveless brown
hand.

"That will enable you to reach the Little Oasis, if such is your
desire," I said. "It is all I can do for you, and now--you must go."

The light of the dawn was growing stronger momentarily, so that I
could see my visitor quite clearly. She rose to her feet, and stood
before me, a straight, slim figure, sweeping me from head to foot with
such a glance of passionate contempt as I had never known or suffered.

She threw back her head magnificently, dashed the money on the ground
at my feet, and, turning, leaped out of the tent.

For a moment I hesitated, doubting, questioning my humanity, testing
my fears; then I took a step forward, and peered out across the
plateau. Not a soul was in sight. The rocks stood up gray and eerie,
and beneath lay the carpet of the desert stretching unbroken to the
shadows of the Nile Valley.


III

We commenced the work of clearing the shaft at an early hour that
morning. The strangest ideas were now playing in my mind, and in some
way I felt myself to be in opposition to definite enmity. My
excavators labored with a will, and, once we had penetrated below the
first three feet or so of tightly packed stone, it became a mere
matter of shoveling, for apparently the lower part of the shaft had
been filled up principally with sand.

I calculated that four days' work at the outside would see the shaft
clear to the base of Condor's excavation. There remained, according
to his own notes, only another six feet or so; but it was solid
limestone--the roof of the passage, if his plans were correct,
communicating with the tomb of Hatasu.

With the approach of night, tired as I was, I felt little inclination
for sleep. I lay down on my bed with a small Browning pistol under the
pillow, but after an hour or so of nervous listening drifted off into
slumber. As on the night before, I awoke shortly before the coming of
dawn.

Again the village dogs were raising a hideous outcry, and again I was
keenly conscious of some ever-nearing menace. This consciousness grew
stronger as the howling of the dogs grew fainter, and the sense of
_approach_ assailed me as on the previous occasion.

I sat up immediately with the pistol in my hand, and, gently raising
the tent flap, looked out over the darksome plateau. For a long time
I could perceive nothing; then, vaguely outlined against the sky, I
detected something that moved above the rocky edge.

It was so indefinite in form that for a time I was unable to identify
it, but as it slowly rose higher and higher, two luminous
eyes--obviously feline eyes, since they glittered greenly in the
darkness--came into view. In character and in shape they were the eyes
of a cat, but in point of size they were larger than the eyes of any
cat I had ever seen. Nor were they jackal eyes. It occurred to me that
some predatory beast from the Sûdan might conceivably have strayed
thus far north.

The presence of such a creature would account for the nightly
disturbance amongst the village dogs; and, dismissing the
superstitious notions which had led me to associate the mysterious
Arab girl with the phenomenon of the howling dogs, I seized upon this
new idea with a sort of gladness.

Stepping boldly out of the tent, I strode in the direction of the
gleaming eyes. Although my only weapon was the Browning pistol, it was
a weapon of considerable power, and, moreover, I counted upon the
well-known cowardice of nocturnal animals. I was not disappointed in
the result.

The eyes dropped out of sight, and as I leaped to the edge of rock
overhanging the temple a lithe shape went streaking off in the
greyness beneath me. Its coloring appeared to be black, but this
appearance may have been due to the bad light. Certainly it was no
cat, was no jackal; and once, twice, thrice my Browning spat into the
darkness.

Apparently I had not scored a hit, but the loud reports of the weapon
aroused the men sleeping in the camp, and soon I was surrounded by a
ring of inquiring faces.

But there I stood on the rock-edge, looking out across the desert in
silence. Something in the long, luminous eyes, something in the
sinuous, flying shape had spoken to me intimately, horribly.

Hassan es-Sugra, the headman, touched my arm, and I knew that I must
offer some explanation.

"Jackals," I said shortly. And with no other word I walked back to my
tent.

The night passed without further event, and in the morning we
addressed ourselves to the work with such a will that I saw, to my
satisfaction, that by noon of the following day the labor of clearing
the loose sand would be completed.

During the preparation of the evening meal I became aware of a certain
disquiet in the camp, and I noted a disinclination on the part of the
native laborers to stray far from the tents. They hung together in a
group, and whilst individually they seemed to avoid meeting my eye,
collectively they watched me in a furtive fashion.

A gang of Moslem workmen calls for delicate handling, and I wondered
if, inadvertently, I had transgressed in some way their iron-bound
code of conduct. I called Hassan es-Sugra aside.

"What ails the men?" I asked him. "Have they some grievance?"

Hassan spread his palms eloquently.

"If they have," he replied, "they are secret about it, and I am not in
their confidence. Shall I thrash three or four of them in order to
learn the nature of this grievance?"

"No thanks all the same," I said, laughing at this characteristic
proposal. "If they refuse to work to-morrow, there will be time enough
for you to adopt those measures."

On this, the third night of my sojourn in the Holy Valley by the
Temple of Hatasu, I slept soundly and uninterruptedly. I had been
looking forward with the keenest zest to the morrow's work, which
promised to bring me within sight of my goal, and when Hassan came to
awaken me, I leaped out of bed immediately.

Hassan es-Sugra, having performed his duty, did not, as was his
custom, retire; he stood there, a tall, angular figure, looking at me
strangely.

"Well?" I said.

"There is trouble," was his simple reply. "Follow me, Neville
Effendi."

Wondering greatly, I followed him across the plateau and down the
slope to the excavation. There I pulled up short with a cry of
amazement.

Condor's shaft was filled in to the very top, and presented, to my
astonished gaze, much the same aspect that had greeted me upon my
first arrival!

"The men----" I began.

Hassan es-Sugra spread wide his palms.

"Gone!" he replied. "Those Coptic dogs, those eaters of carrion, have
fled in the night."

"And this"--I pointed to the little mound of broken granite and
sand--"is their work?"

"So it would seem," was the reply; and Hassan sniffed his sublime
contempt.

I stood looking bitterly at this destruction of my toils. The
strangeness of the thing at the moment did not strike me, in my anger;
I was only concerned with the outrageous impudence of the missing
workmen, and if I could have laid hands upon one of them it had surely
gone hard with him.

As for Hassan es-Sugra, I believe he would cheerfully have broken the
necks of the entire gang. But he was a man of resource.

"It is so newly filled in," he said, "that you and I, in three days,
or in four, can restore it to the state it had reached when those
nameless dogs, who regularly prayed with their shoes on, those
devourers of pork, began their dirty work."

His example was stimulating. _I_ was not going to be beaten, either.

After a hasty breakfast, the pair of us set to work with pick and
shovel and basket. We worked as those slaves must have worked whose
toil was directed by the lash of the Pharaoh's overseer. My back
acquired an almost permanent crook, and every muscle in my body seemed
to be on fire. Not even in the midday heat did we slacken or stay our
toils; and when dusk fell that night a great mound had arisen beside
Condor's shaft, and we had excavated to a depth it had taken our gang
double the time to reach.

When at last we threw down our tools in utter exhaustion, I held out
my hand to Hassan, and wrung his brown fist enthusiastically. His eyes
sparkled as he met my glance.

"Neville Effendi," he said, "you are a true Moslem!"

And only the initiated can know how high was the compliment conveyed.

That night I slept the sleep of utter weariness, yet it was not a
dreamless sleep, or perhaps it was not so deep as I supposed, for
blazing cat-eyes encircled me in my dreams, and a constant feline
howling seemed to fill the night.

When I awoke the sun was blazing down upon the rock outside my tent,
and, springing out of bed, I perceived, with amazement, that the
morning was far advanced. Indeed, I could hear the distant voices of
the donkey-boys and other harbingers of the coming tourists.

Why had Hassan es-Sugra not awakened me?

I stepped out of the tent and called him in a loud voice. There was
no reply. I ran across the plateau to the edge of the hollow.

Condor's shaft had been reclosed to the top!

Language fails me to convey the wave of anger, amazement, incredulity,
which swept over me. I looked across to the deserted camp and back to
my own tent; I looked down at the mound, where but a few hours before
had been a pit, and seriously I began to question whether I was mad or
whether madness had seized upon all who had been with me. Then, pegged
down upon the heap of broken stones, I perceived, fluttering, a small
piece of paper.

Dully I walked across and picked it up. Hassan, a man of some
education, clearly was the writer. It was a pencil scrawl in doubtful
Arabic, and, not without difficulty, I deciphered it as follows:

"Fly, Neville Effendi! This is a haunted place!"

Standing there by the mound, I tore the scrap of paper into minute
fragments, bitterly casting them from me upon the ground. It was
incredible; it was insane.

The man who had written that absurd message, the man who had undone
his own work, had the reputation of being fearless and honorable. He
had been with me before a score of times, and had quelled petty
mutinies in the camp in a manner which marked him a born overseer. I
could not understand; I could scarcely believe the evidence of my own
senses.

What did I do?

I suppose there are some who would have abandoned the thing at once
and for always, but I take it that the national traits are strong
within me. I went over to the camp and prepared my own breakfast;
then, shouldering pick and shovel, I went down into the valley and
set to work. What ten men could not do, what two men had failed to do,
one man was determined to do.

It was about half an hour after commencing my toils, and when, I
suppose, the surprise and rage occasioned by the discovery had begun
to wear off, that I found myself making comparisons between my own
case and that of Condor. It became more and more evident to me that
events--mysterious events--were repeating themselves.

The frightful happenings attendant upon Condor's death were marshaling
in my mind. The sun was blazing down upon me, and distant voices could
be heard in the desert stillness. I knew that the plain below was
dotted with pleasure-seeking tourists, yet nervous tremors shook me.
Frankly, I dreaded the coming of the night.

Well, tenacity or pugnacity conquered, and I worked on until dusk. My
supper despatched, I sat down on my bed and toyed with the Browning.

I realized already that sleep, under existing conditions, was
impossible. I perceived that on the morrow I must abandon my one-man
enterprise, pocket my pride, in a sense, and seek new assistants, new
companions.

The fact was coming home to me conclusively that a menace, real and
not mythical, hung over that valley. Although, in the morning sunlight
and filled with indignation, I had thought contemptuously of Hassan
es-Sugra, now, in the mysterious violet dusk so conducive to calm
consideration, I was forced to admit that he was at least as brave a
man as I. And he had fled! What did that night hold in keeping for me?

       *       *       *       *       *

I will tell you what occurred, and it is the only explanation I have
to give of why Condor's shaft, said to communicate with the real tomb
of Hatasu, to this day remains unopened.

There, on the edge of my bed, I sat far into the night, not daring to
close my eyes. But physical weariness conquered in the end, and,
although I have no recollection of its coming, I must have succumbed
to sleep, since I remember--can never forget--a repetition of the
dream, or what I had assumed to be a dream, of the night before.

A ring of blazing green eyes surrounded me. At one point this ring was
broken, and in a kind of nightmare panic I leaped at that promise of
safety, and found myself outside the tent.

Lithe, slinking shapes hemmed me in--cat shapes, ghoul shapes,
veritable figures of the pit. And the eyes, the shapes, although they
were the eyes and shapes of cats, sometimes changed elusively, and
became the wicked eyes and the sinuous, writhing shapes of women.
Always the ring was incomplete, and always I retreated in the only
direction by which retreat was possible. I retreated from those
cat-things.

In this fashion I came at last to the shaft, and there I saw the tools
which I had left at the end of my day's toil.

Looking around me, I saw also, with such a pang of horror as I cannot
hope to convey to you, that the ring of green eyes was now unbroken
about me.

And it was closing in.

Nameless feline creatures were crowding silently to the edge of the
pit, some preparing to spring down upon me where I stood. A voice
seemed to speak in my brain; it spoke of capitulation, telling me to
accept defeat, lest, resisting, my fate be the fate of Condor.

Peals of shrill laughter rose upon the silence. The laughter was mine.

Filling the night with this hideous, hysterical merriment, I was
working feverishly with pick and with shovel filling in the shaft.

The end? The end is that I awoke, in the morning, lying, not on my
bed, but outside on the plateau, my hands torn and bleeding and every
muscle in my body throbbing agonisingly. Remembering my dream--for
even in that moment of awakening I thought I had dreamed--I staggered
across to the valley of the excavation.

Condor's shaft was reclosed to the top.




VI

POMEGRANATE FLOWER


I

There are not so many _Antereeyeh_ (story-tellers) in Cairo now (said
my acquaintance, Hassan of the Scent Bazaar, staring, reflectively, at
two American ladies paying fabulous prices for the goods of his
mendacious neighbor on the left). They have adopted other, and more
lucrative, professions; but in my father's time, it was an excellent
business.

For one thing, the stories which you call the _Arabian Nights_ are
no longer recited, because they are said to be unlucky. This has
considerably reduced the story-teller's stock-in-trade; for unless
a man is blessed with much originality, he cannot well refrain from
using in his narratives some part of the thousand and one tales.

To this day, however, there is in the city of Cairo a tale-teller of
much repute. With his tale-telling he combines the profession of a
barber; and like the famous barber of the _Arabian Nights_ bears the
nickname Es-Samit (the Silent). An old man is this Es-Samit, who no
more will know his ninetieth year, of dark countenance, and white
beard and eyebrows, with small ears like the ears of a gazelle, and
a long nose like that of a camel, and a haughty aspect. This barber
enjoys every comfort in his declining years by reason of his amusing
manner, and because his ridiculous stories and disclosures respecting
his six brothers (for in all things he resembles, or claims to
resemble, his famous namesake) divert all who hear them, causing him
whose bosom is contracted with woe to swoon with excessive laughter,
and filling the saddest heart with joy; such is the absurd loquacity
and impertinence of the barber called Es-Samit, the Silent.

It chanced one day that I found myself at the wedding festivities of a
prosperous merchant distantly related to me; and for the entertainment
of his guests, this wealthy man, in addition to the usual dances and
songs, had engaged Es-Samit to divert us with one of his untruthful
stories. In order to refresh the _Anteree's_ mendacity, the host thus
addressed the barber--

"O Es-Samit, thou silent one! it hath come to my ears that in thine
exceeding paucity of speech thou hast omitted, hitherto, to relate the
story of thy seventh brother. Since thou hast a seventh brother, let
not thy love of silence (in thee even greater than in thy famous
ancestor) deprive us of a knowledge of his depravity, but acquaint us
with his case."

"O Merchant Prince!" replied the barber, "to none other than
thyself--so handsome, so liberal, and of such excellent
morality--would I break my vow, to speak of that wretched villain,
that malevolent mule, that vilest of the vile, my twin brother
Ahzab."

My cousin, feigning astonishment at the manner of his speech, said--

"Thy twin brother, O Es-Samit, was not, like thee, a man of rectitude,
of exalted mind, and of enlightened intelligence?"

"Alas!" replied the barber, "he was a dog of the most mongrel kind. My
bosom is pierced when I utter his accursed name! At the hands of
Ahzab, my twin brother, I met with every indignity, and with penalties
of a most unfortunate description."

When the host heard this, he laughed exceedingly, saying--

"Acquaint us, O Es-Samit, with his shameless misdeeds."

The barber, sighing as though his soul sought rest from all earthly
afflictions, proceeded as follows:

       *       *       *       *       *

Know, O light of my eyes! that my other brother, Ahzab, was born in
the city of Cairo, and his birth was unattended by a darkening of the
sun and other unpleasant calamities only by reason of the fact that
_I_ was born in the same hour!

My twin brother, Ahzab, was blessed with handsome stature, an elegant
shape, a perfect figure, with cheeks like roses, with eyebrows meeting
above an aquiline nose brightly shining. In short, this shame of my
mother was endowed with all those perfections which Allah (whose name
be exalted) had also bestowed upon me; but his heart was the heart of
a serpent, and he lacked the nobility of mind which thou hast
observed in thy servant, O Paragon, of wisdom!

When we were yet in the bloom and blossom of handsome youth, a dispute
arose between us, and for many moons I saw not Ahzab, but pursued my
occupation as a barber and teller of wonderful stories in a distant
part of the city. In this way it befell that I knew of his state only
by report, until one day as I sat before my shop observing if the
ascendent of the hour were favorable to one who waited to be shaved,
there came to me a negro most handsomely dressed, who said:

"My Master, Ahzab the Merchant, desires that you repair as soon as
possible to his magazine. He hath urgent need of thee."

Upon hearing these words, and observing the richness of the negro's
apparel, I perceived that those reports which had come to me,
respecting Ahzab's wealth, were no more than true; and I spoke thus to
myself:

"Within the vilest heart may bloom the flower of brotherly affection.
Ahzab desires to share with me, the most enlightened of his family,
this good fortune which hath befallen him."

Accordingly, I shut up my shop, dismissing the one who waited to be
shaved, and followed the black to the Khân Khalîl, where were the
shops of the wealthy silk merchants. My brother received me
affectionately, embracing me and saying:

"O Es-Samit, ever have I loved thee. Lo! Thou growest more like myself
each year. Save that thou art more dignified and noble. Enter into
this private apartment with me, for it is important that no one shall
see thee."

Much surprised at his words, I followed him to an elegant apartment
above the shop, and there he ordered the servants to roast a lamb and
to bring to us fruit and wine; and while we thus pleasantly employed
ourselves, he unfolded to me his case.

"Know, O my brother, that I have accumulated great wealth; and this
I have done by observing those wise precepts of conduct laid down by
thee. By the charm of my speech, which I have fashioned upon thine,
and the elegance of my manner, in which I have, though poorly,
imitated thine own, and by the dignity and the modesty of my conduct,
I have endeared all hearts and am esteemed above all the other
merchants in Cairo.

"It is necessary that I repair to Damascus, and during my absence I
wish nothing better than that thou shouldst take my place here. This
will be favorable to both of us; for I will reward thy services with
five hundred piastres and an interest in my affairs, and thou wilt
pass for me; for all will say, 'Lo! Ahzab the Merchant waxes more
handsome each day; such is the benign influence of righteous
prosperity and conscious rectitude!' My affairs stand thus and thus,
and my steward, who will be in our confidences, will acquaint thee
with all matters necessary. Thou wilt wear my costly garments, and sit
in my shop. Each evening thou wilt secretly repair to thine own
abode."

Upon hearing those words, my bosom swelled with joy; for I observed
that Ahzab had not failed to perceive my exalted qualities. We sat
far into the night in conversation respecting our plans; and on the
following day, Ahzab having departed secretly for Damascus, I repaired
to his shop, as arranged, and took my seat there.

By the number of the persons who saluted me, and by the manner of
their speech, I perceived, more and more, the great prosperity of
my brother; and being of a thoughtful mind, I passed the days very
pleasantly in contemplation of my good fortune.

Upon the fourth day after the departure of my brother, as I sat in his
shop, there came past a damsel accompanied by female attendants. This
damsel was riding upon a mule with a richly embroidered saddle, with
stirrups of gold, and she was covered with an _izar_ of exquisite
fabric; and about her slender waist was a girdle of gold-embroidered
silk. I was stricken speechless with the beauty and elegance of her
form; and when she alighted and came into the shop, the odors of sweet
perfumes were diffused from her, and she captivated my reason by her
loveliness.

Seating herself beside me, she raised her _izar_, and I beheld her
black eyes. And they surpassed in beauty the eyes of all human beings,
and were like the eyes of the gazelle. She had a mouth like the Seal
of Suleyman, and hair blacker than the night of affliction; a forehead
like the new moon of Ramadan, and cheeks like anemones, with lips
fresher than rose petals, teeth like pearls from the sea of
distraction, and a neck surpassing in whiteness molten silver, above
a form that put to shame the willow branch.

She spoke to me, saying:

"O Ahzab! I have returned as I promised thee!"

At the sound of her voice, by Allah (whose name be exalted!) I was
entangled in the snare of her love; fire was burning up my heart on
her account; a consuming flame increased within my bosom, and my
reason was drowned in the sea of my desire.

Perceiving my state, she quickly lowered her veil in pretended
displeasure, and desired to look at some pieces of silk. While she
thus employed herself, she surpassed the branches in the beauty of
her bending motions, and my eyes could not remove themselves from
her. I thus communed with myself:

"O Es-Samit, thou didst contract with thy brother to do this and that,
and to render unto him a proper account of thy dealings. But though
he hath made thee no mention of his affair with this damsel--it is
important that thou conductest this matter as he would have done, so
that he cannot reproach thee with negligence!" For I was ever a just
as well as a discreet and silent man.

Accordingly I spoke as follows:

"O my mistress, who art the most lovely person God has created, rend
not my heart with thy displeasure, but take pity upon me. Know that
love is difficult, and the concealment of it melteth iron and
occasioneth disease and infirmity. Thou hast returned as thou didst
promise; therefore I conjure thee, conceal not thy face from thy
slave!"

The damsel thereupon raised her head and put aside her veil, casting
a glance upon me and looked sideways at the attendants, and placed one
finger upon her lips; so that I knew her to be as discreet as she was
lovely. She laughed in my face, and said:

"I will take this piece of embroidered silk that I have chosen. What
is the price?"

And I answered:

"One hundred piasters; but I pray thee let it be thine, and a gift
from Ahzab!"

Upon this, she looked into my eyes and the sight of her face drew from
me a thousand sighs, and took the silk, saying:

"O my master, leave me not desolate!"

So she departed, while I continued sitting in the market-street until
past the hour of afternoon prayer, with disturbed mind enslaved by her
beauty and loveliness. I returned to my house and supper was placed
before me, but reflecting upon the damsel, I could eat nothing. I laid
myself down to rest, but passed the whole night sleepless, communing
with myself how I could best carry out this affair and obtain
possession of the damsel ... for my brother, Ahzab!


II

Scarcely had daybreak appeared when I arose and repaired to the
market-place and put on a suit of my brother's clothing, richer and
more magnificent than that I had worn the day before; and having drunk
a cup of wine, I sat in the shop. But all that day she came not, nor
the next, but upon the third day she came again, attended only by one
attendant, and she saluted me and said in a speech never surpassed in
softness and sweetness:

"O my master, reproach me not that I thus reveal the interest I have
in thee, but I could not speak to thee when my women were in hearing;
and this one is in my confidence. I have told thee that my father will
never give me to thee because of my rank, but thou hast wounded my
heart, and more and more do I love thee each day--for each day thou
growest more beautiful and elegant. Forever I must be desolate. Alas!
I have placed thy letter in the box thou didst give me, and no day
passes that it is not wet with my tears. Farewell! O my beloved!"

On hearing this, my love and passion grew so violent that I almost
became insensible. The damsel rose to leave the shop, and the one who
was with her spoke softly in her ear; but she shook her head,
expressing displeasure, and went away.

When I perceived that indeed she was gone, verily the tears descended
upon my cheek like rain, and my soul had all but departed. My heart
clung to her--I followed in the direction of her steps through the
market-place, and lo! the attendant came running back to me, and said:

"Here is the message of my mistress: 'Know that my love is greater
than thine, and on Friday next my servant will come to thee and tell
thee how thou mayest see me for a short interview before my father
comes back from prayers.'"

When I heard these words of the girl, the anguish of my heart ceased,
and I was intoxicated with love and rapture, and in my joy and
longing, I omitted to ask the girl the abode of her mistress--neither
did I know the name of my beloved; but reflecting upon these matters,
I returned to my brother's shop, and sat there until late, and then I
repaired secretly to my abode.

I paused in a quiet street, and seated myself upon a _mastabah_ to
scent the coolness of the air, and to abandon myself to exquisite
reflections.

But no sooner had I thus seated myself than a negro of gigantic
stature, and most hideous aspect, suddenly appeared from the shadow
of a door, and threw himself upon me, exclaiming:

"This is thine end, as it was written, O Ahzab the Merchant!"

By Allah! (whose name be exalted) I thought it was even as he said;
and none but myself had fallen into sudden dissolution, but that
everything slippery is not a pancake, and the jar that is struck may
yet escape unbroken.

So it befell that by great good fortune and by the exercise of my
agility and intelligence, I tripped the negro and his head came in
contact with the _mastabah_, and before he could recover himself, I
held to his ebony throat the blade of a razor which, by the mercy of
God, and because it was a custom of my profession, I carried in my
_kamar_.

"O thou dog!" I exclaimed, "prepare to depart to that utter darkness
and perdition that awaits assassins! For assuredly I am about to slay
thee!"

But he humbled himself to the ground before me, and embraced my feet,
crying:

"Have mercy, O my master! I but obeyed the commands!"

"Of whom, thou vile and unnamable vermin?" I asked of him.

"Of whom else but Abu-el-Hassan, the son of the Kadî! For hath he not
revealed to thee that for what has passed with Jullanar (Pomegranate
Flower), the daughter of the Walî, he will slay thee?"

"He hath revealed this to me?" I asked of him, astonished at his
words.

And he replied: "Thou knowest, master, it was by my hand that the
message was borne."

Whereupon I praised Allah (whose name be exalted) and spurned the
slave with my foot, saying:

"Depart, O thou black son of filth, and report that I am dead. I give
thee thy wretched life; depart!"

But when he had gone, I again lifted up my voice in thanksgiving. And
having come to my abode, I performed the preparatory ablution, and
recited the prayer of night-fall; after which I recited the chapters
"Ya-Sîn" (The Cow) and "Two Preventatives." For I perceived that this
was the true purport of my brother's absence, and that in his love
and affection he had resigned to me this affair, well knowing that I
should perish!

It was by the mercy of Allah, the Compassionate, the Merciful, that my
case was not as he had foreseen. The damsel called Jullanar, daughter
of the Walî, was famed from Cairo to the uttermost islands of China
for her elegance and loveliness, and I knew that my beloved could be
none other than she, and that Abu-el-Hassan, son of the Kadî, could be
none other than the betrothed chosen of her father the Walî.

I slept not that night, but passed the hours until sunrise reflecting
upon this matter, and upon the dangers which awaited my father's
handsome son on Friday. And I went not to the market on the next day,
but sent a message to my brother's steward saying that I was smitten
with sickness and enjoining him to acquaint the girl, who presently
would come, where I was to be found.

Thus it befell that at noon on Friday the same girl that had been with
Jullanar came to me, sent thither from the shop of Ahzab by the
steward, saying:

"O my master, answer the summons of my mistress. This is the plan that
I have proposed to her: Conceal thyself within one of the large chests
that are in thy shop, and hire a porter to carry thee to the house of
the Walî. I will cause the _bowwab_ to admit the chest to the
apartment of the Lady Jullanar. She doth trust her honor to thy
discretion, by reason of her love for thee, and because she will die
if she see thee not to bid thee farewell. I will arrange for thee to
be secretly conveyed from the house, ere the Walî returns."

And at her words I was like to have swooned with ecstasy; and I
forgot, in the transport of love and delight, the black assassin and
the threatened vengeance of Abu-el-Hassan. I set at naught my fears at
trusting my father's favorite son within the walls of the Walî's
house. I thought only of Jullanar of the slender waist and heavy hips,
of the dewy lips, more intoxicating than wine, and the eyes of my
beloved like wells of temptation to swallow up the souls of men.

I shaved and went to the bath, and repaired to the shop of Ahzab. My
brother's steward was not there, whereat I rejoiced, and arrayed
myself in the most splendid suit that I could find, and having
perfumed myself with essences and sweet scents, I summoned a boy and
said:

"Go thou and bring here a porter. Order him to carry yon large chest
to the house of the Walî, near the Mosque of Ibn-Mizheh, and ask for
the lady Jullanar who hath purchased this box and a number of things
which are in it. See that he be a strong man, for the box is very
heavy."

The boy replied, "On the head," and departed on his errand.

Thereupon I commended my soul to Allah, and entered the box, closing
the lid upon me. Scarcely had I concealed myself, when the porter
entered and lifted the chest. The boy assisted him to take it upon
his back, and he bore it out into the market-street.

"Now by the beard of the Prophet (on whom be peace)," I exclaimed to
myself, "it is well that I am named Es-Samit, the Silent; for had it
been otherwise, I must have lifted up my voice against this son of
perdition who carries me with my soles raised to heaven!"

The porter conveyed me for some distance, panting beneath the weight
of the box, and, presently, coming to a _mastabah_, dropped one end
of the box upon it, whilst he rested himself.

"Now as Allah is great, and Mohammed his only prophet," I said in my
beard, "I am fortunate in that I have acquired a paucity of diction.
There is no other in Cairo, but the joy of my mother, that could
refrain from speech when dropped upon his skull on a stone bench!"

After a while, the porter raised the chest again, and resumed his
journey, presently coming to the house of the Walî, and dropping the
box into the courtyard.

"Allah be praised!" I said. "For if this porter, whose name be
accursed, did but carry me a quinary further, my silence would become
even more surprising than it is; for my affair would finish, and I
should speak no more to any man!"

The _bowwab_ now cried out:

"What is in this chest?"

"Purchases of the lady Jullanar," said the girl, whom I recognized
by her voice. "Permit the porter to carry it to her apartments."

"I must obey the orders of the Walî my master," replied the
door-keeper. "The box must be opened."

I was bereft of the power to control myself, and seized with a colic
from excess of fear; I almost died from the violent spasms of my
limbs.

"O Es-Samit!" I said, "this is the reward of him whom love leads to
the house of the Walî!"

I felt certain that my destruction approached. The intoxication of
love now ceased in me, and reflection came in its place. I repented
of what I had done, and prayed a happy solution of my dangerous case.

Whether as a result of my prayers, I know not, but some arrangement
was come to, and the porter once more raised the chest, and, striking
my head upon the end of it at each step, bore me up to the apartments
of Jullanar, which I thus entered feet first.

He deposited the box, lid downward, upon the soft mattress of a
_dîwan_, so that I found myself upon all fours, like a mule with my
face between my hands! Ere I could break my habitual silence, he
lifted some heavy piece of furniture--I know not what--and placed it
on top of the box!

A voice sweeter than the songs of the Daood spoke:

"Slave! what art thou doing!"

"I _am_ thy slave!" spoke another voice, at the accursed sound whereof
I almost died of spleen. "Knowest thou me not, my beloved? I have
devised a new stratagem and come to thee in the guise of a porter!
But lo! beneath my uncomely garments, I am Ahzab, thy lover!"


III

As a man who sleeps ill after a protracted feast, I heard her answer,
saying:

"Is it true thou hast come to me, or is this a dream?"

"Verily, it is true!" answered the accursed, the vile, the unspeakable
Ahzab, my brother--for it was he. "From the time when I first saw
thee, neither sleep hath been sweet to me, nor hath wine possessed the
slightest flavor! I have come to thee thus, fragrant bloom of the
pomegranate, because I would not have thee see me in a posture so
undignified as that of one crouched in a box! So that thy people might
be compelled to give me access to thine apartments, I have put a
mendicant in my place, rendering the chest heavy!"

And she said, "Thou art welcome!" and embraced him.

By Allah (whose name be exalted), I gnawed my beard until I choked!

"Thou art changed, beloved!" she said to him; "thou art always
beautiful, but to-day thou seemest less rosy-cheeked to mine eyes!"

The accursed Ahzab, like an enraged mule, kicked the box wherein
I dissolved in flames of wrath.

"I am burnt up with love and longing for thee!" he replied. "O my
love! how beautiful thou art!"

Whereat my command of silence forsook me! As Allah is the one god,
and Mohammed his only Prophet, I became as one possessed of a devil!

"Robber!" I cried; and my words lost themselves within the box.
"Cheat! accursed disgrace of my father! infamy of my race! O dog!
O unutterable dirt!"

Jullanar cried out in fear, but my accursed brother took her in his
bosom, soothing her with soft words.

"Fear not, O my beloved!" he said. "I gave the mendicant wine that
his heart might warm to his lowly task, but I fear he has become
intoxicated!"

"O thou liar!" I cried. "O malevolent scoundrel! O son of a disease!"
And with all my strength I sought to raise the weight that bore me
down; but to no purpose.

"Know, my beloved," continued my thrice-accursed brother, "what I have
suffered on thy account. But three days since I was attacked by four
gigantic negro assassins despatched by Abu-el-Hassan to slay me! But I
vanquished them, killing one and maiming a second, whilst the others
escaped and ran back to their wretched master."

"O unutterable liar!" I groaned. For I was near to hastening my
predestined end both from suffocation and consuming rage. "Thou didst
fly, thou jackal! from that peril, and reapest the fruits of my
courage and dexterity! O, mud! O, stench!"

"Lest he should despatch a number too great for me to combat, I have
lurked in hiding, delight of souls! in a most filthy hovel belonging
to a barber!"

"May thy tongue turn into a scorpion and bite thee!" I cried. "My
abode is as clean as the palace of the Khedive! Thou hast never
entered it, O thou gnat's egg! Thou hast hidden in I know not what
hole, like the unclean insect thou art, until thy steward (may his
beard grow backward and smother him!) informed thee of this! O Allah!
(to whom be ascribed all might and glory) give me strength to move
this accursed box that I may crush him!"

Scarce had I uttered the last word, when a girl came running into the
apartment, crying: "Fly, my master! O my mistress! The Walî! the
Walî!"

Upon hearing these words, my rage departed from me and in its place
came excessive fear. My breath left my body, and my heart ceased to
beat.

"He that falleth in the dirt be trodden on by camels," I reflected.
"It is not enough, O Es-Samit, that thou hast suffered the attack of
the assassin; that thou hast all but died of fear at the door of the
Walî's house; that thou hast been torn from the arms of the loveliest
creature God hath created; thou are destined, now, O most unfortunate
of men, to be detected by the Walî in his daughter's apartments,
concealed in a box!"

And I pronounced the _Takbîr_, crying, "O Allah! thy ways are
inscrutable!"

"Fly, my beloved!" cried Jullanar to Ahzab. "My women will conceal
thee!" Wherewith she swooned and fell upon the floor senseless.

"Quick! follow me closely, O my master!" cried the girl, and I heard
my perfidious brother depart from the room by one door, as the Walî
entered by another.

"Ah!" cried the Walî, clapping his hands. "Slaves! what is this?"

And people came running to his command; some carrying out the lady
Jullanar to her sleeping apartment, and sprinkling rose-water upon
her, and some remaining.

"What is in this box upon the _dîwan_!" demanded the Walî. "Bring it
hither and open it!"

At that I knew that I was lost, and my soul as good as departed, and
I bade farewell to life and invoked Mohammed (whom may God preserve)
to intercede for me that I might die an easy death.

The chest was dragged into the middle of the floor and thrown open.

"Name of my mother!" exclaimed the Walî. "It is Ahzab the Merchant! It
is the villain who hath presumed to make love to my daughter! O Allah!
my daughter hath disgraced me! By the beard of the Prophet, I can no
more hold up my head among honest men!"

And he slapped his face and plucked his beard, and fell insensible
upon the floor. As he did so, I leaped from the box and would have
escaped, but two blacks seized me; and the noise, or the refreshing
quality of the rose-water with which the women were sprinkling him,
revived the Walî, who recovered, fixing upon me a terrible gaze.

"O thou dog!" he said; "thou who hast wrought my disgrace! As thou
didst enter my house in yonder box, in yonder box shalt thou quit the
world! Cast him back again, fasten the box with ropes, and throw it
into the Nile at nightfall!"


IV

Now were my powers of silence most surprisingly displayed. For I spoke
no word, but dumb as a tongueless man, I allowed myself to be knocked
backward into the box. The lid closed upon me, ropes bound about the
box, and the seal of the Walî affixed to it. Negroes carried it out,
and threw it into some cellar to await nightfall.

"O Es-Samit!" I said, "this is the end that was appointed to thy
father's wisest son! To this pass thy silence and wisdom have brought
thee! O Allah (to whom be all glory), grant that one of the fishes
that eat me in the Nile shall be served up to Ahzab, my twin brother,
and choke him!"

And then my thoughts turned to Jullanar, and I sighed and groaned;
and the torments I suffered through lying drawn up in the box were
delights to the agonies that my reflections respecting her case
occasioned in me; so that, with the excess of my woe and misery,
I became insensible. How long I remained so I know not, but I was
awakened by a knocking at the lid of the box, and the voice of the
Walî spoke, saying:

"Prepare to die, O wretch! for my servants are about to convey thee
to the river and cast thee in! Thou dog! who didst presume to raise
thine eyes to my daughter!--know that this is the reward of such
malefactors; for assuredly if thou escapest alive, thou shalt wed
Jullanar!"

Whereat he laughed until he almost swooned and kicked the box until
I thought he had burst it. Blacks raised me, and I was borne down a
long flight of steps and onward in I know not what direction.

"From here?" said one of them, and through a crack in the lid, I saw
the light of a torch, and the whispering of the river came to my ears.

"Yes!" replied another.

And I commended my soul to Allah as the box was swung to and fro and
hurled through the air. With a sound in my ears as of the shrieking
of ten thousand _efreets_, I was plunged into the water!

Far under the surface I went and knew all the agonies of dissolution;
but the box was strongly and cunningly made and rose again; then it
began to fill and sink once more, and again I tasted of the final
pangs. Throughout all this time, a strong current was bearing the box
along, and presently, as, for the fiftieth occasion, I was seeking to
die and to end my misery, I heard voices.

The most miserable life is sweet to him who feels it slipping from his
grasp, and I summoned sufficient strength to raise a feeble cry.

"O Allah!" I cried, "if it be thy will, grant that these persons whose
voices I hear take pity upon my unfortunate condition, and draw me
forth."

Even as I spoke, something stayed the onward progress of the box. It
was a fisherman's net! And the fishermen began to draw me into the
boat, I praising Allah the while.

But when they had the box upon the edge of the boat, and heard my
voice proceeding from within, and saw the Walî's seal upon the
lid--"By the beard of the Prophet!" cried one, "this is some evil
_ginn_ or magician whom the Walî hath imprisoned in this chest! Allah
avert the omen! Cast him back, comrades!"

Alas! I could find no words wherewith to entreat them to take pity;
never had paucity of speech served me so ill! A great groan issued
from my bosom as I was consigned again to the Nile!

Allah is great, and it was not written that I should perish in that
manner. For another current now seized upon the box, and just as I was
on the point of dissolution, cast it upon a projecting bank, where it
was perceived by a band of four robbers, who derived a livelihood from
plundering such vessels as lay unprotected in the river.

These waded out and dragged the box ashore. I was too near my end to
have spoken had I desired to speak, but from my unfortunate adventure
with the fishermen, I had learned that silence was wisdom, now as
always. Thus I lay in the box like a dog that has been all but
drowned, and listened to the words of my rescuers.

These were arguing respecting the contents and value of the box, one
holding this opinion and another that. One, who seemed to be their
leader, was about to unfasten the ropes, but another claimed that
this was his due. So, from angry words, they came to blows, and by the
grace of God (whose name be exalted) they drew their knives, and three
of the four were slain. The fourth removed the ropes and opened the
box, thinking to enjoy, alone, the treasures which he supposed it to
contain.

Whereupon I uprose and looked up to where Canopus shone, and said:

"There is no God but God! Praise be to Allah who has preserved me
from an unfortunate and unseemly end!"

At that, the robber, with wild cries of fear, turned and ran, and I
saw him no more. Such, O bountiful patron, is the disgraceful story
of the dog Ahzab, my seventh and twin brother. But all that which I
endured happened by Fate and Destiny, and from that which is written
there is no escape nor flight.

       *       *       *       *       *

Our worthy host (concluded Hassan) laughed heartily at this story,
saying:

"O Es-Samit, it is evident to me that thy paucity of speech alone
preserved thee from drowning! But acquaint us, I beg, with the fate
of thy dog of a brother, and of thy beautiful Pomegranate Flower."

"O glory of beholders!" replied the barber, "by the mouth of the girl
who was in Jullanar's confidence--Ahzab, that shame of mules, learned,
whilst in hiding, how the Walî had said in the presence of many
witnesses: 'Assuredly if thou escapest alive, thou shalt wed
Jullanar.'"

"Tellest thou me that he had the effrontery to demand the fulfilment
of a pledge so spoken, O Es-Samit?"

"Alas!" replied the barber, with tears pouring like rain down the
wrinkles of his aged cheek, "he lived with her the most joyous, and
most agreeable, and most comfortable, and most pleasant life, until
they were visited by the terminator of delights, and the separator
of companions!"


     THE END.




     *       *       *       *       *
         *       *       *       *




Transcriber's Note:


The following printer's errors have been corrected in the text:

  Page 16, passsd amended to passed (passed through the finger ring)
  Page 16, instrinsic amended to intrinsic (Its intrinsic value)
  Page 20, whcih amended to which (for which I would gladly)
  Page 26, wordly amended to worldly (terminated my worldly affairs)
  Page 42, remakable amended to remarkable (taciturnity was remarkable)
  Page 42, dsitinguished amended to distinguished (behavior was
    distinguished)
  Page 46, exhausing amended to exhausting (journey had been most
    exhausting)
  Page 87, Suit amended to Suite (Harêm Suite is occupied)
  Page 89, releived amended to relieved (relieved by a small snowy
    turban)
  Page 98, bougainvillia replaced with bougainvillea (bougainvillea
    blossom)
  Page 104, veiled amended to Veiled (features of the Veiled Prophet)
  Page 112, Chundermyer amended to Chundermeyer (Mr. Chundermeyer
    gagged)
  Page 115, partciular compositoin amended to particular composition
    (particular composition of the perfume)
  Page 130, oportunity ammended to opportunity (the seeming opportunity)
  Page 130, cheecks amended to cheeks (my cheeks were burning)
  Page 149, Abî ammended to Abû (Abû Tabâh assured me)
  Page 149, alhtough amended to although (although she who is)
  Page 152, folowing amended to following (on the following morning)
  Page 158, met amended to me (invited me to visit)
  Page 160, acess amended to access (door giving access to)
  Page 175, pantomine amended to pantomime (grotesque kind of pantomime)
  Page 222, tesitfy amended to testify (as I can testify)
  Page 243, cresent amended to crescent (crescent of Islâm)
  Page 250, menioned amended to mentioned (which I have mentioned)
  Page 275, severly amended to severely (The features were severely)
  Page 280, incliniation amended to inclination (I felt little
    inclination)
  Page 301, sumons amended to summons (summons of my mistress)
  Page 306, dexerity amended to dexterity (my courage and dexterity)
  Page 311, bossom amended to bosom (groan issued from my bosom)



*** END OF THE PROJECT GUTENBERG EBOOK TALES OF SECRET EGYPT ***




Updated editions will replace the previous one—the old editions will
be renamed.

Creating the works from print editions not protected by U.S. copyright
law means that no one owns a United States copyright in these works,
so the Foundation (and you!) can copy and distribute it in the United
States without permission and without paying copyright
royalties. Special rules, set forth in the General Terms of Use part
of this license, apply to copying and distributing Project
Gutenberg™ electronic works to protect the PROJECT GUTENBERG™
concept and trademark. Project Gutenberg is a registered trademark,
and may not be used if you charge for an eBook, except by following
the terms of the trademark license, including paying royalties for use
of the Project Gutenberg trademark. If you do not charge anything for
copies of this eBook, complying with the trademark license is very
easy. You may use this eBook for nearly any purpose such as creation
of derivative works, reports, performances and research. Project
Gutenberg eBooks may be modified and printed and given away—you may
do practically ANYTHING in the United States with eBooks not protected
by U.S. copyright law. Redistribution is subject to the trademark
license, especially commercial redistribution.


START: FULL LICENSE

THE FULL PROJECT GUTENBERG LICENSE

PLEASE READ THIS BEFORE YOU DISTRIBUTE OR USE THIS WORK

To protect the Project Gutenberg™ mission of promoting the free
distribution of electronic works, by using or distributing this work
(or any other work associated in any way with the phrase “Project
Gutenberg”), you agree to comply with all the terms of the Full
Project Gutenberg™ License available with this file or online at
www.gutenberg.org/license.

Section 1. General Terms of Use and Redistributing Project Gutenberg™
electronic works

1.A. By reading or using any part of this Project Gutenberg™
electronic work, you indicate that you have read, understand, agree to
and accept all the terms of this license and intellectual property
(trademark/copyright) agreement. If you do not agree to abide by all
the terms of this agreement, you must cease using and return or
destroy all copies of Project Gutenberg™ electronic works in your
possession. If you paid a fee for obtaining a copy of or access to a
Project Gutenberg™ electronic work and you do not agree to be bound
by the terms of this agreement, you may obtain a refund from the person
or entity to whom you paid the fee as set forth in paragraph 1.E.8.

1.B. “Project Gutenberg” is a registered trademark. It may only be
used on or associated in any way with an electronic work by people who
agree to be bound by the terms of this agreement. There are a few
things that you can do with most Project Gutenberg™ electronic works
even without complying with the full terms of this agreement. See
paragraph 1.C below. There are a lot of things you can do with Project
Gutenberg™ electronic works if you follow the terms of this
agreement and help preserve free future access to Project Gutenberg™
electronic works. See paragraph 1.E below.

1.C. The Project Gutenberg Literary Archive Foundation (“the
Foundation” or PGLAF), owns a compilation copyright in the collection
of Project Gutenberg™ electronic works. Nearly all the individual
works in the collection are in the public domain in the United
States. If an individual work is unprotected by copyright law in the
United States and you are located in the United States, we do not
claim a right to prevent you from copying, distributing, performing,
displaying or creating derivative works based on the work as long as
all references to Project Gutenberg are removed. Of course, we hope
that you will support the Project Gutenberg™ mission of promoting
free access to electronic works by freely sharing Project Gutenberg™
works in compliance with the terms of this agreement for keeping the
Project Gutenberg™ name associated with the work. You can easily
comply with the terms of this agreement by keeping this work in the
same format with its attached full Project Gutenberg™ License when
you share it without charge with others.

1.D. The copyright laws of the place where you are located also govern
what you can do with this work. Copyright laws in most countries are
in a constant state of change. If you are outside the United States,
check the laws of your country in addition to the terms of this
agreement before downloading, copying, displaying, performing,
distributing or creating derivative works based on this work or any
other Project Gutenberg™ work. The Foundation makes no
representations concerning the copyright status of any work in any
country other than the United States.

1.E. Unless you have removed all references to Project Gutenberg:

1.E.1. The following sentence, with active links to, or other
immediate access to, the full Project Gutenberg™ License must appear
prominently whenever any copy of a Project Gutenberg™ work (any work
on which the phrase “Project Gutenberg” appears, or with which the
phrase “Project Gutenberg” is associated) is accessed, displayed,
performed, viewed, copied or distributed:

    This eBook is for the use of anyone anywhere in the United States and most
    other parts of the world at no cost and with almost no restrictions
    whatsoever. You may copy it, give it away or re-use it under the terms
    of the Project Gutenberg License included with this eBook or online
    at www.gutenberg.org. If you
    are not located in the United States, you will have to check the laws
    of the country where you are located before using this eBook.

1.E.2. If an individual Project Gutenberg™ electronic work is
derived from texts not protected by U.S. copyright law (does not
contain a notice indicating that it is posted with permission of the
copyright holder), the work can be copied and distributed to anyone in
the United States without paying any fees or charges. If you are
redistributing or providing access to a work with the phrase “Project
Gutenberg” associated with or appearing on the work, you must comply
either with the requirements of paragraphs 1.E.1 through 1.E.7 or
obtain permission for the use of the work and the Project Gutenberg™
trademark as set forth in paragraphs 1.E.8 or 1.E.9.

1.E.3. If an individual Project Gutenberg™ electronic work is posted
with the permission of the copyright holder, your use and distribution
must comply with both paragraphs 1.E.1 through 1.E.7 and any
additional terms imposed by the copyright holder. Additional terms
will be linked to the Project Gutenberg™ License for all works
posted with the permission of the copyright holder found at the
beginning of this work.

1.E.4. Do not unlink or detach or remove the full Project Gutenberg™
License terms from this work, or any files containing a part of this
work or any other work associated with Project Gutenberg™.

1.E.5. Do not copy, display, perform, distribute or redistribute this
electronic work, or any part of this electronic work, without
prominently displaying the sentence set forth in paragraph 1.E.1 with
active links or immediate access to the full terms of the Project
Gutenberg™ License.

1.E.6. You may convert to and distribute this work in any binary,
compressed, marked up, nonproprietary or proprietary form, including
any word processing or hypertext form. However, if you provide access
to or distribute copies of a Project Gutenberg™ work in a format
other than “Plain Vanilla ASCII” or other format used in the official
version posted on the official Project Gutenberg™ website
(www.gutenberg.org), you must, at no additional cost, fee or expense
to the user, provide a copy, a means of exporting a copy, or a means
of obtaining a copy upon request, of the work in its original “Plain
Vanilla ASCII” or other form. Any alternate format must include the
full Project Gutenberg™ License as specified in paragraph 1.E.1.

1.E.7. Do not charge a fee for access to, viewing, displaying,
performing, copying or distributing any Project Gutenberg™ works
unless you comply with paragraph 1.E.8 or 1.E.9.

1.E.8. You may charge a reasonable fee for copies of or providing
access to or distributing Project Gutenberg™ electronic works
provided that:

    • You pay a royalty fee of 20% of the gross profits you derive from
        the use of Project Gutenberg™ works calculated using the method
        you already use to calculate your applicable taxes. The fee is owed
        to the owner of the Project Gutenberg™ trademark, but he has
        agreed to donate royalties under this paragraph to the Project
        Gutenberg Literary Archive Foundation. Royalty payments must be paid
        within 60 days following each date on which you prepare (or are
        legally required to prepare) your periodic tax returns. Royalty
        payments should be clearly marked as such and sent to the Project
        Gutenberg Literary Archive Foundation at the address specified in
        Section 4, “Information about donations to the Project Gutenberg
        Literary Archive Foundation.”

    • You provide a full refund of any money paid by a user who notifies
        you in writing (or by e-mail) within 30 days of receipt that s/he
        does not agree to the terms of the full Project Gutenberg™
        License. You must require such a user to return or destroy all
        copies of the works possessed in a physical medium and discontinue
        all use of and all access to other copies of Project Gutenberg™
        works.

    • You provide, in accordance with paragraph 1.F.3, a full refund of
        any money paid for a work or a replacement copy, if a defect in the
        electronic work is discovered and reported to you within 90 days of
        receipt of the work.

    • You comply with all other terms of this agreement for free
        distribution of Project Gutenberg™ works.


1.E.9. If you wish to charge a fee or distribute a Project
Gutenberg™ electronic work or group of works on different terms than
are set forth in this agreement, you must obtain permission in writing
from the Project Gutenberg Literary Archive Foundation, the manager of
the Project Gutenberg™ trademark. Contact the Foundation as set
forth in Section 3 below.

1.F.

1.F.1. Project Gutenberg volunteers and employees expend considerable
effort to identify, do copyright research on, transcribe and proofread
works not protected by U.S. copyright law in creating the Project
Gutenberg™ collection. Despite these efforts, Project Gutenberg™
electronic works, and the medium on which they may be stored, may
contain “Defects,” such as, but not limited to, incomplete, inaccurate
or corrupt data, transcription errors, a copyright or other
intellectual property infringement, a defective or damaged disk or
other medium, a computer virus, or computer codes that damage or
cannot be read by your equipment.

1.F.2. LIMITED WARRANTY, DISCLAIMER OF DAMAGES - Except for the “Right
of Replacement or Refund” described in paragraph 1.F.3, the Project
Gutenberg Literary Archive Foundation, the owner of the Project
Gutenberg™ trademark, and any other party distributing a Project
Gutenberg™ electronic work under this agreement, disclaim all
liability to you for damages, costs and expenses, including legal
fees. YOU AGREE THAT YOU HAVE NO REMEDIES FOR NEGLIGENCE, STRICT
LIABILITY, BREACH OF WARRANTY OR BREACH OF CONTRACT EXCEPT THOSE
PROVIDED IN PARAGRAPH 1.F.3. YOU AGREE THAT THE FOUNDATION, THE
TRADEMARK OWNER, AND ANY DISTRIBUTOR UNDER THIS AGREEMENT WILL NOT BE
LIABLE TO YOU FOR ACTUAL, DIRECT, INDIRECT, CONSEQUENTIAL, PUNITIVE OR
INCIDENTAL DAMAGES EVEN IF YOU GIVE NOTICE OF THE POSSIBILITY OF SUCH
DAMAGE.

1.F.3. LIMITED RIGHT OF REPLACEMENT OR REFUND - If you discover a
defect in this electronic work within 90 days of receiving it, you can
receive a refund of the money (if any) you paid for it by sending a
written explanation to the person you received the work from. If you
received the work on a physical medium, you must return the medium
with your written explanation. The person or entity that provided you
with the defective work may elect to provide a replacement copy in
lieu of a refund. If you received the work electronically, the person
or entity providing it to you may choose to give you a second
opportunity to receive the work electronically in lieu of a refund. If
the second copy is also defective, you may demand a refund in writing
without further opportunities to fix the problem.

1.F.4. Except for the limited right of replacement or refund set forth
in paragraph 1.F.3, this work is provided to you ‘AS-IS’, WITH NO
OTHER WARRANTIES OF ANY KIND, EXPRESS OR IMPLIED, INCLUDING BUT NOT
LIMITED TO WARRANTIES OF MERCHANTABILITY OR FITNESS FOR ANY PURPOSE.

1.F.5. Some states do not allow disclaimers of certain implied
warranties or the exclusion or limitation of certain types of
damages. If any disclaimer or limitation set forth in this agreement
violates the law of the state applicable to this agreement, the
agreement shall be interpreted to make the maximum disclaimer or
limitation permitted by the applicable state law. The invalidity or
unenforceability of any provision of this agreement shall not void the
remaining provisions.

1.F.6. INDEMNITY - You agree to indemnify and hold the Foundation, the
trademark owner, any agent or employee of the Foundation, anyone
providing copies of Project Gutenberg™ electronic works in
accordance with this agreement, and any volunteers associated with the
production, promotion and distribution of Project Gutenberg™
electronic works, harmless from all liability, costs and expenses,
including legal fees, that arise directly or indirectly from any of
the following which you do or cause to occur: (a) distribution of this
or any Project Gutenberg™ work, (b) alteration, modification, or
additions or deletions to any Project Gutenberg™ work, and (c) any
Defect you cause.

Section 2. Information about the Mission of Project Gutenberg™

Project Gutenberg™ is synonymous with the free distribution of
electronic works in formats readable by the widest variety of
computers including obsolete, old, middle-aged and new computers. It
exists because of the efforts of hundreds of volunteers and donations
from people in all walks of life.

Volunteers and financial support to provide volunteers with the
assistance they need are critical to reaching Project Gutenberg™’s
goals and ensuring that the Project Gutenberg™ collection will
remain freely available for generations to come. In 2001, the Project
Gutenberg Literary Archive Foundation was created to provide a secure
and permanent future for Project Gutenberg™ and future
generations. To learn more about the Project Gutenberg Literary
Archive Foundation and how your efforts and donations can help, see
Sections 3 and 4 and the Foundation information page at www.gutenberg.org.

Section 3. Information about the Project Gutenberg Literary Archive Foundation

The Project Gutenberg Literary Archive Foundation is a non-profit
501(c)(3) educational corporation organized under the laws of the
state of Mississippi and granted tax exempt status by the Internal
Revenue Service. The Foundation’s EIN or federal tax identification
number is 64-6221541. Contributions to the Project Gutenberg Literary
Archive Foundation are tax deductible to the full extent permitted by
U.S. federal laws and your state’s laws.

The Foundation’s business office is located at 809 North 1500 West,
Salt Lake City, UT 84116, (801) 596-1887. Email contact links and up
to date contact information can be found at the Foundation’s website
and official page at www.gutenberg.org/contact

Section 4. Information about Donations to the Project Gutenberg
Literary Archive Foundation

Project Gutenberg™ depends upon and cannot survive without widespread
public support and donations to carry out its mission of
increasing the number of public domain and licensed works that can be
freely distributed in machine-readable form accessible by the widest
array of equipment including outdated equipment. Many small donations
($1 to $5,000) are particularly important to maintaining tax exempt
status with the IRS.

The Foundation is committed to complying with the laws regulating
charities and charitable donations in all 50 states of the United
States. Compliance requirements are not uniform and it takes a
considerable effort, much paperwork and many fees to meet and keep up
with these requirements. We do not solicit donations in locations
where we have not received written confirmation of compliance. To SEND
DONATIONS or determine the status of compliance for any particular state
visit www.gutenberg.org/donate.

While we cannot and do not solicit contributions from states where we
have not met the solicitation requirements, we know of no prohibition
against accepting unsolicited donations from donors in such states who
approach us with offers to donate.

International donations are gratefully accepted, but we cannot make
any statements concerning tax treatment of donations received from
outside the United States. U.S. laws alone swamp our small staff.

Please check the Project Gutenberg web pages for current donation
methods and addresses. Donations are accepted in a number of other
ways including checks, online payments and credit card donations. To
donate, please visit: www.gutenberg.org/donate.

Section 5. General Information About Project Gutenberg™ electronic works

Professor Michael S. Hart was the originator of the Project
Gutenberg™ concept of a library of electronic works that could be
freely shared with anyone. For forty years, he produced and
distributed Project Gutenberg™ eBooks with only a loose network of
volunteer support.

Project Gutenberg™ eBooks are often created from several printed
editions, all of which are confirmed as not protected by copyright in
the U.S. unless a copyright notice is included. Thus, we do not
necessarily keep eBooks in compliance with any particular paper
edition.

Most people start at our website which has the main PG search
facility: www.gutenberg.org.

This website includes information about Project Gutenberg™,
including how to make donations to the Project Gutenberg Literary
Archive Foundation, how to help produce our new eBooks, and how to
subscribe to our email newsletter to hear about new eBooks.
